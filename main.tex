%   Environment     :   Ubuntu 24.04 LTS ~ "Noble Numbat"
%                       Gnome 46
%                       TeXLive 2023
%
%   Platform        :   x86_64-pc-linux-gnu (64bits)
%
%   Author          :   Bylli Andre Guel ~ byliguel@gmail.com
%
%   File            :   main.tex
%
%   Description     :   Main TeX source for research paper on the
%   retrograde heat equation
%       ~/Documents/research/postDoc-projects/retrograde-heat-equation/
%
%   Created date    :   2024-06-11 ~ 17:01
%   Version         :   v24.06.0
%   Last revision   :   ...
%=========================================================================

\documentclass[a4paper,12pt]{article}

\usepackage[utf8]{inputenc}
\usepackage[T1]{fontenc}
\usepackage[english,french]{babel}

\usepackage[left=2cm,right=2cm,top=2cm,bottom=2cm]{geometry}

\usepackage{amsmath,amssymb,amsfonts,mathrsfs,amsthm,shortcommands}
\usepackage{textcomp,pifont,authblk,enumitem}

\usepackage{biblatex}
\addbibresource{biblio.bib}
\usepackage{csquotes}

\newtheorem{definition}{Definition}
\newtheorem{proposition}{Proposition}
\newtheorem{corollaire}{Corollary}
\newtheorem{lemme}{Lemma}
\newtheorem{remarque}{Remark}
\newtheorem{theoreme}{Theorem}

\setlist[itemize,1]{label=\tiny\Pisymbol{pzd}{110}}
\setlist[itemize,2]{label=\tiny$\blacktriangleright$}
\setlist[itemize,3]{label=\tiny\textbullet}

\usepackage[hidelinks]{hyperref}
\hypersetup{%
    citecolor=black,%
    filecolor=black,%
    colorlinks=true,%
    linkcolor=black,%
    urlcolor=black,%
    pdftitle={Control of Singular Distributed Systems by Controllability:
    The Ill-Posed Backwards Heat Equation}
}

\title{Control of Singular Distributed Systems by Controllability: The
Ill-Posed Backwards Heat Equation}

\author{Bylli André Guel}

% \author[,1,2]{Bylli
% André~\textsc{Guel}\thanks{bylli.guel@uv.bf\,/\,01~BV~30770~Ouagadougou~01,~Burkina~Faso}}
% \affil[1]{Université Virtuelle du Burkina Faso,\authorcr{} Ouagadougou,
% Burkina~Faso}
% \affil[2]{Laboratoire d'Analyse Numérique, d'Informatique et de
% BIOmathématiques (LANIBIO), Burkina Faso}

% \author[2,3]{Elie~\textsc{Ouedraogo}}
% \author[2,3]{Sadou~\textsc{Tao}}

% \affil[3]{Université Joseph Ki-Zerbo, Département de Mathématiques}

\begin{document}

\selectlanguage{english}

\parindent=0pt{}
\parskip=6pt{}

\maketitle{}

\begin{abstract}
    This paper deals with the ill-posed backwards heat equation. To do
    this, we propose the controllability method. The point of view adopted,
    which consists in interpreting the state equation as an inverse problem,
    allows us to obtain a decoupled and strong singular optimality system
    for the optimal control-state pair, but also to propose an existence
    criteria for a regular solution of the backwards heat equation. It is
    important to note that this results are obtained without recourse to a
    Slater-type assumption, an assumption to which many analyses have had
    to recourse.

    \vspace{\baselineskip}
    \textbf{Keywords:} Singular distributed systems, Optimal control,
    Ill-posed backwards heat equation, Controllability, Inverse Problem.
\end{abstract}

\section{Introduction}

Let us consider an open and bounded subset $\Omega$ of ${\Rbb}^{n}$, $n\in
{\mathbb{N}}^{*}$, of boundary ${\Gamma}$, twice continuously
differentiable, with $\Omega$ locally on one side only of $\Gamma$; that is
to say that $\bar{\Omega}$ is a variety with a boundary of class
${\Cscr}^{2}$.

For $T > 0$, we denote $\displaystyle Q = \Omega\times(0,T)$, $\displaystyle
\Sigma = \Gamma\times (0,T)$, and $\displaystyle {\Uscr}_{ad}$ a closed and
non-empty convex of $\Ldeux{Q}$.

For $v\in\Ldeux{Q}$, we on consider the following problem

\begin{equation}\label{eq:initial-problem}
    \begin{cases}\displaystyle
        \begin{array}{rclll}
            \primetemps{z} - \Delta z & = & v & \text{in} & Q,\\
            \\
            z & = & 0 & \text{on} & \Sigma,\\
            \\
            \condfinale{z} & = & 0 & \text{in} & \Omega,
        \end{array}
    \end{cases}
\end{equation}
being interested by the evolution, in $\Omega$ and at a time $t\in (0,T)$,
of the temperature $z$ knowing its value at the final time $T$.

It is well known that problem~\eqref{eq:initial-problem}, so called the
ill-posed backwards heat equation, does not admits solution for any initial
data.

So we consider, a priori, pairs $(v,z)\in {\left({\Ldeux{Q}}\right)}^{2}$
satisfying~\eqref{eq:initial-problem}, saying that such pairs constitute
the set of control-state pairs.

A control-state pair $(v,z)$ for~\eqref{eq:initial-problem} will be said
admissible if
\begin{equation*}\label{eq:admissible-control-state-pair}
    v\in {\Uscr}_{ad};
\end{equation*}
and we will denote $(v,z)\in \Ascr$, designating by $\Ascr$ the set of
optimal control-state pairs.

Supposing that the set $\Ascr$ of admissible control-state pairs is
non-empty, we introduce the cost function
\begin{equation*}\label{eq:initial-cost-function}
    J(v,z) = \dfrac{1}{2}\normecq{z - z_{d}} + \dfrac{N}{2}\normecq{v},
\end{equation*}
where $z_{d}\in\Ldeux{Q}$ and $N > 0$.

Then, we are interested in the control problem which consists in finding
the control-state pair
\begin{equation}\label{eq:initial-control-problem}
    (u,y)\in {\Ascr}_{ad}:\quad J(u,y) = \inf_{(v,z)\in
    {\Ascr}_{ad}}\,J(v,z).
\end{equation}

The following result in then immediate
\begin{theoreme}%
    The optimal control problem~\eqref{eq:initial-control-problem} admits a
    unique solution $(u,y)$, called the optimal control-state pair.
\end{theoreme}

\begin{proof}%
    Due to its structure, the cost functional $J$ is coercive, strictly
    convex and lower semi-continuous. From where, with the non-vacuity
    assumption of the closed convex set of admissible control-state pairs
    $\Ascr$, we can conclude to existence and unicity of the optimal
    control-state pair $(u,y)$.
\end{proof}

Hence, one can establish, (cf.~\cite{lions2}), using the first-order
Euler-Lagrange optimality condition, that the optimal control-state pair
$(u,y)$ is characterized by
\begin{equation*}\label{eq:initial-variational-inequation}
    \begin{array}{rcl}
        \scalaireq{%
            y - z_{d}
        }{%
            z - y
        } + N\scalaireq{u}{v - u} & \geq & 0,\qquad \forall\,(v,z)\in\Ascr.
    \end{array}
\end{equation*}
We now focus on the characterization of the optimal pair $(u,y)$ through a
strong and decoupled singular optimality system.

A classic method to do so is the penalization method introduced by
J.~L.~Lions in~\cite{lions2}. This one consists of approaching the optimal
control-state pair $(u,y)$ by a penalized problem. More precisely, for
$\veps > 0$, we define the penalized cost function
\begin{equation*}\label{eq:penalized-cost-function}
    J_{\veps}\!\left({v,z}\right) = J(v,z) + \dfrac{1}{\veps}\normecq{%
        \primetemps{z} - \Delta z - v
    }.
\end{equation*}
We then establish that the optimal control-state pair
$\left({u_{\veps},z_{\veps}}\right)$ corresponding to this last cost
function converges towards the optimal control-state pair $(u,y)$. We
therefore have a theoretical approximation process for the optimal pair
$(u,y)$. Thanks to the first-order Euler-Lagrange optimality condition, we
characterize the approached optimal control-state pair
$\left({u_{\veps},y_{\veps}}\right)$ by a system of variational
inequalities which we interpret as an approached optimality system after
introducing the approached adjoint state
\begin{equation*}
    p_{\veps} = -\dfrac{1}{\veps}\left({%
        \primetemps{y_{\veps}} - \Delta y_{\veps} - u_{\veps}
    }\right).
\end{equation*}
The main step of this technique is the use of the \textit{a priori}
estimation method to obtain a strongly decoupled optimality system by
passing to the limit in the approached optimality system. But this final
step requires the Slater-type assumption that
\begin{equation}\label{eq:slater-type-assumption}
    \text{“the set of admissible controls }{\Uscr}_{ad}\ \text{is of non
    empty interior in}\ \Ldeux{Q}\text{”}.
\end{equation}
But in many situations, the Slater-type assumption does not hold; for
example in the case
\begin{equation*}
    \uad = {\left({\Ldeux{Q}}\right)}^{+} = \left\{{%
        v\in\Ldeux{Q}:\ v\geq 0
    }\right\}.
\end{equation*}
It therefore appears relevant to know how to do without this unrealistic
assumption.

For this purpose,  R.~Dorville, O.~Nakoulima et A.~Omrane propose
in~\cite{dorville}, the notion of least regrets control, applied to a
regularized elliptic state equation with missing data. An approach that
allows them to characterize the least regrets control $u^{\gamma}$ through
a strongly decoupled singular optimality system. This, certainly without
recourse to the Slater-type assumption~\eqref{eq:slater-type-assumption},
but to the detriment of the final condition
\begin{equation*}
    \begin{array}{rclll}
        \condfinale{y^{\gamma}} & = & 0 & \text{in} & \Omega.
    \end{array}
\end{equation*}
That is to say, more precisely, that the state $y^{\gamma}$ associated with
the least regrets control $u^{\gamma}$ is such that, in all generality,
\begin{equation*}
    \begin{array}{rclll}
        \condfinale{y^{\gamma}} & \neq & 0 & \text{in} & \Omega.
    \end{array}
\end{equation*}
In response to this problem, the authors introduce the notion of zero-order
corrector, which calls for the regularity assumption
\begin{equation}\label{eq:dorvilleassumption}
    \condfinale{y^{\gamma}}\in H_{0}^{1}(\Omega):\
    \primetemps{y^{\gamma}}\in \Ldeux{Q}.
\end{equation}
This last one allows them to obtain the characterization specified below of
the initial optimal control-state pair.

\begin{theoreme}[cf.~\cite{dorville}]%
    The no-regret control $u$ of the
    problem~\eqref{eq:initial-problem}\eqref{eq:initial-cost-function}\eqref{eq:initial-control-problem}
    is characterized by the unique $\left\{{u,y,\rho,p,\xi}\right\}$
    solution of the system:
    \begin{equation}
        \begin{cases}\displaystyle
            \begin{array}{rclrclrclrclll}
                Ly & = & u, & L\rho & = & 0, & L^{*}p & = & y - z_{d} +
                \rho, & L^{*}\xi & = & y & \text{in} & Q,\\
                y & = & 0, & \rho & = & 0, & p & = & 0, & \xi & = & 0, &
                \text{on} & \Sigma,\\
                y(0) & = & 0, & \rho(0) & = & \lambda(0) & {} & {} & {} &
                {} & {} & {} & \text{in} & \Omega,\\
                {} & {} & {} & {} & {} & {} & p(T) & = & 0 & \xi(T) & = & 0
                & \text{in} & \Omega,
            \end{array}
        \end{cases}
    \end{equation}
    and we have the variational inequality
    \begin{equation}
        \begin{array}{rcll}
            \scalaire{p + Nu}{v - u}{} & \geq & 0 & \forall\,v\in\uad,
        \end{array}
    \end{equation}
    with
    \begin{equation*}
        \left|{%
            \begin{array}{l}
                L = \primetemps{} - \Delta\quad\text{and}\quad L^{*} =
                -\primetemps{} - \Delta\quad\text{its adjoint operator};\\
                u, y, p, \rho, \xi\in \Ldeux{]0,T[;\Ldeux{\Omega}},\quad
                \lambda(0)\in\Ldeux{\Omega}.
            \end{array}
        }\right.
    \end{equation*}
\end{theoreme}

Ultimately, the no-regret control thus obtained does not satisfy the final
condition
\begin{equation*}
    \begin{array}{rclll}
        \condfinale{y} & = & 0 & \text{in} & \Omega,
    \end{array}
\end{equation*}
so that, in all generality, the problem remains, to the best of our
knowledge, globally open.

We propose in this paper the controllability method for the analysis,
without recourse to the Slater-type
assumption~\eqref{eq:slater-type-assumption}, of the control problem of the
ill-posed backwards heat equation. Note that this method has made it
possible to propose
in~\cite{ownElliptic},~\cite{ownAAA},~\cite{ownParabolic}
and~\cite{ownhyperbolic}, an answer to the same question of the control,
without recourse to~\eqref{eq:slater-type-assumption}, of control problems
of ill-posed Cauchy system for elliptic, parabolic and hyperbolic
operators.

The rest of this article is as follows. Section~\ref{sec:controllability}
is devoted to the announced interpretation of the initial problem as an
inverse problem. We define there the so-called exact controllability
problem which we approach, by density argument, by an equivalent problem
(this one called the approached controllability problem). In
Section~\ref{sec:controlproblem}, we return to the control
problem~\eqref{eq:initial-control-problem}, starting by regularized it
based on the controllability results previously obtained. After established
the convergence of the process in Section~\ref{sec:convergence}, then the
approached optimality system in Section~\ref{sec:approachedso}, we end in
Section~\ref{sec:singularso} with the optimality system for the initial
control problem.

\section{Controllability of the ill-posed backwards heat
equation}\label{sec:controllability}

In the present section, we introduce controllability viewpoint here
proposed. Which consists, starting from the notion of controllability
introduced by J.-L.~Lions in~\cite[222]{lions1}, in interpreting the
initial state equation as an inverse problem (we also say a controllability
problem).

We establish that, when it exists, the solution of the ill-posed backwards
heat equation~\eqref{eq:initial-problem} is also, in the limit, a solution
of the approached controllability problem. Which then makes it possible to
establish a necessary and sufficient condition characterizing the existence
of a regular solution to the problem~\eqref{eq:initial-problem}.

We therefore interpret~\eqref{eq:initial-problem} as an inverse problem
whose observation objective consists in the final condition
\begin{equation*}
    \begin{array}{rclll}
        \condfinale{z} & = & 0 & \text{in} & \Omega.
    \end{array}
\end{equation*}

More precisely, we consider the following problem
\begin{equation*}\label{eq:inverse-problem-1}
    \begin{cases}\displaystyle
        \begin{array}{rclll}
            \primetemps{y} - \Delta y & = & v & \text{in} & Q,\\
            \\
            y & = & 0 & \text{on} & \Sigma,
        \end{array}
    \end{cases}
\end{equation*}
posing the problem of finding $\displaystyle y_{0}\in\Ldeux{\Omega}$ such
that if
\begin{equation}\label{eq:inverse-problem-2}
    \begin{cases}\displaystyle
        \begin{array}{rclll}
            \primetemps{y} - \Delta y & = & v & \text{in} & Q,\\
            \\
            y & = & 0 & \text{on} & \Sigma,\\
            \\
            \condinitiale{y} & = & y_{0} & \text{in} & \Omega,
        \end{array}
    \end{cases}
\end{equation}
then
\begin{equation}\label{eq:inverse-problem-3}
    \begin{array}{rclll}
        \condfinale{y} & = & 0 & \text{in} & \Omega.
    \end{array}
\end{equation}
We say that~\eqref{eq:inverse-problem-2}\eqref{eq:inverse-problem-3}
constitutes an exact controllability problem associated with the ill-posed
backward heat equation~\eqref{eq:initial-problem}.

\begin{remarque}%
    The controllability problem is well defined. Indeed, it is well known
    that for any $v\in\Ldeux{Q}$ and
    $y_{0}\in\Ldeux{\Omega}$,~\eqref{eq:inverse-problem-2} is well posed in
    the sense of Hadamard. That is to say that its admits a unique solution
    \begin{equation*}\label{eq:well-posedness-heat-equation}
        y\!\left({v,y_{0}}\right)\in \Vbb := \Ldeux{0,T;
        H_{0}^{1}(\Omega)}\cap \Cscr\!\left({[0,T];\Ldeux{\Omega}}\right)
    \end{equation*}
    which depends continuously on the data. We deduce from this that, after
    possible modification on a set of zero measure, the solution
    $y\!\left({v,y_{0}}\right)$ of~\eqref{eq:inverse-problem-2} merges with
    a continuous function from $(0,T)$ to $\Ldeux{\Omega}$.

    Thus we can indeed speak of the final value
    $y\!\left({\cdot,T;v,y_{0}}\right)$ of $y\!\left({v,y_{0}}\right)$ in
    $\Omega$.
\end{remarque}

\begin{remarque}%
    Let us denote
    \begin{itemize}
        \item $y^{v}$ the unique solution of~\eqref{eq:inverse-problem-2};
        \item $y_{0}^{v}\in \Vbb \subset \Ldeux{Q}$ that of
            \begin{equation*}\label{eq:null-initial-condition}
                \begin{cases}\displaystyle
                    \begin{array}{rclll}
                        \primetemps{y_{0}^{v}} - \Delta y_{0}^{v} & = & v &
                        \text{in} & Q,\\
                        \\
                        y_{0}^{v} & = & 0 & \text{on} & \Sigma,\\
                        \\
                        y_{0}^{v}\!\left({\cdot,0}\right) & = & 0 &
                        \text{in} & \Omega,
                    \end{array}
                \end{cases}
            \end{equation*}
        \item and $y^{0}\in \Vbb \subset \Ldeux{Q}$ that of
            \begin{equation}\label{eq:null-source-term}
                \begin{cases}\displaystyle
                    \begin{array}{rclll}
                        \primetemps{y^{0}} - \Delta y^{0} & = & 0 &
                        \text{in} & Q,\\
                        \\
                        y^{0} & = & 0 & \text{on} & \Sigma,\\
                        \\
                        y^{0}\!\left({\cdot,0}\right) & = & y_{0} &
                        \text{in} & \Omega.
                    \end{array}
                \end{cases}
            \end{equation}
    \end{itemize}
    Then we have that the mapping
    \begin{equation*}
        \left({v,y_{0}}\right)\longmapsto y^{v} = y_{0}^{v} + y^{0},
    \end{equation*}
    is linear and continuous from $\Ldeux{Q}\times \Ldeux{Q}$ to $\Vbb\subset
    \Ldeux{Q}$.

    We deduce that the exact controllability
    problem~\eqref{eq:inverse-problem-2}\eqref{eq:inverse-problem-3} is
    equivalent to the following
    \begin{equation}\label{eq:inverse-problem-4}
        \begin{cases}\displaystyle
            \text{find}\ y_{0}\in \Ldeux{\Omega}\ \text{such that:}\\
            \text{if}\ y^{0}\ \text{is solution
            of}~\eqref{eq:null-source-term},\ \text{then}\\
            y^{0}\!\left({\cdot,T}\right) = -y_{0}^{v}\!\left({\cdot,
            T}\right)\quad \text{in}\ \Omega.
        \end{cases}
    \end{equation}
\end{remarque}

% \begin{remarque}\label{rq:naturalassumption}%
%     Avec ce qui précède, il apparaît naturel de considérer, dans le cadre
%     du problème de contrôle, que c'est l'ensemble
%     \begin{equation*}
%         \left\{{%
%             (v,z)\in\Ascr:\ \condinitiale{z}\in\Ldeux{\Omega}
%         }\right\}
%     \end{equation*}
%     qui est non vide et non seulement $\Ascr$.
% \end{remarque}

We approach the exact controllability problem~\eqref{eq:inverse-problem-4}
by density argument as specified below.

\begin{proposition}\label{propo:controllability-result-1}%
    When the initial data $y_{0}$ traverses $\Ldeux{\Omega}$, the set
    \begin{equation*}\label{eq:controllability-result-1}
        E = \left\{{%
            y^{0}\!\left({\cdot,T}\right)\,;\ y_{0}\in\Ldeux{\Omega}
        }\right\},
    \end{equation*}
    described by the final values of the solution $y^{0}$
    of~\eqref{eq:null-source-term}, is dense in $\Ldeux{\Omega}$.
\end{proposition}

\begin{proof}%
    It is clear that the set $E$ constitute a vector subspace of
    $\Ldeux{\Omega}$. From where, by the Hahn-Banach Theorem, $E$ is dense
    in $\Ldeux{\Omega}$ if and only if $E^{\perp} = \{0\}$.

    Let us consider $k\in E^{\perp}$; so we have
    \begin{equation*}\label{eq:proof-controllability-26}
        \forall\,y_{0}\in\Ldeux{\Omega},\qquad
        \scalaireohm{k}{y^{0}\!\left({\cdot,T}\right)} = 0.
    \end{equation*}
    But it comes from~\eqref{eq:null-source-term} that, for any test
    function $\vphi\in\Dscr(Q)$, we have:
    \begin{equation*}
        \begin{split}
            \scalaireq{\primetemps{y^{0}} - \Delta y^{0}}{\vphi} = 0 &\iff
            \scalaireq{\primetemps{y^{0}}}{\vphi} - \scalaireq{\Delta
            y^{0}}{\vphi} = 0\\
            &\iff \scalaireohm{%
                y^{0}\!\left({\cdot,T}\right)
            }{%
                \vphi\!\left({\cdot,T}\right)
            } - \scalaireohm{%
                y_{0}
            }{%
                \vphi\!\left({\cdot,0}\right)
            }\\
            &\qquad - \scalaireq{y^{0}}{\primetemps{\vphi}} - \scalaireq{y^{0}
            }{%
                \Delta \vphi
            } + \scalaires{y^{0}
            }{\deriveenormale{\vphi}
            }\\
            &\qquad\qquad - \scalaires{\deriveenormale{y^{0}}}{\vphi} = 0
        \end{split}
    \end{equation*}
    \ie{}
    \begin{equation}\label{eq:controllability-proof-30}
        \begin{split}
            \scalaireohm{%
                y^{0}\!\left({\cdot,T}\right)
            }{%
                \vphi\!\left({\cdot,T}\right)
            } &- \scalaireohm{%
                y_{0}
            }{%
                \vphi\!\left({\cdot,0}\right)
            } - \scalaireq{%
                y^{0}
            }{%
                \primetemps{\vphi}
            }\\
            &\qquad - \scalaireq{%
                y^{0}
            }{%
                \Delta\vphi
            } - \scalaires{%
                \deriveenormale{y^{0}}
            }{\vphi} = 0.
        \end{split}
    \end{equation}
    Choosing $\vphi$, in the above, such that
    \begin{equation}\label{eq:controllability-proof-31}
        \begin{cases}\displaystyle
            \begin{array}{rclll}
                -\primetemps{\vphi} - \Delta\vphi & = & 0 & \text{in} & Q,\\
                \\
                \vphi & = & 0 & \text{on} & \Sigma,\\
                \\
                \vphi\!\left({\cdot,T}\right) & = & k & \text{in} &
                \Omega,
            \end{array}
        \end{cases}
    \end{equation}
    it comes that~\eqref{eq:controllability-proof-30} is equivalent to
    \begin{equation}\label{eq:controllability-proof-32}
        \scalaireohm{k}{y^{0}\!\left({\cdot,T}\right)} - \scalaireohm{%
            y_{0}
        }{%
            \vphi\!\left({\cdot,0}\right)
        } = 0,
    \end{equation}
    where
    \begin{equation*}
        k\in E^{\perp}\iff \scalaireohm{k}{y^{0}\!\left({\cdot,T}\right)} =
        0.
    \end{equation*}
    Thus~\eqref{eq:controllability-proof-32} becomes
    \begin{equation}\label{eq:controllability-proof-33}
        \forall\,y_{0}\in\Ldeux{\Omega},\qquad \scalaireohm{y_{0}}{%
            \vphi\!\left({\cdot,0}\right)
        } = 0.
    \end{equation}
    But we can still choose, in~\eqref{eq:controllability-proof-33},
    $\displaystyle y_{0} = \vphi\!\left({\cdot,0}\right)$ in $\Omega$, and
    then it follows
    \begin{equation*}
        \normecohm{\vphi\!\left({\cdot,0}\right)} = 0\qquad\ie\qquad
        \begin{array}{rclll}
            \vphi\!\left({\cdot,0}\right) & = & 0 & \text{in} & \Omega.
        \end{array}
    \end{equation*}
    Which brings, with~\eqref{eq:controllability-proof-31}, that $\vphi$ is
    solution of
    \begin{equation*}
        \begin{cases}\displaystyle
            \begin{array}{rclll}
                -\primetemps{\vphi} - \Delta\vphi & = & 0 & \text{in} &
                Q,\\
                \\
                \vphi & = & 0 & \text{on} & \Sigma,\\
                \\
                \vphi\!\left({\cdot,0}\right) & = & 0 & \text{in} &
                \Omega,
            \end{array}
        \end{cases}
    \end{equation*}
    that is to say that $\vphi\equiv 0$ and consequently that
    \begin{equation*}
        \begin{array}{rclll}
            \vphi\!\left({\cdot,T}\right) & = & k = 0 & \text{in} &
            \Omega.
        \end{array}
    \end{equation*}
    From where we deduce that $E^{\perp} = \{0\}$ and therefore that $E$ is
    dense in $\Ldeux{\Omega}$.
\end{proof}

The following result is then immediate.

\begin{corollaire}\label{coro:controllability-result-2}%
    For any $\veps > 0$ and $\kappa\in\Ldeux{\Omega}$, there exists
    ${y_{0}}_{\veps}\in\Ldeux{\Omega}$ such that the solution $\displaystyle
    y_{\veps}^{0}\in \Vbb$ of
    \begin{equation}\label{eq:controllability-result-3}
        \begin{cases}\displaystyle
            \begin{array}{rclll}
                \primetemps{y_{\veps}^{0}} - \Delta y_{\veps}^{0} & = & v &
                \text{in} & Q,\\
                \\
                y_{\veps}^{0} & = & 0 & \text{on} & \Sigma,\\
                \\
                y_{\veps}^{0}\!\left({\cdot,0}\right) & = & {y_{0}}_{\veps}
                & \text{in} & \Omega,
            \end{array}
        \end{cases}
    \end{equation}
    also satisfies
    \begin{equation*}\label{eq:controllability-result-4}
        \normeohm{y_{\veps}^{0}\!\left({\cdot,T}\right) - \kappa} < \veps.
    \end{equation*}
\end{corollaire}

From which it also follows that

\begin{corollaire}\label{coro:controllability-result-3}%
    Given $v\in\Ldeux{Q}$ and $\veps > 0$, there exists ${y_{0}}_{\veps}\in
    \Ldeux{\Omega}$ such that the solution $\displaystyle y_{\veps}^{v}\in
    \Vbb$ of
    \begin{equation}\label{eq:controllability-result-5}
        \begin{cases}\displaystyle
            \begin{array}{rclll}
                \primetemps{y_{\veps}^{v}} - \Delta y_{\veps}^{v} & = & v &
                \text{in} & Q,\\
                \\
                y_{\veps}^{v} & = & 0 & \text{on} & \Sigma,\\
                \\
                y_{\veps}^{v}\!\left({\cdot,0}\right) & = & {y_{0}}_{\veps}
                & \text{in} & \Omega,
            \end{array}
        \end{cases}
    \end{equation}
    satisfies
    \begin{equation*}\label{eq:controllability-result-6}
        \normeohm{y_{\veps}^{v}\!\left({\cdot,T}\right)} < \veps.
    \end{equation*}
\end{corollaire}

\begin{proof}%
    Let $v\in\Ldeux{Q}$ and $\veps > 0$. We know that there exists a unique
    $y_{0}^{v}\in \Vbb$, almost everywhere equal to a continuous mapping
    from $(0,T)$ to $\Ldeux{\Omega}$, solution of
    \begin{equation*}\label{eq:controllability-proof-39}
        \begin{cases}\displaystyle
            \begin{array}{rclll}
                \primetemps{y_{0}^{v}} - \Delta y_{0}^{v} & = & v &
                \text{in} & Q,\\
                \\
                y_{0}^{v} & = & 0 & \text{on} & \Sigma,\\
                \\
                y_{0}^{v}\!\left({\cdot,0}\right) & = & 0 & \text{in} &
                \Omega.
            \end{array}
        \end{cases}
    \end{equation*}
    It follows that, with Corollary~\ref{coro:controllability-result-2},
    there exists ${y_{0}}_{\veps}\in\Ldeux{\Omega}$ such that the solution
    $y_{\veps}^{0}\in \Vbb$ of~\eqref{eq:controllability-result-3}
    satisfies
    \begin{equation*}\label{eq:controllability-proof-40}
        \normeohm{%
            y_{\veps}^{0}\!\left({\cdot,T}\right) - \left({%
                -y_{0}^{v}\!\left({\cdot,T}\right)
            }\right)
        } < \veps.
    \end{equation*}
    Which allows us to conclude that $\displaystyle y_{\veps}^{v} =
    \left({%
        y_{0}^{v} + y_{\veps}^{0}
    }\right) \in \Vbb$ is unique solution
    of~\eqref{eq:controllability-result-5}, with
    \begin{equation*}
        \normeohm{y_{\veps}^{v}\!\left({\cdot,T}\right)} = \normeohm{%
            y_{0}^{\veps}\!\left({\cdot,T}\right) +
            y_{0}^{v}\!\left({\cdot,T}\right)
        } < \veps.
    \end{equation*}
\end{proof}

More over, we have the following theorem, characterizing the existence of a
regular solution to the ill-posed backwards heat equation.

\begin{theoreme}\label{thm:controllability-result-5}%
    Let $v\in\Ldeux{Q}$. The ill-posed backwards heat equation
    \begin{equation}\label{eq:initial-problem-2}
        \begin{cases}\displaystyle
            \begin{array}{rclll}
                \primetemps{z} - \Delta z & = & v & \text{in} & Q,\\
                \\
                z & = & 0 & \text{on} & \Sigma,\\
                \\
                z(\cdot,T) & = & 0 & \text{in} & \Omega,
            \end{array}
        \end{cases}
    \end{equation}
    admits a regular solution $z\in \Vbb$ if and only if the sequence
    $\displaystyle {\left({{y_{0}}_{\veps}}\right)}_{\veps}$ is bounded in
    $\Ldeux{\Omega}$.
\end{theoreme}

\begin{proof}%
    \begin{enumerate}
        \item Let $\veps > 0$. From
            Corollary~\ref{coro:controllability-result-2}, there exists
            ${y_{0}}_{\veps}\in \Ldeux{\Omega}$ such as $\displaystyle
            y_{\veps}^{v}\in \Vbb$ is solution of
            \begin{equation}\label{eq:proof-thm-controllability-42}
                \begin{cases}\displaystyle
                    \begin{array}{rclll}
                        \primetemps{y_{\veps}^{v}} - \Delta y_{\veps}^{v} &
                        = & v & \text{in} & Q,\\
                        \\
                        y_{\veps}^{v} & = & 0 & \text{on} & \Sigma,\\
                        \\
                        y_{\veps}^{v}\!\left({\cdot,0}\right) & = &
                        {y_{0}}_{\veps} & \text{in} & \Omega,
                    \end{array}
                \end{cases}
            \end{equation}
            with the estimate
            \begin{equation*}\label{eq:proof-thm-controllability-43}
                \normeohm{y_{\veps}^{v}\!\left({\cdot, T}\right)} < \veps.
            \end{equation*}
            Then, we generate sequences
            \begin{equation*}
                {\left({{y_{0}}_{\veps}}\right)}_{\veps}\subset
                \Ldeux{\Omega}\qquad\text{and}\qquad
                {\left({y_{\veps}^{v}}\right)}_{\veps}\subset \Vbb.
            \end{equation*}
            Assuming that the sequence
            ${\left({{y_{0}}_{\veps}}\right)}_{\veps}$ is bounded in
            $\Ldeux{\Omega}$. Therefore, it follows that, the homogeneous
            Dirichlet problem~\eqref{eq:proof-thm-controllability-42} for
            the heat equation being well-posed, the sequence
            ${\left({y_{\veps}^{v}}\right)}_{\veps}$ is bounded in $\Vbb$,
            and therefore also in $\Ldeux{Q}$.

            So we can extract from
            ${\left({{y_{0}}_{\veps}}\right)}_{\veps}$ and
            ${\left({y_{\veps}^{v}}\right)}_{\veps}$ subsequences, still
            denoted in the same way, which converge weakly in
            $\Ldeux{\Omega}$ and $\Vbb$, respectively.

            So that, there exists $\displaystyle y_{0}\in \Ldeux{\Omega}$
            and $\displaystyle y^{v}\in \Vbb$, such as
            \begin{equation*}\label{eq:proof-thm-controllability-44}
                \begin{cases}\displaystyle
                    \begin{array}{rclll}
                        {y_{0}}_{\veps} & \longrightarrow & y_{0} &
                        \text{weakly in} & \Ldeux{\Omega},\\
                        \\
                        y_{\veps}^{v} & \longrightarrow & y^{v} &
                        \text{weakly in} & \Vbb.
                    \end{array}
                \end{cases}
            \end{equation*}
            Thus we have on the one hand, that
            \begin{equation*}\label{eq:proof-thm-controllability-45}
                \normeohm{y_{\veps}^{v}\!\left({\cdot,T}\right)} < \veps
                \quad\text{and}\quad
                \begin{array}{rclll}
                    y_{\veps}^{v} & \longrightarrow & y^{v} & \text{weakly
                    in} & \Vbb
                \end{array}
            \end{equation*}
            imply, $y_{\veps}^{v}$ being almost everywhere equal to a
            continuous mapping from $(0,T)$ to $\Ldeux{\Omega}$, that
            \begin{equation}\label{eq:proof-thm-controllability-46}
                \begin{array}{rclll}
                    y^{v}\!\left({\cdot, T}\right) & = & 0 & \text{in} &
                    \Omega.
                \end{array}
            \end{equation}
            On the other hand, we have for any $\vphi\in\Cscr(Q)$ that:
            \begin{equation*}
                \begin{split}
                    {} & \primetemps{y_{\veps}^{v}} - \Delta y_{\veps}^{v}
                    = v\ \text{in}\ Q \iff \scalaireq{%
                        \primetemps{y_{\veps}^{v}} - \Delta y_{\veps}^{v}
                    }{\vphi} = \scalaireq{v}{\vphi}\\
                    &\qquad\iff \scalaireohm{%
                        \condfinale{y_{\veps}^{v}}
                    }{%
                        \condfinale{\vphi}
                    } - \scalaireohm{%
                        {y_{0}}_{\veps}
                    }{%
                        \condinitiale{\vphi}
                    }\\
                    &\qquad\qquad - \scalaireq{%
                        y_{\veps}^{v}
                    }{%
                        \primetemps{\vphi}
                    } - \scalaireq{%
                        y_{\veps}^{v}
                    }{%
                        \Delta\vphi{}
                    } - \scalaires{%
                        \deriveenormale{y_{\veps}^{v}}
                    }{%
                        \vphi{}
                    }\\
                    &\qquad\qquad\qquad = \scalaireq{v}{\vphi},
                \end{split}
            \end{equation*}
            so, by passing to the limit,
            \begin{equation*}
                \begin{split}
                    {} & \scalaireohm{%
                        \condfinale{y^{v}}
                    }{%
                        \condfinale{\vphi}
                    } - \scalaireohm{%
                        y_{0}
                    }{%
                        \condinitiale{\vphi}
                    } - \scalaireq{%
                        y^{v}
                    }{%
                        \primetemps{\vphi}
                    }\\
                    &\qquad\qquad - \scalaireq{%
                        y^{v}
                    }{%
                        \Delta\vphi{}
                    } - \scalaires{%
                        \deriveenormale{y^{v}}
                    }{%
                        \vphi{}
                    } = \scalaireq{v}{\vphi}\\
                    &\qquad \iff \scalaireohm{%
                        \condinitiale{y^{v}} - y_{0}
                    }{%
                        \condinitiale{\vphi}
                    } + \scalaireq{%
                        \primetemps{y^{v}} - \Delta y^{v}
                    }{%
                        \vphi{}
                    }\\
                    &\qquad\qquad - \scalaires{%
                        y^{v}
                    }{%
                        \deriveenormale{\vphi}
                    } = \scalaireq{v}{\vphi}.
                \end{split}
            \end{equation*}
            This last equality being valid for any $\vphi\in\Cscr(Q)$, it
            follows that
            \begin{equation*}
                \begin{cases}\displaystyle
                    \begin{array}{rclll}
                        \primetemps{y^{v}} - \Delta y^{v} & = & v &
                        \text{in} & Q,\\
                        \\
                        y^{v} & = & 0 & \text{on} & \Sigma,\\
                        \\
                        \condinitiale{y^{v}} & = & y_{0} & \text{in} &
                        \Omega,
                    \end{array}
                \end{cases}
            \end{equation*}
            which implies, with~\eqref{eq:proof-thm-controllability-46},
            that
            \begin{equation*}
                \begin{cases}\displaystyle
                    \begin{array}{rclll}
                        \primetemps{y^{v}} - \Delta y^{v} & = & v &
                        \text{in} & Q,\\
                        \\
                        y^{v} & = & 0 & \text{on} & \Sigma,\\
                        \\
                        \condfinale{y^{v}} & = & 0 & \text{in} & \Omega,
                    \end{array}
                \end{cases}
            \end{equation*}
            that is to say that $y^{v}\in \Vbb$ is solution of the
            ill-posed backwards heat equation~\eqref{eq:initial-problem-2}.
        \item Now, let assume that the ill-posed backwards heat
            equation~\eqref{eq:initial-problem-2} admits a regular solution
            $z\in \Vbb$.

            Then, since $z\in \Vbb$ implies that $z$ is almost everywhere
            equal to a continuous function from $(0,T)$ to
            $\Ldeux{\Omega}$, we can, after possible modification on a set
            of zero measure, to speak of the initial value $\displaystyle
            \condinitiale{z}\in\Ldeux{\Omega}$.

            Then, by choosing, for any $\veps > 0$,
            \begin{equation*}
                \begin{array}{rclll}
                    {y_{0}}_{\veps} & = & \condinitiale{z} & \text{in} &
                    \Omega,
                \end{array}
            \end{equation*}
            we obtain that the sequence
            ${\left({{y_{0}}_{\veps}}\right)}_{\veps}$, since constant, is
            bounded in $\Ldeux{\Omega}$.
    \end{enumerate}
\end{proof}

\section{The optimal control problem}\label{sec:controlproblem}

Let us start by recalling that we are here interested in the control
problem of the ill-posed backwards heat equation. More precisely, that is
to say that, for $v,z\in \Ldeux{Q}$ satisfying
\begin{equation}\label{eq:control80}
    \begin{cases}\displaystyle
        \begin{array}{rclll}
            \primetemps{z} - \Delta z & = & v & \text{in} & Q,\\
            \\
            z & = & 0 & \text{on} & \Sigma,\\
            \\
            \condfinale{z} & = & 0 & \text{on} & \Omega,
        \end{array}
    \end{cases}
\end{equation}
we introduce the cost function
\begin{equation}\label{eq:control81}
    J(v,z) = \dfrac{1}{2}\normecq{z - z_{d}} + \dfrac{N}{2}\normecq{v},
\end{equation}
being interested in the control problem
\begin{equation}\label{eq:control82}
    \inf\!\left\{{%
        J(v,z);\ (v,z)\in\Ascr
    }\right\}.
\end{equation}

As underlined in the introduction, the optimal control
problem~\eqref{eq:control80}\eqref{eq:control81}\eqref{eq:control82} admits
a unique solution $(u,y)$ whose characterization, via a strong and
decoupled singular optimality system, and this without using the
Slater-type assumption~\eqref{eq:slater-type-assumption}, is the main
objective.

To do so, starting from the non-vacuity assumption of the set of admissible
control-state pairs, and using the results previously obtained, we have
that, for any $v\in\Ldeux{Q}$ and $\veps > 0$, there exist
\begin{equation*}
    {y_{0}}_{\veps}\in\Ldeux{\Omega}\qquad\text{and}\qquad y_{\veps}^{v}\in
    \Vbb
\end{equation*}
such that
\begin{equation*}
    \begin{cases}\displaystyle
        \begin{array}{rclll}
            \primetemps{y_{\veps}^{v}} - \Delta y_{\veps}^{v} & = & v &
            \text{in} & Q,\\
            \\
            y_{\veps}^{v} & = & 0 & \text{on} & \Sigma,\\
            \\
            \condinitiale{y_{\veps}^{v}} & = & {y_{0}}_{\veps} &
            \text{in} & \Omega,
        \end{array}
    \end{cases}
\end{equation*}
with the estimate
\begin{equation*}
    \normeohm{%
        \condfinale{y_{\veps}^{v}}
    } < \veps.
\end{equation*}
Assuming that the optimal solution $(u,y)$
for~\eqref{eq:control80}\eqref{eq:control81}\eqref{eq:control82} is such
that $\displaystyle \condinitiale{y}\in\Ldeux{\Omega}$, we introduce the
functional
\begin{equation*}
    J_{\veps}(v) = \dfrac{1}{2}\normecq{y_{\veps}^{v} - z_{d}} +
    \dfrac{N}{2}\normecq{v} + \dfrac{1}{2\veps}\normecohm{{y_{0}}_{\veps} -
    \condinitiale{y}},
\end{equation*}
being interested in the control problem
\begin{equation}\label{eq:control87}
    \inf\!\left\{{%
        J_{\veps}(v);\ v\in\uad
    }\right\}.
\end{equation}
The following result is then immediate.

\begin{proposition}%
    For any $\veps > 0$, the control problem~\eqref{eq:control87} admits a
    unique solution: the approached optimal control-state pair $\bareps{u}$.
\end{proposition}

\subsection{Convergence of the method}\label{sec:convergence}

Let $\veps > 0$. For the approached optimal control, the results previously
obtained lead to the existence of
\begin{equation*}
    {\bar{y}_{0}}_{\veps}\in\Ldeux{\Omega}\qquad\text{and}\qquad
    \bareps{y}\in\Vbb
\end{equation*}
satisfying
\begin{equation}\label{eq:control89}
    \begin{cases}\displaystyle
        \begin{array}{rclll}
            \primetemps{\bareps{y}} - \Delta\bareps{y} & = & \bareps{u} &
            \text{in} & Q,\\
            \\
            \bareps{y} & = & 0 & \text{on} & \Sigma,\\
            \\
            \condinitiale{\bareps{y}} & = & {\bar{y}_{0}}_{\veps} &
            \text{in} & \Omega,
        \end{array}
    \end{cases}
\end{equation}
and the estimate
\begin{equation*}
    \normeohm{\condfinale{\bareps{y}} } < \veps,
\end{equation*}
with
\begin{equation*}
    J_{\veps}\!\left({\bareps{u}}\right)\leq J_{\veps}(v),\quad
    \forall\,v\in\uad.
\end{equation*}
In particular
\begin{equation}\label{eq:control92}
    J_{\veps}\!\left({\bareps{u}}\right) \leq J_{\veps}(u),
\end{equation}
where $u$ is the optimal control
for~\eqref{eq:control80}\eqref{eq:control81}\eqref{eq:control82}.

We have in fact that $J_{\veps}(u)$ is independant of $\veps$. Indeed,
since the optimal state $y$ associated with the optimal control $u$
satisfies $\condinitiale{y}\in\Ldeux{\Omega}$, we can take
${y_{0}^{*}}_{\veps} = \condinitiale{y}$ to obtain that $y$ satisfies
\begin{equation*}
    \begin{cases}\displaystyle
        \begin{array}{rclll}
            \primetemps{y} - \Delta y & = & u & \text{in} & Q,\\
            \\
            y & = & 0 & \text{on} & \Sigma,\\
            \\
            \condinitiale{y} & = & {y_{0}^{*}}_{\veps} & \text{in} &
            \Omega
        \end{array}
    \end{cases}
\end{equation*}
with
\begin{equation*}
    \begin{array}{rclll}
        \condfinale{y} & = & 0 & \text{in} & \Omega,
    \end{array}
    \qquad\text{\textit{a fortiori}}\qquad \normeohm{\condfinale{y}} <
    \veps.
\end{equation*}
It follows that $J_{\veps}(u)$ is defined, with
\begin{equation*}
    \begin{split}
        J_{\veps}(u) &= \dfrac{1}{2}\normecq{y - z_{d}} +
        \dfrac{N}{2}\normecq{u} + \dfrac{1}{2\veps}\normecohm{%
            {y_{0}^{*}}_{\veps} - \condinitiale{y}
        }\\
        & = \dfrac{1}{2}\normecq{y - z_{d}} + \dfrac{N}{2}\normecq{u} =
        J(u,y).
    \end{split}
\end{equation*}
Hence~\eqref{eq:control92} becomes
\begin{equation}\label{eq:control95}
    J_{\veps}\!\left({\bareps{u}}\right) \leq J_{\veps}(u) = J(u,y),
\end{equation}
and it follows that there exist constants $C_{i}\in{\Rbb}_{+}^{*}$
independant of $\veps$ such as
\begin{equation*}
    \normeq{\bareps{y}} \leq C_{1},\qquad
    \normeq{\bareps{u}}\leq C_{2}\qquad\text{and}\qquad
    \normeohm{{\bar{y}_{0}}_{\veps}}\leq C_{3}.
\end{equation*}
Then, we immediately deduce that there exists $\hat{u}\in\Ldeux{Q}$,
$\hat{y}\in\Ldeux{Q}$ and ${\hat{y}}_{0}\in\Ldeux{\Omega}$ such as
\begin{equation*}
    \begin{cases}\displaystyle
        \begin{array}{rclll}
            \bareps{u} & \longrightarrow & \hat{u} & \text{weakly in} &
            \Ldeux{Q},\\
            \bareps{y} & \longrightarrow & \hat{y} & \text{weakly in} &
            \Ldeux{Q},\\
            {\bar{y}_{0}}_{\veps} & \longrightarrow & {\hat{y}}_{0} &
            \text{weakly in} & \Ldeux{\Omega}.
        \end{array}
    \end{cases}
\end{equation*}
But much more,~\eqref{eq:control95} also implies
\begin{equation}\label{eq:control106}
    \normeohm{{\bar{y}_{0}}_{\veps} - \condinitiale{y}} \leq 2\veps\,C_{4},
\end{equation}
which leads, since
\begin{equation}\label{eq:control106p}
    \begin{array}{rclll}
        {\bar{y}_{0}}_{\veps} & \longrightarrow & {\hat{y}}_{0} &
        \text{weakly in} & \Ldeux{\Omega},
    \end{array}
\end{equation}
to
\begin{equation}\label{eq:control107}
    \begin{array}{rclll}
        {\hat{y}}_{0} & = & \condinitiale{y} & \text{in} & \Omega.
    \end{array}
\end{equation}
Then it comes, with~\eqref{eq:control89}, that for any $\vphi\in\Dscr(Q)$,
\begin{equation*}
    \begin{split}
        \scalaireq{\primetemps{\bareps{y}}}{\vphi} &- \scalaireq{%
            \Delta\bareps{y}
        }{%
            \vphi%
        } = \scalaireq{\bareps{u}}{\vphi}\\
        % &\iff \scalaireohm{%
        %     \condfinale{\bareps{y}}
        % }{%
        %     \condfinale{\vphi}
        % } - \scalaireohm{%
        %     \condinitiale{\bareps{y}}
        % }{%
        %     \condinitiale{\vphi}
        % } - \scalaireq{%
        %     \bareps{y}
        % }{%
        %     \primetemps{\vphi}
        % }\\
        % &\qquad - \scalaireq{%
        %     \bareps{y}
        % }{%
        %     \Delta\vphi%
        % } - \scalaires{%
        %     \deriveenormale{\bareps{y}}
        % }{%
        %     \vphi%
        % } + \scalaires{%
        %     \bareps{y}
        % }{%
        %     \deriveenormale{\vphi}
        % } = \scalaireq{\bareps{u}}{\vphi}\\
        &\iff \scalaireohm{%
            \condfinale{\bareps{y}}
        }{%
            \condfinale{\vphi}
        } - \scalaireohm{%
            {\bar{y}_{0}}_{\veps}
        }{%
            \condinitiale{\vphi}
        } - \scalaireq{%
            \bareps{y}
        }{%
            \primetemps{\vphi}
        }\\
        &\qquad - \scalaireq{%
            \bareps{y}
        }{%
            \Delta\vphi%
        } - \scalaires{%
            \deriveenormale{\bareps{y}}
        }{%
            \vphi%
        } = \scalaireq{\bareps{u}}{\vphi},
    \end{split}
\end{equation*}
which gives, passing to the limit,
\begin{equation*}
    \begin{split}
        \scalaireohm{%
            \condfinale{\hat{y}}
        }{%
            \condfinale{\vphi}
        } & - \scalaireohm{%
            {\hat{y}}_{0}
        }{%
            \condinitiale{\vphi}
        } - \scalaireq{%
            \hat{y}
        }{%
            \primetemps{\vphi}
        } - \scalaireq{%
            \hat{y}
        }{%
            \Delta\vphi%
        }\\
        &\qquad - \scalaires{%
            \deriveenormale{\hat{y}}
        }{%
            \vphi%
        } = \scalaireq{\hat{u}}{\vphi}\\
        &\iff \scalaireohm{%
            \condinitiale{\hat{y}} - {\hat{y}}_{0}
        }{%
            \condinitiale{\vphi}
        } + \scalaireq{%
            \primetemps{\hat{y}} - \Delta\hat{y}
        }{%
            \vphi%
        }\\
        &\qquad - \scalaires{%
            \hat{y}
        }{%
            \deriveenormale{\vphi}
        } = \scalaireq{\hat{u}}{\vphi},
    \end{split}
\end{equation*}
\ie{}
\begin{equation}\label{eq:control102}
    \begin{cases}\displaystyle
        \begin{array}{rclll}
            \primetemps{\hat{y}} - \Delta\hat{y} & = & \hat{u} &
            \text{in} & Q,\\
            \\
            \hat{y} & = & 0 & \text{on} & \Sigma,\\
            \\
            \condinitiale{\hat{y}} & = & {\hat{y}}_{0} & \text{in} &
            \Omega.
        \end{array}
    \end{cases}
\end{equation}
Moreover, the norm $\normeohm{\cdot}$ being continuous, \textit{a fortiori}
weakly continuous,
\begin{equation*}
    \begin{array}{rclll}
        \bareps{y} & \longrightarrow & \hat{y} & \text{weakly in} &
        \Ldeux{Q}
    \end{array}\qquad\text{and}\qquad
    \normeohm{\condfinale{\bareps{y}}} < \veps
\end{equation*}
bring
\begin{equation}\label{eq:control103}
    \begin{array}{rclll}
        \condfinale{\hat{y}} & = & 0 & \text{in} & \Omega.
    \end{array}
\end{equation}
Thus~\eqref{eq:control102}~and~\eqref{eq:control103} allow to conclude that
$\hat{y}\in\Ldeux{Q}$ satisfies
\begin{equation*}
    \begin{cases}\displaystyle
        \begin{array}{rclll}
            \primetemps{\hat{y}} - \Delta\hat{y} & = & \hat{u} &
            \text{in} & Q,\\
            \\
            \hat{y} & = & 0 & \text{on} & \Sigma,\\
            \\
            \condfinale{\hat{y}} & = & 0 & \text{in} & \Omega.
        \end{array}
    \end{cases}
\end{equation*}
Then, noting that $\hat{u}\in\uad$, since $\bareps{u}\in\uad$ and $\uad$ is
closed and therefore weakly closed, we get that the control-state pair
$\left({\hat{u},\hat{y}}\right)$ is admissible
for~\eqref{eq:control80}\eqref{eq:control81}\eqref{eq:control82}, so that
\begin{equation}\label{eq:control105}
    J(u,y)\leq J\!\left({\hat{u},\hat{y}}\right).
\end{equation}
On the other hand, passing~\eqref{eq:control95} to the limit when $\veps\to
0$, it comes $J\!\left({\hat{u},\hat{y}}\right)\leq J(u,y)$; that is to
say, with~\eqref{eq:control105}, that $J(u,y) =
J\!\left({\hat{u},\hat{y}}\right)$.

Then we conclude, by uniqueness of the optimal control-state pair $(u,y)$,
that $\left({\hat{u},\hat{y}}\right) = (u,y)$, which ends up proving the
following result.

\begin{proposition}%
    For any $\veps > 0$, the approached optimal control $\bareps{u}$ and
    the associated state $\bareps{y}$ satisfy
    \begin{equation*}
        \begin{cases}\displaystyle
            \begin{array}{rclll}
                \bareps{u} & \longrightarrow & u & \text{weakly in} &
                \Ldeux{Q},\\
                % \\
                \bareps{y} & \longrightarrow & y & \text{weakly in} &
                \Ldeux{Q},
            \end{array}
        \end{cases}
    \end{equation*}
    where $(u,y)$ is the optimal control-state pair
    for~\eqref{eq:control80}\eqref{eq:control81}\eqref{eq:control82}.
\end{proposition}

We establish below that we actually have more: the strong convergence.

\begin{theoreme}%
    For any $\veps > 0$, the approached optimal control $\bareps{u}$ and
    the associated state $\bareps{y}$ are such that, when $\veps \to 0$,
    \begin{equation*}
        \begin{cases}\displaystyle
            \begin{array}{rclll}
                \bareps{u} & \longrightarrow & u & \text{strongly in} &
                \Ldeux{Q},\\
                % \\
                \bareps{y} & \longrightarrow & y & \text{strongly in} &
                \Ldeux{Q},
            \end{array}
        \end{cases}
    \end{equation*}
    where $(u,y)$ is the optimal control-state pair
    for~\eqref{eq:control80}\eqref{eq:control81}\eqref{eq:control82}.
\end{theoreme}

\begin{proof}%
    From the results previously obtained, we have that
    \begin{equation}\label{eq:control112}
        \begin{array}{rclll}
            \bareps{u} & \longrightarrow & u & \text{weakly in} & \Ldeux{Q},
        \end{array}
    \end{equation}

    \begin{equation}\label{eq:control113}
        \begin{array}{rcllll}
            \bareps{y} & \longrightarrow & y & \text{weakly in} & \Ldeux{Q},
        \end{array}
    \end{equation}
    and
    \begin{equation*}
        J(u,y) = \lim_{\veps\to 0}J_{\veps}\!\left({\bareps{u}}\right).
    \end{equation*}
    Where, from~\eqref{eq:control106}\eqref{eq:control106p}
    and~\eqref{eq:control107}, the last equality above can still be written
    \begin{equation}\label{eq:control115}
        \normecq{y - z_{d}} + N\normecq{u} = \lim_{\veps\to 0}\left({%
            \normecq{\bareps{y} - z_{d}} + N\normecq{\bareps{u}} +
            \dfrac{1}{\veps}\normecq{\bareps{y} - \condinitiale{y}}
        }\right).
    \end{equation}
    But then, the norm $\normeq{\cdot}$ being continous, \textit{a
    fortiori} weakly lower semi-continous, it comes,
    with~\eqref{eq:control112} and~\eqref{eq:control113}, that
    \begin{equation*}
        \begin{cases}\displaystyle
            \begin{array}{rcll}
                \normecq{y - z_{d}} & \leq & \underset{\veps\to
                0}{\lim\inf}\, \normecq{\bareps{y} - z_{d}},\\
                \\
                \normecq{u} & \leq & \underset{\veps\to 0}{\lim\inf}\,
                \normecq{\bareps{u}}.
            \end{array}
        \end{cases}
    \end{equation*}
    From where it follows, with~\eqref{eq:control115}, that
    \begin{equation}\label{eq:control118}
        \normecq{y -z_{d}} = \lim_{\veps\to 0}\normecq{\bareps{y} - z_{d}},
    \end{equation}
    and
    \begin{equation}\label{eq:control119}
        \normecq{u} = \lim_{\veps\to 0}\normecq{\bareps{u}}.
    \end{equation}
    Hence, since
    \begin{equation*}
        \normecq{\bareps{y} - y} = \normecq{\bareps{y} - z_{d}} + \normeq{y
        - z_{d}} -2\scalaireq{%
            \bareps{y} - z_{d}
        }{%
            y - z_{d}
        },
    \end{equation*}
    we conclude with~\eqref{eq:control113} and~\eqref{eq:control118} that
    \begin{equation*}\label{eq:control120}
        \lim_{\veps\to 0}\normecq{\bareps{y} - y} = 0\qquad\ie\qquad
        \begin{array}{rclll}
            \bareps{y} & \longrightarrow & y & \text{strongly in} &
            \Ldeux{Q}.
        \end{array}
    \end{equation*}
    In a similar way,~\eqref{eq:control112} and~\eqref{eq:control119} lead
    to
    \begin{equation*}
        \begin{array}{rclll}
            \bareps{u} & \longrightarrow & u & \text{strongly in} &
            \Ldeux{Q},
        \end{array}
    \end{equation*}
    which ends up proving the announced result.
\end{proof}

\subsection{Approached optimality system}\label{sec:approachedso}

Let us start by recalling that, for any $\veps > 0$ and for the optimal
control $\bareps{u}\in\uad$, we have the existence of
\begin{equation*}
    {\bar{y}_{0}}_{\veps}\in\Ldeux{\Omega}\qquad\text{and}\qquad
    \bareps{y}\in\Ldeux{Q},
\end{equation*}
verifying
\begin{equation*}
    \begin{cases}\displaystyle
        \begin{array}{rclll}
            \primetemps{\bareps{y}} - \Delta\bareps{y} & = & \bareps{u} &
            \text{in} & Q,\\
            \\
            \bareps{y} & = & 0 & \text{on} & \Sigma,\\
            \\
            \condinitiale{\bareps{y}} & = & {\bar{y}_{0}}_{\veps} &
            \text{in} & \Omega,
        \end{array}
    \end{cases}
\end{equation*}
and the estimate
\begin{equation*}
    \normeohm{\condfinale{\bareps{y}}} < \veps.
\end{equation*}
So, given $v\in\uad$ and $\lambda\in{\Rbb}^{*}$; we have:
\begin{equation*}
    \begin{split}
        J_{\veps}\!\left({%
            \bareps{u} + \lambda\left({v - \bareps{u}}\right)
        }\right) &= \dfrac{1}{2}\normecq{%
            y\!\left({%
                \bareps{u} + \lambda\left({v - \bareps{u}}\right),
                {\bar{y}_{0}}_{\veps}
            }\right) - z_{d}
        } + \dfrac{N}{2}\normecq{%
            \bareps{u} + \lambda\left({v - \bareps{u}}\right)
        }\\
        &\qquad + \dfrac{1}{2\veps}\normecohm{%
            {\bar{y}_{0}}_{\veps} - \condinitiale{y}
        }\\
        & = \dfrac{1}{2}\normecq{%
            \bareps{y} - z_{d} + \lambda{\fieps}
        } + \dfrac{N}{2}\normecq{%
            \bareps{u} + \lambda\left({v - \bareps{u}}\right)
        }\\
        &\qquad + \dfrac{1}{2\veps}\normecohm{%
            {\bar{y}_{0}}_{\veps} - \condinitiale{y}
        }\\
        & = J_{\veps}\!\left({\bareps{u}}\right) +
        \dfrac{{\lambda}^{2}}{2}\left({%
            \normecq{\fieps} + N\normecq{v - \bareps{u}}
        }\right)\\
        &\qquad + \lambda\scalaireq{%
            \bareps{y} - z_{d}
        }{%
            \fieps{}
        } + \lambda N\scalaireq{%
            \bareps{u}
        }{%
            v - \bareps{u}
        },
    \end{split}
\end{equation*}
which gives
\begin{equation*}
    {\left.{%
        \dfrac{\diff}{\diff{\lambda}}J_{\veps}\!\left({%
            \bareps{u} + \lambda\left({v - \bareps{u}}\right)
        }\right)
    }\right|}_{\lambda = 0} = \scalaireq{%
        \bareps{y} - z_{d}
    }{%
        \fieps{}
    } + N\scalaireq{\bareps{u}}{v - \bareps{u}},
\end{equation*}
where $\fieps = y\!\left({v - \bareps{u}, {\bar{y}_{0}}_{\veps}}\right) -
y\!\left({0, {\bar{y}_{0}}_{\veps}}\right)$ is defined by
\begin{equation}\label{eq:control125}
    \begin{cases}\displaystyle
        \begin{array}{rclll}
            \primetemps{\fieps} - \Delta\fieps & = & v - \bareps{u} &
            \text{in} & Q,\\
            \\
            \fieps & = & 0 & \text{on} & \Sigma,\\
            \\
            \condinitiale{\fieps} & = & 0 & \text{in} & \Omega.
        \end{array}
    \end{cases}
\end{equation}
Hence, with the first-order optimality condition of Euler-Lagrange, we
obtain that the approached optimal control $\bareps{u}$ is the unique
element of $\uad$ satisfying
\begin{equation}\label{eq:control126}
    \scalaireq{%
        \bareps{y} - z_{d}
    }{%
        \fieps{}
    } + N\scalaireq{%
        \bareps{u}
    }{%
        v - \bareps{u}
    } \geq 0,\qquad\forall\,v\in\uad.
\end{equation}
Let introduce here the adjoint state $p_{\veps}\in\Ldeux{Q}$ by
\begin{equation*}
    \begin{cases}\displaystyle
        \begin{array}{rclll}
            -\primetemps{p_{\veps}} - \Delta p_{\veps} & = & \bareps{y} -
            z_{d} & \text{in} & Q,\\
            \\
            p_{\veps} & = & 0 & \text{on} & \Sigma,\\
            \\
            \condfinale{p_{\veps}} & = & 0 & \text{in} & \Omega.
        \end{array}
    \end{cases}
\end{equation*}
It comes with~\eqref{eq:control125} that
\begin{equation*}
    \begin{array}{rclll}
        -\primetemps{p_{\veps}} - \Delta p_{\veps} & = & \bareps{y} - z_{d}
        & \text{in} & Q
    \end{array}
\end{equation*}
leads to
\begin{equation*}
    \begin{split}
         \scalaireq{%
            \bareps{y} - z_{d}
        }{%
            \fieps{}
        } & = -\scalaireq{%
            \primetemps{p_{\veps}}
        }{%
            \fieps{}
        } - \scalaireq{%
            \Delta p_{\veps}
        }{%
            \fieps{}
        }\\
        & = \scalaireohm{%
            \condinitiale{p_{\veps}}
        }{%
            \condinitiale{\fieps}
        } - \scalaireohm{%
            \condfinale{p_{\veps}}
        }{%
            \condfinale{\fieps}
        } + \scalaireq{%
            p_{\veps}
        }{%
            \primetemps{\fieps}
        }\\
        &\qquad - \scalaireq{%
            p_{\veps}
        }{%
            \Delta\fieps{}
        } - \scalaires{%
            \deriveenormale{p_{\veps}}
        }{%
            \fieps{}
        } + \scalaires{%
            p_{\veps}
        }{%
            \deriveenormale{\fieps}
        }\\
        & = \scalaireq{%
            p_{\veps}
        }{%
            \primetemps{\fieps} - \Delta\fieps{}
        } = \scalaireq{%
            p_{\veps}
        }{%
            v - \bareps{u}
        },
    \end{split}
\end{equation*}
so that the optimality condition~\eqref{eq:control126} reduces to
\begin{equation*}
    \begin{array}{rcl}
        \scalaireq{%
            p_{\veps} + N\bareps{u}
        }{%
            v - \bareps{u}
        } & \geq & 0,\qquad \forall\,v\in\uad.
    \end{array}
\end{equation*}

We thus obtain the following result.

\begin{theoreme}\label{thm:soapproche}%
    Let $\veps > 0$. The control $\bareps{u}$ is unique solution
    of~\eqref{eq:control87} if and only if the quadruplet
    \begin{equation*}
        \left\{{%
            {\bar{y}_{0}}_{\veps}, \bareps{u}, \bareps{y}, p_{\veps}
        }\right\}\in \Ldeux{\Omega}\times {\left({\Ldeux{Q}}\right)}^{3}
    \end{equation*}
    is solution of the approached optimality system defined by the partial
    differential equations systems
    \begin{equation}
        \begin{cases}\displaystyle
            \begin{array}{rclll}
                \primetemps{\bareps{y}} - \Delta\bareps{y} & = & \bareps{u}
                & \text{in} & Q,\\
                \\
                \bareps{y} & = & 0 & \text{on} & \Sigma,\\
                \\
                \condinitiale{\bareps{y}} & = & {\bar{y}_{0}}_{\veps} &
                \text{in} & \Omega,
            \end{array}
        \end{cases}
    \end{equation}
    and
    \begin{equation}\label{eq:control130}
        \begin{cases}\displaystyle
            \begin{array}{rclll}
                -\primetemps{p_{\veps}} - \Delta p_{\veps} & = & \bareps{y}
                - z_{d} & \text{in} & Q,\\
                \\
                p_{\veps} & = & 0 & \text{on} & \Sigma,\\
                \\
                \condfinale{p_{\veps}} & = & 0 & \text{in} & \Omega,
            \end{array}
        \end{cases}
    \end{equation}
    the estimate
    \begin{equation}
        \begin{array}{rcl}
            \normeohm{%
                \condfinale{\bareps{y}}
            } & < & \veps,
        \end{array}
    \end{equation}
    and the variational inequality
    \begin{equation}
        \begin{array}{rcl}
            \scalaireq{%
                p_{\veps} + N\bareps{u}
            }{%
                v - \bareps{u}
            } & \geq & 0,\qquad \forall\,v\in\uad.
        \end{array}
    \end{equation}
\end{theoreme}

\subsection{Singular optimality system}\label{sec:singularso}

From the results obtained in Section~\ref{sec:convergence}, we have:
\begin{equation*}
    \begin{array}{rclll}
        \bareps{u} & \longrightarrow & u & \text{strongly in} & \Ldeux{Q}
    \end{array}
\end{equation*}
\begin{equation*}
    \begin{array}{rclll}
        \bareps{y} & \longrightarrow & y & \text{strongly in} & \Ldeux{Q},
    \end{array}
\end{equation*}
where $(u,y)$ is the optimal control-state pair
of~\eqref{eq:control80}\eqref{eq:control81}\eqref{eq:control82}.

So that,~\eqref{eq:control130} being well-posed in the sense of Hadamard,
it follows that there exists $p\in\Ldeux{Q}$ such that
\begin{equation*}
    \begin{array}{rclll}
        p_{\veps} & \longrightarrow & p & \text{strongly in} & \Ldeux{Q}.
    \end{array}
\end{equation*}
Then, we easily pass the results of Theorem~\ref{thm:soapproche} to the
limit, when $\veps\to 0$, to obtain that the strong singular optimaly
system characterizing the optimal control-state pair
of~\eqref{eq:control80}\eqref{eq:control81}\eqref{eq:control82} is as
specified below.

\begin{theoreme}\label{thm:sosingulier}%
    The control-state pair $(u,y)$ is unique solution of the control
    problem~\eqref{eq:control80}\eqref{eq:control81}\eqref{eq:control82} if
    and only if the triple
    \begin{equation*}
        \left\{{%
            u,y,p
        }\right\}\in {\left({\Ldeux{Q}}\right)}^{3}
    \end{equation*}
    is solution of the singular optimality system defined by the partial
    differential equation systems
    \begin{equation}
        \begin{cases}\displaystyle
            \begin{array}{rclll}
                \primetemps{y} - \Delta y & = & u & \text{in} & Q,\\
                \\
                y & = & 0 & \text{on} & \Sigma,\\
                \\
                \condfinale{y} & = & 0 & \text{in} & \Omega,
            \end{array}
        \end{cases}
    \end{equation}
    and
    \begin{equation}
        \begin{cases}\displaystyle
            \begin{array}{rclll}
                -\primetemps{p} - \Delta p & = & y - z_{d} & \text{in} &
                Q,\\
                \\
                p & = & 0 & \text{on} & \Sigma,\\
                \\
                \condfinale{p} & = & 0 & \text{in} & \Omega,
            \end{array}
        \end{cases}
    \end{equation}
    and the variational inequality
    \begin{equation}
        \begin{array}{rcll}
            \scalaireq{%
                p + Nu
            }{%
                v - u
            } & \geq & 0,\qquad\forall\,v\in\uad.
        \end{array}
    \end{equation}
\end{theoreme}


\section{Conclusion}

In this work, we succeed in characterizing the optimal control-state pair
of the control problem for the ill-posed backwards heat equation, using the
controllability concept. The method consists in interpreting the initial
problem as an inverse problem, we are also saying a controllability
problem. An approach that allows us to obtain a strong and decoupled
singular optimality system. As expected, the approach here proposed does
well without using an additional assumption of the class of the regularity
assumption~\eqref{eq:dorvilleassumption} used in~\cite{dorville} and the
Slater-type one~\eqref{eq:slater-type-assumption}. All that is required is
the following
\begin{equation}
    (u,y):\quad \condinitiale{y}\in\Ldeux{\Omega}.
\end{equation}

Finally, in view of the similar results obtained for the ill-posed Cauchy
system for elliptic (see~\cite{ownElliptic} and~\cite{ownAAA}), parabolic
(see~\cite{ownParabolic}) and hyperbolic (see~\cite{ownhyperbolic})
operators,
% (cf.~\cite{ownElliptic},~\cite{ownAAA},~\cite{ownParabolic}
% and~\cite{ownhyperbolic}),
the controllability method here proposed seems
relevant for control problems (of singular distributed systems) which
require recourse to Slater-type assumptions such
as~\eqref{eq:slater-type-assumption}.

\nocite{*}
\printbibliography{}

\end{document}
