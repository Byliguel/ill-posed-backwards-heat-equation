%   Environment     :   Ubuntu 24.04 LTS ~ "Noble Numbat"
%                       Gnome 46
%                       TeXLive 2023
%
%   Platform        :   x86_64-pc-linux-gnu (64bits)
%
%   Author          :   Bylli Andre Guel ~ byliguel@gmail.com
%
%   File            :   main.tex
%
%   Description     :   Main TeX source for research paper on the
%   retrograde heat equation
%       ~/Documents/research/postDoc-projects/retrograde-heat-equation/
%
%   Created date    :   2024-06-11 ~ 17:01
%   Version         :   v24.06.0
%   Last revision   :   ...
%=========================================================================

\documentclass[a4paper,12pt]{article}

\usepackage[utf8]{inputenc}
\usepackage[T1]{fontenc}
\usepackage[french]{babel}

\usepackage[left=2cm,right=2cm,top=2cm,bottom=2cm]{geometry}

\usepackage{amsmath,amssymb,amsfonts,mathrsfs,amsthm,shortcommands}
\usepackage{textcomp,pifont,authblk,enumitem}

\usepackage{biblatex}
\addbibresource{biblio.bib}
\usepackage{csquotes}

% \usepackage[notref,notcite]{showkeys}

\newtheorem{definition}{Définition}
\newtheorem{proposition}{Proposition}
\newtheorem{corollaire}{Corollaire}
\newtheorem{lemme}{Lemme}
\newtheorem{remarque}{Remarque}
\newtheorem{theoreme}{Théorème}

\setlist[itemize,1]{label=\tiny\Pisymbol{pzd}{110}}
\setlist[itemize,2]{label=\tiny$\blacktriangleright$}
\setlist[itemize,3]{label=\tiny\textbullet}

% \newcommand{\ie}{\textit{i.e.}}

\usepackage[hidelinks]{hyperref}
\hypersetup{%
    citecolor=black,%
    filecolor=black,%
    colorlinks=true,%
    linkcolor=black,%
    urlcolor=black,%
    pdftitle={Control of Singular Distributed Systems by Controllability:
    The Ill-Posed Backwards Heat Equation}
}

\title{Control of Singular Distributed Systems by Controllability: The
Ill-Posed Backwards Heat Equation}

% \author{Bylli André Guel}

\author[,1,2]{Bylli
André~\textsc{Guel}\thanks{bylli.guel@uv.bf\,/\,01~BV~30770~Ouagadougou~01,~Burkina~Faso}}
\affil[1]{Virtual University of Burkina Faso,\authorcr{} Ouagadougou,
Burkina~Faso}
\affil[2]{Laboratoire d'Analyse Numérique, d'Informatique et de
BIOmathématiques (LANIBIO), Burkina Faso}

\author[3,4]{Diaraf~\textsc{Seck}}
\affil[3]{Université Cheikh Anta Diop, FASEG, BP~16889, Dakar, Fann,
Sénégal}
\affil[4]{Laboratoire de Mathématiques de la Décision et d'Analyse
Numérique (LMDAN), Sénégal}

\begin{document}

\parindent=0pt{}
\parskip=6pt{}

\maketitle{}

\begin{abstract}
    This paper deals with the ill-posed backwards heat equation.

    Dans cet article, nous traitons du problème de contrôle de l'équation
    de la chaleur rétrograde. Nous proposons, pour ce faire, la méthode
    dite de contrôlabilité. Le point de vue adopté, consistant à
    interpréter le problème initial comme un problème inverse, nous permet
    d'obtenir un système d'optimalité singulier découplé fort pour le couple
    contrôle-état optimal, mais aussi de proposer un critère d'existence
    d'une solution régulière à l'équation de la chaleur rétrograde. Il
    importe de noter que ces résultats sont obtenus sans recours à une
    hypothèse de type Slater, hypothèse dont on avait du mal à se défaire
    jusqu'alors.

    \vspace{\baselineskip}
    \textbf{Keywords:} Singular distributed systems, Optimal control,
    Ill-posed backwards heat equation, Controllability, Inverse Problem.
\end{abstract}

\section{Introduction}

On considère $\Omega$ un ouvert borné de ${\Rbb}^{n}$, de frontière
${\Gamma}$ deux fois continûment différentiable avec $\Omega$ localement
d'un seul côté de $\Gamma$; soit que $\bar{\Omega}$ est une variété à bord
de classe ${\Cscr}^{2}$.

Pour $T > 0$, on note $\displaystyle Q = \Omega\times(0,T)$, $\displaystyle
\Sigma = \Gamma\times (0,T)$, et $\displaystyle {\Uscr}_{ad}$ un
sous-ensemble convexe fermé non vide de $\Ldeux{Q}$.

Étant donné $v\in\Ldeux{Q}$, on considère le problème

\begin{equation}\label{eq:initial-problem}
    \begin{cases}\displaystyle
        \begin{array}{rclll}
            \primetemps{z} - \Delta z & = & v & \text{dans} & Q,\\
            \\
            z & = & 0 & \text{sur} & \Sigma,\\
            \\
            \condfinale{z} & = & 0 & \text{dans} & \Omega,
        \end{array}
    \end{cases}
\end{equation}
étudiant l'évolution, dans le domaine $\Omega$ et à l'instant $t\in (0,T)$,
de la chaleur (la température) $z$ connaissant la valeur de celle-ci à
l'instant final $T$.

Il est bien connu que le problème~\eqref{eq:initial-problem}, dit équation
de la chaleur rétrograde, n'admet pas de solution pour des données
initiales quelconques.

On considère donc a priori les couples $(v,z)\in
{\left({\Ldeux{Q}}\right)}^{2}$ satisfaisant~\eqref{eq:initial-problem},
disant de ceux-ci qu'ils constituent l'ensemble des couples contrôle-état.

Un couple contrôle-état $(v,z)$ de~\eqref{eq:initial-problem} sera dit
admissible si on a en plus
\begin{equation*}\label{eq:admissible-control-state-pair}
    v\in {\Uscr}_{ad};
\end{equation*}
on notera alors $(v,z)\in \Ascr$, désignant par $\Ascr$ l'ensemble des
couples contrôle-état admissibles.

Supposant que l'ensemble $\Ascr$ des couples contrôle-état admissibles est
non vide, on introduit la fonctionnelle
\begin{equation*}\label{eq:initial-cost-function}
    J(v,z) = \dfrac{1}{2}\normecq{z - z_{d}} + \dfrac{N}{2}\normecq{v},
\end{equation*}
avec $z_{d}\in\Ldeux{Q}$ et $N > 0$.

On s'intéresse alors au problème de contrôle consistant à trouver un couple
contrôle-état
\begin{equation}\label{eq:initial-control-problem}
    (u,y)\in {\Ascr}_{ad}:\quad J(u,y) = \inf_{(v,z)\in
    {\Ascr}_{ad}}\,J(v,z).
\end{equation}

On a immédiatement le résultat suivant.
\begin{theoreme}%
    Le problème de contrôle optimal~\eqref{eq:initial-control-problem}
    admet une solution unique $(u,y)$ appelée couple contrôle-état optimal.
\end{theoreme}

\begin{proof}%
    De par sa structure, que la fonctionnelle $J$ est coercive, strictement
    convexe et semi-continue inférieurement. Ce qui, avec l'hypothèse de
    non vacuité du convexe fermé $\Ascr$ des couples contrôle-état
    admissibles, permet bien de conclure à l'existence et l'unicité du
    couple contrôle-état optimal $(u,y)$.
\end{proof}

Dès lors on établit (cf.~\cite{lions2}), avec la condition d'optimalité du
premier ordre d'Euler-Lagrange, que le couple optimal $(u,y)$ est
caractérisé par
\begin{equation*}\label{eq:initial-variational-inequation}
    \begin{array}{rcl}
        \scalaireq{%
            y - z_{d}
        }{%
            z - y
        } + N\scalaireq{u}{v - u} & \geq & 0,\qquad \forall\,(v,z)\in\Ascr.
    \end{array}
\end{equation*}
Le problème consiste alors à trouver un système d'optimalité découplé
caractérisant $(u,y)$.

Une méthode classique en la matière est la méthode de pénalisation
introduite par J.-L.~Lions dans~\cite{lions2}. Celle-ci consiste à
approcher $(u,y)$ par un problème pénalisé. Plus précisément, pour $\veps >
0$, on définit la fonction coût pénalisée
\begin{equation*}\label{eq:penalized-cost-function}
    J_{\veps}\!\left({v,z}\right) = J(v,z) + \dfrac{1}{\veps}\normecq{%
        \primetemps{z} - \Delta z - v
    }.
\end{equation*}
On établit alors que le couple optimal $\left({u_{\veps},z_{\veps}}\right)$
correspondant à cette nouvelle fonction coût converge vers le couple
optimal $(u,y)$. On a donc un procédé d'approximation théorique du couple
optimal $(u,y)$. À l'aide des conditions du premier ordre d'Euler-Lagrange,
on caractérise le couple optimal approché
$\left({u_{\veps},y_{\veps}}\right)$ par un système d'inégalités
variationnelles qu'on interprète en un système d'optimalité approché après
introduction de l'état adjoint approché
\begin{equation*}
    p_{\veps} = -\dfrac{1}{\veps}\left({%
        \primetemps{y_{\veps}} - \Delta y_{\veps} - u_{\veps}
    }\right).
\end{equation*}
Le point essentiel de cette technique consiste en l'utilisation de la
méthode des estimations \textit{a priori} pour l'obtention d'un système
d'optimalité découplé fort par passage à la limite dans le système
d'optimalité approché. Mais cette ultime étape requiert l'hypothèse de type
Slater que
\begin{equation}\label{eq:slater-type-assumption}
    \text{“}{\Uscr}_{ad}\ \text{soit d'intérieur non vide dans}\
    \Ldeux{Q}\text{”}.
\end{equation}
Mais dans bien de situations, la condition de Slater n'est pas vérifiée; on
peut citer en exemple le cas
\begin{equation*}
    \uad = {\left({\Ldeux{Q}}\right)}^{+} = \left\{{%
        v\in\Ldeux{Q}:\ v\geq 0
    }\right\}.
\end{equation*}
Il apparaît donc pertinent de savoir faire sans cette hypothèse très peu
réaliste.

Pour ce faire, R.~Dorville, O.~Nakoulima et A.~Omrane proposent
dans~\cite{dorville}, la notion de contrôle à moindres regrets, appliquée à
une régularisée elliptique à données manquantes du
problème~\eqref{eq:initial-problem}. Une approche qui leur permet de
caractériser le contrôle à moindres regrets $u^{\gamma}$ via un système
d'optimalité singulier découplé fort. Cela, certes sans recours à
l'hypothèse~\eqref{eq:slater-type-assumption}, mais au détriment de la
condition finale
\begin{equation*}
    \begin{array}{rclll}
        \condfinale{y^{\gamma}} & = & 0 & \text{dans} & \Omega,
    \end{array}
\end{equation*}
soit plus précisément que l'état associé au contrôle à moindres regrets
$u^{\gamma}$ est alors tel qu'en toute généralité
\begin{equation*}
    \begin{array}{rclll}
        \condfinale{y^{\gamma}} & \neq & 0 & \text{dans} & \Omega.
    \end{array}
\end{equation*}
Pour contourner le problème,
les auteurs introduisent la notion de correcteur d'ordre $0$, laquelle
appelle à l'hypothèse de régularité
\begin{equation}\label{eq:dorvilleassumption}
    \condfinale{y^{\gamma}}\in H_{0}^{1}(\Omega):\
    \primetemps{y^{\gamma}}\in \Ldeux{Q}.
\end{equation}
Laquelle hypothèse permet d'obtenir la caractérisation ci-dessous précisé
du couple contrôle-état optimal initial.

\begin{theoreme}[cf.~\cite{dorville}]%
    Le contrôle sans regret $u$ du
    problème~\eqref{eq:initial-problem}\eqref{eq:initial-cost-function}\eqref{eq:initial-control-problem}
    est caractérisé par l'unique $\left\{{u,y,\rho,p,\xi}\right\}$ solution
    du système:
    \begin{equation}
        \begin{cases}\displaystyle
            \begin{array}{rclrclrclrclll}
                Ly & = & u, & L\rho & = & 0, & L^{*}p & = & y - z_{d} +
                \rho, & L^{*}\xi & = & y & \text{dans} & Q,\\
                y & = & 0, & \rho & = & 0, & p & = & 0, & \xi & = & 0, &
                \text{sur} & \Sigma,\\
                y(0) & = & 0, & \rho(0) & = & \lambda(0) & {} & {} & {} &
                {} & {} & {} & \text{dans} & \Omega,\\
                {} & {} & {} & {} & {} & {} & p(T) & = & 0 & \xi(T) & = & 0
                & \text{dans} & \Omega,
            \end{array}
        \end{cases}
    \end{equation}
    et on a l'inégalité variationnelle
    \begin{equation}
        \begin{array}{rcll}
            \scalaire{p + Nu}{v - u}{} & \geq & 0 & \forall\,v\in\uad,
        \end{array}
    \end{equation}
    avec
    \begin{equation*}
        \left|{%
            \begin{array}{l}
                L = \primetemps{} - \Delta\quad\text{et}\quad L^{*} =
                -\primetemps{} - \Delta\quad\text{son opérateur adjoint};\\
                u, y, p, \rho, \xi\in \Ldeux{]0,T[;\Ldeux{\Omega}},\quad
                \lambda(0)\in\Ldeux{\Omega}.
            \end{array}
        }\right.
    \end{equation*}
\end{theoreme}

En définitive, le contrôle sans regret ainsi obtenu ne satisfait pas la
condition finale
\begin{equation*}
    \begin{array}{rclll}
        \condfinale{y} & = & 0 & \text{dans} & \Omega,
    \end{array}
\end{equation*}
de sorte qu'en toute généralité le problème reste, du mieux de nos
connaissances, globalement ouvert.

Nous proposons à travers le présent article la méthode de contrôlabilité
pour l'analyse, sans recours à
l'hypothèse~\eqref{eq:slater-type-assumption}, du problème de contrôle de
l'équation de la chaleur rétrograde. Notons que cette méthode a permis de
proposer dans~\cite{ownElliptic}, puis~\cite{ownAAA}, une réponse à la même
question du contrôle, sans recours à~\eqref{eq:slater-type-assumption}, de
problèmes de contrôle du système de Cauchy mal posé pour opérateur
elliptique.

La suite de cet article se présente comme suit. La
Section~\ref{sec:controllability} est consacrée à l'interprétation annoncée
du problème initial comme un problème inverse. On y définit le problème dit
de contrôlabilité exacte que nous approchons, par argument de densité, par
un problème équivalent (celui-ci dit de contrôlabilité approchée). Dans la
Section~\ref{sec:controlproblem}, nous revenons au problème de
contrôle~\eqref{eq:initial-control-problem}, commençant par le régulariser
d'après les résultats de contrôlabilité précédemment obtenus. Après avoir
établit la convergence du procédé dans la Section~\ref{sec:convergence},
puis le système d'optimalité approché dans la
Section~\ref{sec:approachedso}, nous finissons dans la
Section~\ref{sec:singularso} par le système d'optimalité singulier.

\section{Controllability of the ill-posed backwards heat
equation}\label{sec:controllability}

In the present section, we introduce the controllability viewpoint here
proposed. Which consists, starting from the notion of controllability
introduced by J.-L.~Lions in~\cite[222]{lions1}, in interpreting the
initial state equation as an inverse problem (we also say a controllability
problem).

We establish that, when it exists, the solution of the ill-posed backwards
heat equation~\eqref{eq:initial-problem} can be approximated by that of the
approached controllability problem. Which implicitly allows us to propose a
necessary and sufficient condition for the existence of a regular solution
to the problem~\eqref{eq:initial-problem}.

We therefore interpret~\eqref{eq:initial-problem} as an inverse problem
whose observation objective consists in the final condition
\begin{equation*}
    \begin{array}{rclll}
        \condfinale{z} & = & 0 & \text{in} & \Omega.
    \end{array}
\end{equation*}

More precisely, we consider the following problem
\begin{equation*}\label{eq:inverse-problem-1}
    \begin{cases}\displaystyle
        \begin{array}{rclll}
            \primetemps{y} - \Delta y & = & v & \text{in} & Q,\\
            \\
            y & = & 0 & \text{on} & \Sigma,
        \end{array}
    \end{cases}
\end{equation*}
posing the problem of finding $\displaystyle y_{0}\in\Ldeux{\Omega}$ such
that if
\begin{equation}\label{eq:inverse-problem-2}
    \begin{cases}\displaystyle
        \begin{array}{rclll}
            \primetemps{y} - \Delta y & = & v & \text{in} & Q,\\
            \\
            y & = & 0 & \text{on} & \Sigma,\\
            \\
            \condinitiale{y} & = & y_{0} & \text{in} & \Omega,
        \end{array}
    \end{cases}
\end{equation}
then
\begin{equation}\label{eq:inverse-problem-3}
    \begin{array}{rclll}
        \condfinale{y} & = & 0 & \text{in} & \Omega.
    \end{array}
\end{equation}
We say that~\eqref{eq:inverse-problem-2}\eqref{eq:inverse-problem-3}
constitute an exact controllability problem associated with the ill-posed
backwards heat equation~\eqref{eq:initial-problem}.

\begin{remarque}%
    The controllability problem is well defined. Indeed, it is well known
    that for any $v\in\Ldeux{Q}$ and
    $y_{0}\in\Ldeux{\Omega}$,~\eqref{eq:inverse-problem-2} is well posed in
    the sense of Hadamard. That is to say that its admits a unique solution
    \begin{equation*}\label{eq:well-posedness-heat-equation}
        y\!\left({v,y_{0}}\right)\in \Vbb :=
        \Cscr\!\left({[0,T];\Ldeux{\Omega}}\right)
    \end{equation*}
    which depends continuously on the data. We deduce from this, after
    possible modification on a set of zero measure, that the solution
    $y\!\left({v,y_{0}}\right)$ of~\eqref{eq:inverse-problem-2} merges with
    a continuous function from $(0,T)$ to $\Ldeux{\Omega}$.

    Thus we can indeed speak of the final value
    $y\!\left({\cdot,T;v,y_{0}}\right)$ of $y\!\left({v,y_{0}}\right)$ in
    $\Omega$.
\end{remarque}

\begin{remarque}%
    Let us denote
    \begin{itemize}
        \item $y^{v}$ the unique solution of~\eqref{eq:inverse-problem-2};
        \item $y_{0}^{v}\in \Vbb \subset \Ldeux{Q}$ that of
            \begin{equation*}\label{eq:null-initial-condition}
                \begin{cases}\displaystyle
                    \begin{array}{rclll}
                        \primetemps{y_{0}^{v}} - \Delta y_{0}^{v} & = & v &
                        \text{in} & Q,\\
                        \\
                        y_{0}^{v} & = & 0 & \text{on} & \Sigma,\\
                        \\
                        y_{0}^{v}\!\left({\cdot,0}\right) & = & 0 &
                        \text{in} & \Omega,
                    \end{array}
                \end{cases}
            \end{equation*}
        \item and $y^{0}\in \Vbb \subset \Ldeux{Q}$ that of
            \begin{equation}\label{eq:null-source-term}
                \begin{cases}\displaystyle
                    \begin{array}{rclll}
                        \primetemps{y^{0}} - \Delta y^{0} & = & 0 &
                        \text{in} & Q,\\
                        \\
                        y^{0} & = & 0 & \text{on} & \Sigma,\\
                        \\
                        y^{0}\!\left({\cdot,0}\right) & = & y_{0} &
                        \text{in} & \Omega.
                    \end{array}
                \end{cases}
            \end{equation}
    \end{itemize}
    Then, the mapping
    \begin{equation*}
        \left({v,y_{0}}\right)\longmapsto y^{v} = y_{0}^{v} + y^{0},
    \end{equation*}
    being linear and continuous from $\Ldeux{Q}\times \Ldeux{Q}$ to
    $\Vbb\subset \Ldeux{Q}$, we deduce that the exact controllability
    problem~\eqref{eq:inverse-problem-2}\eqref{eq:inverse-problem-3} is
    equivalent to the following
    \begin{equation}\label{eq:inverse-problem-4}
        \begin{cases}\displaystyle
            \text{find}\ y_{0}\in \Ldeux{\Omega}\ \text{such that:}\\
            \text{if}\ y^{0}\ \text{is solution
            of}~\eqref{eq:null-source-term},\ \text{then}\\
            y^{0}\!\left({\cdot,T}\right) = -y_{0}^{v}\!\left({\cdot,
            T}\right)\quad \text{in}\ \Omega.
        \end{cases}
    \end{equation}
\end{remarque}

% \begin{remarque}\label{rq:naturalassumption}%
%     Avec ce qui précède, il apparaît naturel de considérer, dans le cadre
%     du problème de contrôle, que c'est l'ensemble
%     \begin{equation*}
%         \left\{{%
%             (v,z)\in\Ascr:\ \condinitiale{z}\in\Ldeux{\Omega}
%         }\right\}
%     \end{equation*}
%     qui est non vide et non seulement $\Ascr$.
% \end{remarque}

We approach the exact controllability problem~\eqref{eq:inverse-problem-4}
by density argument as specified below.

\begin{proposition}\label{propo:controllability-result-1}%
    When the initial data $y_{0}$ traverses $\Ldeux{\Omega}$, the set
    \begin{equation*}\label{eq:controllability-result-1}
        E = \left\{{%
            y^{0}\!\left({\cdot,T}\right)\,;\ y_{0}\in\Ldeux{\Omega}
        }\right\},
    \end{equation*}
    described by the final values of the solution $y^{0}$
    of~\eqref{eq:null-source-term}, is dense in $\Ldeux{\Omega}$.
\end{proposition}

\begin{proof}%
    It is clear that the set $E$ constitute a vector subspace of
    $\Ldeux{\Omega}$. From where, by the Hahn-Banach Theorem, $E$ is dense
    in $\Ldeux{\Omega}$ if and only if $E^{\perp} = \{0\}$.

    Let us consider $k\in E^{\perp}$; so we have
    \begin{equation*}\label{eq:proof-controllability-26}
        \forall\,y_{0}\in\Ldeux{\Omega},\qquad
        \scalaireohm{k}{y^{0}\!\left({\cdot,T}\right)} = 0.
    \end{equation*}
    But it comes from~\eqref{eq:null-source-term} that, for any test
    function $\vphi\in\Dscr(Q)$, we have:
    \begin{equation*}
        \begin{split}
            \scalaireq{\primetemps{y^{0}} - \Delta y^{0}}{\vphi} = 0 &\iff
            \scalaireq{\primetemps{y^{0}}}{\vphi} - \scalaireq{\Delta
            y^{0}}{\vphi} = 0\\
            &\iff \scalaireohm{%
                y^{0}\!\left({\cdot,T}\right)
            }{%
                \vphi\!\left({\cdot,T}\right)
            } - \scalaireohm{%
                y_{0}
            }{%
                \vphi\!\left({\cdot,0}\right)
            }\\
            &\qquad - \scalaireq{y^{0}}{\primetemps{\vphi}} - \scalaireq{y^{0}
            }{%
                \Delta \vphi
            } + \scalaires{y^{0}
            }{\deriveenormale{\vphi}
            }\\
            &\qquad\qquad - \scalaires{\deriveenormale{y^{0}}}{\vphi} = 0
        \end{split}
    \end{equation*}
    \ie{}
    \begin{equation}\label{eq:controllability-proof-30}
        \begin{split}
            \scalaireohm{%
                y^{0}\!\left({\cdot,T}\right)
            }{%
                \vphi\!\left({\cdot,T}\right)
            } &- \scalaireohm{%
                y_{0}
            }{%
                \vphi\!\left({\cdot,0}\right)
            } - \scalaireq{%
                y^{0}
            }{%
                \primetemps{\vphi}
            }\\
            &\qquad - \scalaireq{%
                y^{0}
            }{%
                \Delta\vphi
            } - \scalaires{%
                \deriveenormale{y^{0}}
            }{\vphi} = 0.
        \end{split}
    \end{equation}
    Choosing in the above, $\vphi$ such that
    \begin{equation}\label{eq:controllability-proof-31}
        \begin{cases}\displaystyle
            \begin{array}{rclll}
                -\primetemps{\vphi} - \Delta\vphi & = & 0 & \text{in} & Q,\\
                \\
                \vphi & = & 0 & \text{on} & \Sigma,\\
                \\
                \vphi\!\left({\cdot,T}\right) & = & k & \text{in} &
                \Omega,
            \end{array}
        \end{cases}
    \end{equation}
    it comes that~\eqref{eq:controllability-proof-30} is equivalent to
    \begin{equation}\label{eq:controllability-proof-32}
        \scalaireohm{k}{y^{0}\!\left({\cdot,T}\right)} - \scalaireohm{%
            y_{0}
        }{%
            \vphi\!\left({\cdot,0}\right)
        } = 0,
    \end{equation}
    where
    \begin{equation*}
        k\in E^{\perp}\iff \scalaireohm{k}{y^{0}\!\left({\cdot,T}\right)} =
        0.
    \end{equation*}
    Thus~\eqref{eq:controllability-proof-32} becomes
    \begin{equation}\label{eq:controllability-proof-33}
        \forall\,y_{0}\in\Ldeux{\Omega},\qquad \scalaireohm{y_{0}}{%
            \vphi\!\left({\cdot,0}\right)
        } = 0.
    \end{equation}
    But we can still choose, in~\eqref{eq:controllability-proof-33},
    $\displaystyle y_{0} = \vphi\!\left({\cdot,0}\right)$ in $\Omega$, and
    then it follows
    \begin{equation*}
        \normecohm{\vphi\!\left({\cdot,0}\right)} = 0\qquad\ie\qquad
        \begin{array}{rclll}
            \vphi\!\left({\cdot,0}\right) & = & 0 & \text{in} & \Omega.
        \end{array}
    \end{equation*}
    Which brings, with~\eqref{eq:controllability-proof-31}, that $\vphi$ is
    solution of
    \begin{equation*}
        \begin{cases}\displaystyle
            \begin{array}{rclll}
                -\primetemps{\vphi} - \Delta\vphi & = & 0 & \text{in} &
                Q,\\
                \\
                \vphi & = & 0 & \text{on} & \Sigma,\\
                \\
                \vphi\!\left({\cdot,0}\right) & = & 0 & \text{in} &
                \Omega,
            \end{array}
        \end{cases}
    \end{equation*}
    that is to say that $\vphi\equiv 0$ and consequently that
    \begin{equation*}
        \begin{array}{rclcll}
            \vphi\!\left({\cdot,T}\right) & = & k & = 0 & \text{in} &
            \Omega.
        \end{array}
    \end{equation*}
    From where we deduce that $E^{\perp} = \{0\}$ and therefore that $E$ is
    dense in $\Ldeux{\Omega}$.
\end{proof}

The following result is then immediate.

\begin{corollaire}\label{coro:controllability-result-2}%
    For any $\veps > 0$ and $\kappa\in\Ldeux{\Omega}$, there exists
    ${y_{0}}_{\veps}\in\Ldeux{\Omega}$ such that the solution $\displaystyle
    y_{\veps}^{0}\in \Vbb$ of
    \begin{equation}\label{eq:controllability-result-3}
        \begin{cases}\displaystyle
            \begin{array}{rclll}
                \primetemps{y_{\veps}^{0}} - \Delta y_{\veps}^{0} & = & 0 &
                \text{in} & Q,\\
                \\
                y_{\veps}^{0} & = & 0 & \text{on} & \Sigma,\\
                \\
                y_{\veps}^{0}\!\left({\cdot,0}\right) & = & {y_{0}}_{\veps}
                & \text{in} & \Omega,
            \end{array}
        \end{cases}
    \end{equation}
    also satisfies
    \begin{equation*}\label{eq:controllability-result-4}
        \normeohm{y_{\veps}^{0}\!\left({\cdot,T}\right) - \kappa} < \veps.
    \end{equation*}
\end{corollaire}

From which it also follows that

\begin{corollaire}\label{coro:controllability-result-3}%
    Given $v\in\Ldeux{Q}$ and $\veps > 0$, there exists ${y_{0}}_{\veps}\in
    \Ldeux{\Omega}$ such that the solution $\displaystyle y_{\veps}^{v}\in
    \Vbb$ of
    \begin{equation}\label{eq:controllability-result-5}
        \begin{cases}\displaystyle
            \begin{array}{rclll}
                \primetemps{y_{\veps}^{v}} - \Delta y_{\veps}^{v} & = & v &
                \text{in} & Q,\\
                \\
                y_{\veps}^{v} & = & 0 & \text{on} & \Sigma,\\
                \\
                y_{\veps}^{v}\!\left({\cdot,0}\right) & = & {y_{0}}_{\veps}
                & \text{in} & \Omega,
            \end{array}
        \end{cases}
    \end{equation}
    satisfies
    \begin{equation*}\label{eq:controllability-result-6}
        \normeohm{y_{\veps}^{v}\!\left({\cdot,T}\right)} < \veps.
    \end{equation*}
\end{corollaire}

\begin{proof}%
    Let $v\in\Ldeux{Q}$ and $\veps > 0$. We know that there exists a unique
    $y_{0}^{v}\in \Vbb$, almost everywhere equal to a continuous mapping
    from $(0,T)$ to $\Ldeux{\Omega}$, solution of
    \begin{equation*}\label{eq:controllability-proof-39}
        \begin{cases}\displaystyle
            \begin{array}{rclll}
                \primetemps{y_{0}^{v}} - \Delta y_{0}^{v} & = & v &
                \text{in} & Q,\\
                \\
                y_{0}^{v} & = & 0 & \text{on} & \Sigma,\\
                \\
                y_{0}^{v}\!\left({\cdot,0}\right) & = & 0 & \text{in} &
                \Omega.
            \end{array}
        \end{cases}
    \end{equation*}
    It follows that, with Corollary~\ref{coro:controllability-result-2},
    there exists ${y_{0}}_{\veps}\in\Ldeux{\Omega}$ such that the solution
    $y_{\veps}^{0}\in \Vbb$ of~\eqref{eq:controllability-result-3}
    satisfies
    \begin{equation*}\label{eq:controllability-proof-40}
        \normeohm{%
            y_{\veps}^{0}\!\left({\cdot,T}\right) - \left({%
                -y_{0}^{v}\!\left({\cdot,T}\right)
            }\right)
        } < \veps.
    \end{equation*}
    Which allows us to conclude that $\displaystyle y_{\veps}^{v} =
    \left({%
        y_{0}^{v} + y_{\veps}^{0}
    }\right) \in \Vbb$ is unique solution
    of~\eqref{eq:controllability-result-5}, with
    \begin{equation*}
        \normeohm{y_{\veps}^{v}\!\left({\cdot,T}\right)} = \normeohm{%
            y_{0}^{\veps}\!\left({\cdot,T}\right) +
            y_{0}^{v}\!\left({\cdot,T}\right)
        } < \veps.
    \end{equation*}
\end{proof}

More over, we have the following theorem, characterizing the existence of a
regular solution to the ill-posed backwards heat equation.

\begin{theoreme}\label{thm:controllability-result-5}%
    Let $v\in\Ldeux{Q}$. The ill-posed backwards heat equation
    \begin{equation}\label{eq:initial-problem-2}
        \begin{cases}\displaystyle
            \begin{array}{rclll}
                \primetemps{z} - \Delta z & = & v & \text{in} & Q,\\
                \\
                z & = & 0 & \text{on} & \Sigma,\\
                \\
                z(\cdot,T) & = & 0 & \text{in} & \Omega,
            \end{array}
        \end{cases}
    \end{equation}
    admits a regular solution $z\in \Vbb$ if and only if the sequence
    $\displaystyle {\left({{y_{0}}_{\veps}}\right)}_{\veps}$ is bounded in
    $\Ldeux{\Omega}$.
\end{theoreme}

\begin{proof}%
    \begin{enumerate}
        \item Let $\veps > 0$. From
            Corollary~\ref{coro:controllability-result-2}, there exists
            ${y_{0}}_{\veps}\in \Ldeux{\Omega}$ such as $\displaystyle
            y_{\veps}^{v}\in \Vbb$ is solution of
            \begin{equation}\label{eq:proof-thm-controllability-42}
                \begin{cases}\displaystyle
                    \begin{array}{rclll}
                        \primetemps{y_{\veps}^{v}} - \Delta y_{\veps}^{v} &
                        = & v & \text{in} & Q,\\
                        \\
                        y_{\veps}^{v} & = & 0 & \text{on} & \Sigma,\\
                        \\
                        y_{\veps}^{v}\!\left({\cdot,0}\right) & = &
                        {y_{0}}_{\veps} & \text{in} & \Omega,
                    \end{array}
                \end{cases}
            \end{equation}
            with the estimate
            \begin{equation*}\label{eq:proof-thm-controllability-43}
                \normeohm{y_{\veps}^{v}\!\left({\cdot, T}\right)} < \veps.
            \end{equation*}
            Then, we generate sequences
            \begin{equation*}
                {\left({{y_{0}}_{\veps}}\right)}_{\veps}\subset
                \Ldeux{\Omega}\qquad\text{and}\qquad
                {\left({y_{\veps}^{v}}\right)}_{\veps}\subset \Vbb.
            \end{equation*}
            Let suppose that the sequence
            ${\left({{y_{0}}_{\veps}}\right)}_{\veps}$ is bounded in
            $\Ldeux{\Omega}$. It therefore follows, the homogeneous
            Dirichlet problem~\eqref{eq:proof-thm-controllability-42} for
            the heat equation being well-posed, that the sequence
            ${\left({y_{\veps}^{v}}\right)}_{\veps}$ is bounded in $\Vbb$,
            and also in $\Ldeux{Q}$.

            So we can extract from
            ${\left({{y_{0}}_{\veps}}\right)}_{\veps}$ and
            ${\left({y_{\veps}^{v}}\right)}_{\veps}$ subsequences, still
            denoted in the same way, which converge weakly in
            $\Ldeux{\Omega}$ and $\Vbb$, respectively.

            So that, there exist $\displaystyle y_{0}\in \Ldeux{\Omega}$
            and $\displaystyle y^{v}\in \Vbb$, such as
            \begin{equation*}\label{eq:proof-thm-controllability-44}
                \begin{cases}\displaystyle
                    \begin{array}{rclll}
                        {y_{0}}_{\veps} & \longrightarrow & y_{0} &
                        \text{weakly in} & \Ldeux{\Omega},\\
                        \\
                        y_{\veps}^{v} & \longrightarrow & y^{v} &
                        \text{weakly in} & \Vbb.
                    \end{array}
                \end{cases}
            \end{equation*}
            Thus we have on the one hand, that
            \begin{equation*}\label{eq:proof-thm-controllability-45}
                \normeohm{y_{\veps}^{v}\!\left({\cdot,T}\right)} < \veps
                \quad\text{and}\quad
                \begin{array}{rclll}
                    y_{\veps}^{v} & \longrightarrow & y^{v} & \text{weakly
                    in} & \Vbb
                \end{array}
            \end{equation*}
            imply, $y_{\veps}^{v}$ being almost everywhere equal to a
            continuous mapping from $(0,T)$ to $\Ldeux{\Omega}$, that
            \begin{equation}\label{eq:proof-thm-controllability-46}
                \begin{array}{rclll}
                    y^{v}\!\left({\cdot, T}\right) & = & 0 & \text{in} &
                    \Omega.
                \end{array}
            \end{equation}
            On the other hand, we have for any $\vphi\in\Dscr(Q)$ that:
            \begin{equation*}
                \begin{split}
                    {} & \primetemps{y_{\veps}^{v}} - \Delta y_{\veps}^{v}
                    = v\ \text{in}\ Q \iff \scalaireq{%
                        \primetemps{y_{\veps}^{v}} - \Delta y_{\veps}^{v}
                    }{\vphi} = \scalaireq{v}{\vphi}\\
                    &\qquad\iff \scalaireohm{%
                        \condfinale{y_{\veps}^{v}}
                    }{%
                        \condfinale{\vphi}
                    } - \scalaireohm{%
                        {y_{0}}_{\veps}
                    }{%
                        \condinitiale{\vphi}
                    }\\
                    &\qquad\qquad - \scalaireq{%
                        y_{\veps}^{v}
                    }{%
                        \primetemps{\vphi}
                    } - \scalaireq{%
                        y_{\veps}^{v}
                    }{%
                        \Delta\vphi{}
                    } - \scalaires{%
                        \deriveenormale{y_{\veps}^{v}}
                    }{%
                        \vphi{}
                    }\\
                    &\qquad\qquad\qquad = \scalaireq{v}{\vphi},
                \end{split}
            \end{equation*}
            so, by passing to the limit, with the continuity of the zero
            and one order trace operators,
            \begin{equation*}
                \begin{split}
                    {} & \scalaireohm{%
                        \condfinale{y^{v}}
                    }{%
                        \condfinale{\vphi}
                    } - \scalaireohm{%
                        y_{0}
                    }{%
                        \condinitiale{\vphi}
                    } - \scalaireq{%
                        y^{v}
                    }{%
                        \primetemps{\vphi}
                    }\\
                    &\qquad\qquad - \scalaireq{%
                        y^{v}
                    }{%
                        \Delta\vphi{}
                    } - \scalaires{%
                        \deriveenormale{y^{v}}
                    }{%
                        \vphi{}
                    } = \scalaireq{v}{\vphi}\\
                    &\qquad \iff \scalaireohm{%
                        \condinitiale{y^{v}} - y_{0}
                    }{%
                        \condinitiale{\vphi}
                    } + \scalaireq{%
                        \primetemps{y^{v}} - \Delta y^{v}
                    }{%
                        \vphi{}
                    }\\
                    &\qquad\qquad - \scalaires{%
                        y^{v}
                    }{%
                        \deriveenormale{\vphi}
                    } = \scalaireq{v}{\vphi}.
                \end{split}
            \end{equation*}
            This last equality being valid for any $\vphi\in\Dscr(Q)$, it
            follows that
            \begin{equation*}
                \begin{cases}\displaystyle
                    \begin{array}{rclll}
                        \primetemps{y^{v}} - \Delta y^{v} & = & v &
                        \text{in} & Q,\\
                        \\
                        y^{v} & = & 0 & \text{on} & \Sigma,\\
                        \\
                        \condinitiale{y^{v}} & = & y_{0} & \text{in} &
                        \Omega,
                    \end{array}
                \end{cases}
            \end{equation*}
            which implies, with~\eqref{eq:proof-thm-controllability-46},
            that
            \begin{equation*}
                \begin{cases}\displaystyle
                    \begin{array}{rclll}
                        \primetemps{y^{v}} - \Delta y^{v} & = & v &
                        \text{in} & Q,\\
                        \\
                        y^{v} & = & 0 & \text{on} & \Sigma,\\
                        \\
                        \condfinale{y^{v}} & = & 0 & \text{in} & \Omega,
                    \end{array}
                \end{cases}
            \end{equation*}
            that is to say that $y^{v}\in \Vbb$ is solution of the
            ill-posed backwards heat equation~\eqref{eq:initial-problem-2}.
        \item Now, let assume that the ill-posed backwards heat
            equation~\eqref{eq:initial-problem-2} admits a regular solution
            $z\in \Vbb$.

            Then, since $z\in \Vbb$ implies that $z$ is almost everywhere
            equal to a continuous function from $(0,T)$ to
            $\Ldeux{\Omega}$, we can, after possible modification on a set
            of zero measure, speak of the initial value $\displaystyle
            \condinitiale{z}\in\Ldeux{\Omega}$.

            Then, by choosing, for any $\veps > 0$,
            \begin{equation*}
                \begin{array}{rclll}
                    {y_{0}}_{\veps} & = & \condinitiale{z} & \text{in} &
                    \Omega,
                \end{array}
            \end{equation*}
            we obtain that the sequence
            ${\left({{y_{0}}_{\veps}}\right)}_{\veps}$, since constant, is
            bounded in $\Ldeux{\Omega}$.
    \end{enumerate}
\end{proof}

\section{La méthode de contrôlabilité}\label{sec:controlproblem}

Commençons par rappeler que nous sommes ici intéressés par le contrôle de
l'équation de la chaleur rétrograde. Soit, plus précisément, que pour
$v,z\in \Ldeux{Q}$ vérifiant
\begin{equation}\label{eq:control80}
    \begin{cases}\displaystyle
        \begin{array}{rclll}
            \primetemps{z} - \Delta z & = & v & \text{dans} & Q,\\
            \\
            z & = & 0 & \text{sur} & \Sigma,\\
            \\
            \condfinale{z} & = & 0 & \text{sur} & \Omega,
        \end{array}
    \end{cases}
\end{equation}
on pose
\begin{equation}\label{eq:control81}
    J(v,z) = \dfrac{1}{2}\normecq{z - z_{d}} + \dfrac{N}{2}\normecq{v},
\end{equation}
s'intéressant au problème de contrôle
\begin{equation}\label{eq:control82}
    \inf\!\left\{{%
        J(v,z);\ (v,z)\in\Ascr
    }\right\}.
\end{equation}

Comme souligné à l'introduction, le
problème~\eqref{eq:control80}\eqref{eq:control81}\eqref{eq:control82} admet
une solution unique $(u,y)$ dont la caractérisation, via un système
d'optimalité singulier découplé fort, et cela sans recours à l'hypothèse de
Slater~\eqref{eq:slater-type-assumption}, est le principal objet ici.

% \subsection{La méthode de contrôlabilité}

Pour ce faire, partant de l'hypothèse de non-vacuité de l'ensemble des
couples contrôle-état admissibles
pour~\eqref{eq:control80}\eqref{eq:control81}\eqref{eq:control82} et des
résultats de la section précédente, on a pour tous $v\in\Ldeux{Q}$ et
$\veps > 0$ qu'il existe
\begin{equation*}
    {y_{0}}_{\veps}\in\Ldeux{\Omega}\qquad\text{et}\qquad y_{\veps}^{v}\in
    \Vbb
\end{equation*}
% comme souligné à la Remarque~\eqref{rq:naturalassumption}, de
% l'hypothèse
% \begin{equation*}
%     \left\{{%
%         (v,z)\in \Ascr: \condinitiale{z} \in\Ldeux{\Omega}
%     }\right\}\neq \emptyset,
% \end{equation*}
% on a pour tous $v\in\Ldeux{Q}$ et $\veps > 0$ qu'il existe
% \begin{equation*}
%     {y_{0}}_{\veps}\in\Ldeux{\Omega}\qquad\text{et}\qquad y_{\veps}^{v}\in
%     \Vbb
% \end{equation*}
tels que
\begin{equation*}
    \begin{cases}\displaystyle
        \begin{array}{rclll}
            \primetemps{y_{\veps}^{v}} - \Delta y_{\veps}^{v} & = & v &
            \text{dans} & Q,\\
            \\
            y_{\veps}^{v} & = & 0 & \text{sur} & \Sigma,\\
            \\
            \condinitiale{y_{\veps}^{v}} & = & {y_{0}}_{\veps} &
            \text{dans} & \Omega,
        \end{array}
    \end{cases}
\end{equation*}
avec l'estimation
\begin{equation*}
    \normeohm{%
        \condfinale{y_{\veps}^{v}}
    } < \veps.
\end{equation*}
Supposant alors que la solution optimale $(u,y)$
de~\eqref{eq:control80}\eqref{eq:control81}\eqref{eq:control82} est telle
que $\displaystyle \condinitiale{y}\in\Ldeux{\Omega}$, on introduit la
fonctionnelle
\begin{equation*}
    J_{\veps}(v) = \dfrac{1}{2}\normecq{y_{\veps}^{v} - z_{d}} +
    \dfrac{N}{2}\normecq{v} + \dfrac{1}{2\veps}\normecohm{{y_{0}}_{\veps} -
    \condinitiale{y}},
\end{equation*}
et on s'intéresse au problème
\begin{equation}\label{eq:control87}
    \inf\!\left\{{%
        J_{\veps}(v);\ v\in\uad
    }\right\}.
\end{equation}
Le résultat suivant est immédiat.

\begin{proposition}%
    Pour tout $\veps > 0$, le problème de contrôle~\eqref{eq:control87}
    admet une solution unique: le contrôle optimal approché $\bareps{u}$.
\end{proposition}

\subsection{Convergence de la méthode}\label{sec:convergence}

Soit $\veps > 0$. On a existence et unicité du contrôle approché
$\bareps{u}$. Soit, avec les résultats de la section précédente, qu'il
existe
\begin{equation*}
    {\bar{y}_{0}}_{\veps}\in\Ldeux{\Omega}\qquad\text{et}\qquad
    \bareps{y}\in\Vbb
\end{equation*}
vérifiant
\begin{equation}\label{eq:control89}
    \begin{cases}\displaystyle
        \begin{array}{rclll}
            \primetemps{\bareps{y}} - \Delta\bareps{y} & = & \bareps{u} &
            \text{dans} & Q,\\
            \\
            \bareps{y} & = & 0 & \text{sur} & \Sigma,\\
            \\
            \condinitiale{\bareps{y}} & = & {\bar{y}_{0}}_{\veps} &
            \text{dans} & \Omega,
        \end{array}
    \end{cases}
\end{equation}
et l'estimation
\begin{equation*}
    \normeohm{\condfinale{\bareps{y}} } < \veps,
\end{equation*}
avec
\begin{equation*}
    J_{\veps}\!\left({\bareps{u}}\right)\leq J_{\veps}(v),\quad
    \forall\,v\in\uad.
\end{equation*}
En particulier donc
\begin{equation}\label{eq:control92}
    J_{\veps}\!\left({\bareps{u}}\right) \leq J_{\veps}(u),
\end{equation}
où $u$ est le contrôle optimal
pour~\eqref{eq:control80}\eqref{eq:control81}\eqref{eq:control82}.

On a en fait que $J_{\veps}(u)$ est indépendant de $\veps$. En effet, comme
l'état optimal $y$ associé à $u$ vérifie
$\condinitiale{y}\in\Ldeux{\Omega}$, on obtient en posant
${y_{0}^{*}}_{\veps} = \condinitiale{y}$ que $y$ satisfait
\begin{equation*}
    \begin{cases}\displaystyle
        \begin{array}{rclll}
            \primetemps{y} - \Delta y & = & u & \text{dans} & Q,\\
            \\
            y & = & 0 & \text{sur} & \Sigma,\\
            \\
            \condinitiale{y} & = & {y_{0}^{*}}_{\veps} & \text{dans} &
            \Omega
        \end{array}
    \end{cases}
\end{equation*}
avec
\begin{equation*}
    \begin{array}{rclll}
        \condfinale{y} & = & 0 & \text{dans} & \Omega,
    \end{array}
    \qquad\text{\textit{a fortiori}}\qquad \normeohm{\condfinale{y}} <
    \veps.
\end{equation*}
Il suit que $J_{\veps}(u)$ est défini, avec
\begin{equation*}
    \begin{split}
        J_{\veps}(u) &= \dfrac{1}{2}\normecq{y - z_{d}} +
        \dfrac{N}{2}\normecq{u} + \dfrac{1}{2\veps}\normecohm{%
            {y_{0}^{*}}_{\veps} - \condinitiale{y}
        }\\
        & = \dfrac{1}{2}\normecq{y - z_{d}} + \dfrac{N}{2}\normecq{u} =
        J(u,y).
    \end{split}
\end{equation*}
Ainsi~\eqref{eq:control92} devient
\begin{equation}\label{eq:control95}
    J_{\veps}\!\left({\bareps{u}}\right) \leq J_{\veps}(u) = J(u,y),
\end{equation}
et il vient qu'il existe des constantes $C_{i}\in{\Rbb}^{*}$ indépendantes
de $\veps$ telles qu'on a
\begin{equation*}
    \normeq{\bareps{y}} \leq C_{1},\qquad
    \normeq{\bareps{u}}\leq C_{2}\qquad\text{et}\qquad
    \normeohm{{\bar{y}_{0}}_{\veps}}\leq C_{3}.
\end{equation*}
On en déduit immédiatement qu'il existe $\hat{u}\in\Ldeux{Q}$,
$\hat{y}\in\Ldeux{Q}$ et ${\hat{y}}_{0}\in\Ldeux{\Omega}$ tels que
\begin{equation*}
    \begin{cases}\displaystyle
        \begin{array}{rcllll}
            \bareps{u} & \longrightarrow & \hat{u} & \text{dans} &
            \Ldeux{Q} & \text{faible},\\
            \bareps{y} & \longrightarrow & \hat{y} & \text{dans} &
            \Ldeux{Q} & \text{faible},\\
            {\bar{y}_{0}}_{\veps} & \longrightarrow & {\hat{y}}_{0} &
            \text{dans} & \Ldeux{\Omega} & \text{faible}.
        \end{array}
    \end{cases}
\end{equation*}
Mais encore,~\eqref{eq:control95} entraîne aussi
\begin{equation}\label{eq:control106}
    \normeohm{{\bar{y}_{0}}_{\veps} - \condinitiale{y}} \leq 2\veps\,C_{4},
\end{equation}
ce qui conduit, comme
\begin{equation}\label{eq:control106p}
    \begin{array}{rcllll}
        {\bar{y}_{0}}_{\veps} & \longrightarrow & {\hat{y}}_{0} &
        \text{dans} & \Ldeux{\Omega} & \text{faible},
    \end{array}
\end{equation}
à
\begin{equation}\label{eq:control107}
    \begin{array}{rclll}
        {\hat{y}}_{0} & = & \condinitiale{y} & \text{dans} & \Omega.
    \end{array}
\end{equation}
Alors, il vient avec~\eqref{eq:control89}, que pour tout
$\vphi\in\Dscr(Q)$,
\begin{equation*}
    \begin{split}
        \scalaireq{\primetemps{\bareps{y}}}{\vphi} &- \scalaireq{%
            \Delta\bareps{y}
        }{%
            \vphi%
        } = \scalaireq{\bareps{u}}{\vphi}\\
        &\iff \scalaireohm{%
            \condfinale{\bareps{y}}
        }{%
            \condfinale{\vphi}
        } - \scalaireohm{%
            \condinitiale{\bareps{y}}
        }{%
            \condinitiale{\vphi}
        } - \scalaireq{%
            \bareps{y}
        }{%
            \primetemps{\vphi}
        }\\
        &\qquad - \scalaireq{%
            \bareps{y}
        }{%
            \Delta\vphi%
        } - \scalaires{%
            \deriveenormale{\bareps{y}}
        }{%
            \vphi%
        } + \scalaires{%
            \bareps{y}
        }{%
            \deriveenormale{\vphi}
        } = \scalaireq{\bareps{u}}{\vphi}\\
        &\iff \scalaireohm{%
            \condfinale{\bareps{y}}
        }{%
            \condfinale{\vphi}
        } - \scalaireohm{%
            {\bar{y}_{0}}_{\veps}
        }{%
            \condinitiale{\vphi}
        } - \scalaireq{%
            \bareps{y}
        }{%
            \primetemps{\vphi}
        }\\
        &\qquad - \scalaireq{%
            \bareps{y}
        }{%
            \Delta\vphi%
        } - \scalaires{%
            \deriveenormale{\bareps{y}}
        }{%
            \vphi%
        } = \scalaireq{\bareps{u}}{\vphi},
    \end{split}
\end{equation*}
soit à la limite
\begin{equation*}
    \begin{split}
        \scalaireohm{%
            \condfinale{\hat{y}}
        }{%
            \condfinale{\vphi}
        } & - \scalaireohm{%
            {\hat{y}}_{0}
        }{%
            \condinitiale{\vphi}
        } - \scalaireq{%
            \hat{y}
        }{%
            \primetemps{\vphi}
        } - \scalaireq{%
            \hat{y}
        }{%
            \Delta\vphi%
        }\\
        &\qquad - \scalaires{%
            \deriveenormale{\hat{y}}
        }{%
            \vphi%
        } = \scalaireq{\hat{u}}{\vphi}\\
        &\iff \scalaireohm{%
            \condinitiale{\hat{y}} - {\hat{y}}_{0}
        }{%
            \condinitiale{\vphi}
        } + \scalaireq{%
            \primetemps{\hat{y}} - \Delta\hat{y}
        }{%
            \vphi%
        }\\
        &\qquad - \scalaires{%
            \hat{y}
        }{%
            \deriveenormale{\vphi}
        } = \scalaireq{\hat{u}}{\vphi},
    \end{split}
\end{equation*}
\ie{}
\begin{equation}\label{eq:control102}
    \begin{cases}\displaystyle
        \begin{array}{rclll}
            \primetemps{\hat{y}} - \Delta\hat{y} & = & \hat{u} &
            \text{dans} & Q,\\
            \\
            \hat{y} & = & 0 & \text{sur} & \Sigma,\\
            \\
            \condinitiale{\hat{y}} & = & {\hat{y}}_{0} & \text{dans} &
            \Omega.
        \end{array}
    \end{cases}
\end{equation}
Par ailleurs, la norme $\normeohm{\cdot}$ étant continue, \textit{a
fortiori} faiblement continue,
\begin{equation*}
    \begin{array}{rcllll}
        \bareps{y} & \longrightarrow & \hat{y} & \text{dans} &
        \Ldeux{Q} &  \text{faible}
    \end{array}\qquad\text{et}\qquad
    \normeohm{\condfinale{\bareps{y}}} < \veps
\end{equation*}
amènent
\begin{equation}\label{eq:control103}
    \begin{array}{rclll}
        \condfinale{\hat{y}} & = & 0 & \text{dans} & \Omega.
    \end{array}
\end{equation}
Ainsi~\eqref{eq:control102}~et~\eqref{eq:control103} permettent de conclure
que $\hat{y}\in\Ldeux{Q}$ satisfait
\begin{equation*}
    \begin{cases}\displaystyle
        \begin{array}{rclll}
            \primetemps{\hat{y}} - \Delta\hat{y} & = & \hat{u} &
            \text{dans} & Q,\\
            \\
            \hat{y} & = & 0 & \text{sur} & \Sigma,\\
            \\
            \condfinale{\hat{y}} & = & 0 & \text{dans} & \Omega,
        \end{array}
    \end{cases}
\end{equation*}
notant que $\hat{u}\in\uad$, puisque $\bareps{u}\in\uad$ et $\uad$ est
fermé donc fermé faible: le couple $\left({\hat{u},\hat{y}}\right)$ est
admissible
pour~\eqref{eq:control80}\eqref{eq:control81}\eqref{eq:control82}, et donc
\begin{equation}\label{eq:control105}
    J(u,y)\leq J\!\left({\hat{u},\hat{y}}\right).
\end{equation}
D'autre part, passant~\eqref{eq:control95} à la limite lorsque $\veps\to
0$, il vient $J\!\left({\hat{u},\hat{y}}\right)\leq J(u,y)$; soit,
avec~\eqref{eq:control105}, que $J(u,y) =
J\!\left({\hat{u},\hat{y}}\right)$.

On conclut bien alors, par unicité du couple contrôle-état optimal $(u,y)$,
que $\left({\hat{u},\hat{y}}\right) = (u,y)$, ce qui finit de prouver le
résultat suivant.

\begin{proposition}%
    Pour tout $\veps > 0$, le contrôle optimal approché $\bareps{u}$ et
    l'état $\bareps{y}$ associé vérifient
    \begin{equation*}
        \begin{cases}\displaystyle
            \begin{array}{rcllll}
                \bareps{u} & \longrightarrow & u & \text{dans} & \Ldeux{Q}
                & \text{faible},\\
                % \\
                \bareps{y} & \longrightarrow & y & \text{dans} & \Ldeux{Q}
                & \text{faible},
            \end{array}
        \end{cases}
    \end{equation*}
    où $(u,y)$ est le couple contrôle-état optimal
    pour~\eqref{eq:control80}\eqref{eq:control81}\eqref{eq:control82}.
\end{proposition}

On établit ci-après, qu'on a en fait plus: la convergence forte.

\begin{theoreme}%
    Pour tout $\veps > 0$, le contrôle optimal approché $\bareps{u}$ et
    l'état $\bareps{y}$ associé sont tels que, lorsque $\veps \to 0$,
    \begin{equation*}
        \begin{cases}\displaystyle
            \begin{array}{rcllll}
                \bareps{u} & \longrightarrow & u & \text{dans} &
                \Ldeux{Q} & \text{fort},\\
                % \\
                \bareps{y} & \longrightarrow & y & \text{dans} & \Ldeux{Q}
                & \text{fort},
            \end{array}
        \end{cases}
    \end{equation*}
    où $(u,y)$ est le couple contrôle-état optimal
    pour~\eqref{eq:control80}\eqref{eq:control81}\eqref{eq:control82}.
\end{theoreme}

\begin{proof}%
    Des résultats précédents, on a
    \begin{equation}\label{eq:control112}
        \begin{array}{rcllll}
            \bareps{u} & \longrightarrow & u & \text{dans} & \Ldeux{Q} &
            \text{faible},
        \end{array}
    \end{equation}

    \begin{equation}\label{eq:control113}
        \begin{array}{rcllll}
            \bareps{y} & \longrightarrow & y & \text{dans} & \Ldeux{Q} &
            \text{faible},
        \end{array}
    \end{equation}
    et
    \begin{equation*}
        J(u,y) = \lim_{\veps\to 0}J_{\veps}\!\left({\bareps{u}}\right).
    \end{equation*}
    Où,
    d'après~\eqref{eq:control106}\eqref{eq:control106p}
    et~\eqref{eq:control107}, cette dernière égalité s'écrit encore
    \begin{equation}\label{eq:control115}
        \normecq{y - z_{d}} + N\normecq{u} = \lim_{\veps\to 0}\left({%
            \normecq{\bareps{y} - z_{d}} + N\normecq{\bareps{u}}
        }\right).
    \end{equation}
    Mais alors, la norme $\normeq{\cdot}$ étant continue, \textit{a
    fortiori} semi-continue inférieurement faible, il vient
    avec~\eqref{eq:control112} et~\eqref{eq:control113} que
    \begin{equation*}
        \begin{cases}\displaystyle
            \begin{array}{rcll}
                \normecq{y - z_{d}} & \leq & \underset{\veps\to
                0}{\lim\inf}\, \normecq{\bareps{y} - z_{d}},\\
                \\
                \normecq{u} & \leq & \underset{\veps\to 0}{\lim\inf}\,
                \normecq{\bareps{u}}.
            \end{array}
        \end{cases}
    \end{equation*}
    D'où il suit, avec~\eqref{eq:control115}, que
    \begin{equation}\label{eq:control118}
        \normecq{y -z_{d}} = \lim_{\veps\to 0}\normecq{\bareps{y} - z_{d}},
    \end{equation}
    et
    \begin{equation}\label{eq:control119}
        \normecq{u} = \lim_{\veps\to 0}\normecq{\bareps{u}}.
    \end{equation}
    Ainsi, comme
    \begin{equation*}
        \normecq{\bareps{y} - y} = \normecq{\bareps{y} - z_{d}} + \normeq{y
        - z_{d}} -2\scalaireq{%
            \bareps{y} - z_{d}
        }{%
            y - z_{d}
        },
    \end{equation*}
    on conclut avec~\eqref{eq:control113} et~\eqref{eq:control118} que
    \begin{equation*}\label{eq:control120}
        \lim_{\veps\to 0}\normecq{\bareps{y} - y} = 0\qquad\ie\qquad
        \begin{array}{rcllll}
            \bareps{y} & \longrightarrow & y & \text{dans} & \Ldeux{Q} &
            \text{fort}.
        \end{array}
    \end{equation*}
    De manière analogue,~\eqref{eq:control112} et~\eqref{eq:control119}
    amènent que
    \begin{equation*}
        \begin{array}{rcllll}
            \bareps{u} & \longrightarrow & u & \text{dans} & \Ldeux{Q} &
            \text{fort},
        \end{array}
    \end{equation*}
    ce qui finit de prouver le résultat annoncé.
\end{proof}

\subsection{Système d'optimalité approché}\label{sec:approachedso}

Soit $\veps > 0$. Rappelons qu'on a pour le contrôle $\bareps{u}\in\uad$,
solution optimale de~\eqref{eq:control87}, qu'il existe
\begin{equation*}
    {\bar{y}_{0}}_{\veps}\in\Ldeux{\Omega}\qquad\text{et}\qquad
    \bareps{y}\in\Ldeux{Q},
\end{equation*}
vérifiant
\begin{equation*}
    \begin{cases}\displaystyle
        \begin{array}{rclll}
            \primetemps{\bareps{y}} - \Delta\bareps{y} & = & \bareps{u} &
            \text{dans} & Q,\\
            \\
            \bareps{y} & = & 0 & \text{sur} & \Sigma,\\
            \\
            \condinitiale{\bareps{y}} & = & {\bar{y}_{0}}_{\veps} &
            \text{dans} & \Omega,
        \end{array}
    \end{cases}
\end{equation*}
et l'estimation
\begin{equation*}
    \normeohm{\condfinale{\bareps{y}}} < \veps.
\end{equation*}
Soit alors $v\in\uad$ et $\lambda\in{\Rbb}^{*}$; on a:
\begin{equation*}
    \begin{split}
        J_{\veps}\!\left({%
            \bareps{u} + \lambda\left({v - \bareps{u}}\right)
        }\right) &= \dfrac{1}{2}\normecq{%
            y\!\left({%
                \bareps{u} + \lambda\left({v - \bareps{u}}\right),
                {\bar{y}_{0}}_{\veps}
            }\right) - z_{d}
        } + \dfrac{N}{2}\normecq{%
            \bareps{u} + \lambda\left({v - \bareps{u}}\right)
        }\\
        &\qquad + \dfrac{1}{2\veps}\normecohm{%
            {\bar{y}_{0}}_{\veps} - \condinitiale{y}
        }\\
        & = \dfrac{1}{2}\normecq{%
            \bareps{y} - z_{d} + \lambda{\fieps}
        } + \dfrac{N}{2}\normecq{%
            \bareps{u} + \lambda\left({v - \bareps{u}}\right)
        }\\
        &\qquad + \dfrac{1}{2\veps}\normecohm{%
            {\bar{y}_{0}}_{\veps} - \condinitiale{y}
        }\\
        & = J_{\veps}\!\left({\bareps{u}}\right) +
        \dfrac{{\lambda}^{2}}{2}\left({%
            \normecq{\fieps} + N\normecq{v - \bareps{u}}
        }\right)\\
        &\qquad + \lambda\scalaireq{%
            \bareps{y} - z_{d}
        }{%
            \fieps{}
        } + \lambda N\scalaireq{%
            \bareps{u}
        }{%
            v - \bareps{u}
        },
    \end{split}
\end{equation*}
ce qui amène que
\begin{equation*}
    {\left.{%
        \dfrac{\diff}{\diff{\lambda}}J_{\veps}\!\left({%
            \bareps{u} + \lambda\left({v - \bareps{u}}\right)
        }\right)
    }\right|}_{\lambda = 0} = \scalaireq{%
        \bareps{y} - z_{d}
    }{%
        \fieps{}
    } + N\scalaireq{\bareps{u}}{v - \bareps{u}},
\end{equation*}
où $\fieps = y\!\left({v - \bareps{u}, {\bar{y}_{0}}_{\veps}}\right) -
y\!\left({0, {\bar{y}_{0}}_{\veps}}\right)$ est donné par
\begin{equation}\label{eq:control125}
    \begin{cases}\displaystyle
        \begin{array}{rclll}
            \primetemps{\fieps} - \Delta\fieps & = & v - \bareps{u} &
            \text{dans} & Q,\\
            \\
            \fieps & = & 0 & \text{sur} & \Sigma,\\
            \\
            \condinitiale{\fieps} & = & 0 & \text{dans} & \Omega.
        \end{array}
    \end{cases}
\end{equation}
Ainsi donc on obtient par la condition d'optimalité du premier ordre
d'Euler-Lagrange que le contrôle optimal approché $\bareps{u}$ est l'unique
élément de $\uad$ satisfaisant
\begin{equation}\label{eq:control126}
    \scalaireq{%
        \bareps{y} - z_{d}
    }{%
        \fieps{}
    } + N\scalaireq{%
        \bareps{u}
    }{%
        v - \bareps{u}
    } \geq 0,\qquad\forall\,v\in\uad.
\end{equation}
Introduisons alors l'état adjoint $p_{\veps}\in\Ldeux{Q}$ par
\begin{equation*}
    \begin{cases}\displaystyle
        \begin{array}{rclll}
            -\primetemps{p_{\veps}} - \Delta p_{\veps} & = & \bareps{y} -
            z_{d} & \text{dans} & Q,\\
            \\
            p_{\veps} & = & 0 & \text{sur} & \Sigma,\\
            \\
            \condfinale{p_{\veps}} & = & 0 & \text{dans} & \Omega.
        \end{array}
    \end{cases}
\end{equation*}
Il vient avec~\eqref{eq:control125} que
\begin{equation*}
    \begin{array}{rclll}
        -\primetemps{p_{\veps}} - \Delta p_{\veps} & = & \bareps{y} - z_{d}
        & \text{dans} & Q
    \end{array}
\end{equation*}
amène
\begin{equation*}
    \begin{split}
         \scalaireq{%
            \bareps{y} - z_{d}
        }{%
            \fieps{}
        } & = -\scalaireq{%
            \primetemps{p_{\veps}}
        }{%
            \fieps{}
        } - \scalaireq{%
            \Delta p_{\veps}
        }{%
            \fieps{}
        }\\
        & = \scalaireohm{%
            \condinitiale{p_{\veps}}
        }{%
            \condinitiale{\fieps}
        } - \scalaireohm{%
            \condfinale{p_{\veps}}
        }{%
            \condfinale{\fieps}
        } + \scalaireq{%
            p_{\veps}
        }{%
            \primetemps{\fieps}
        }\\
        &\qquad - \scalaireq{%
            p_{\veps}
        }{%
            \Delta\fieps{}
        } - \scalaires{%
            \deriveenormale{p_{\veps}}
        }{%
            \fieps{}
        } + \scalaires{%
            p_{\veps}
        }{%
            \deriveenormale{\fieps}
        }\\
        & = \scalaireq{%
            p_{\veps}
        }{%
            \primetemps{\fieps} - \Delta\fieps{}
        } = \scalaireq{%
            p_{\veps}
        }{%
            v - \bareps{u}
        },
    \end{split}
\end{equation*}
de sorte que la condition d'optimalité~\eqref{eq:control126} se réduit à
\begin{equation*}
    \begin{array}{rcl}
        \scalaireq{%
            p_{\veps} + N\bareps{u}
        }{%
            v - \bareps{u}
        } & \geq & 0,\qquad \forall\,v\in\uad.
    \end{array}
\end{equation*}

On obtient ainsi le résultat suivant.

\begin{theoreme}\label{thm:soapproche}%
    Soit $\veps > 0$. Le contrôle $\bareps{u}$ est solution unique
    de~\eqref{eq:control87} si et seulement si le quadruplet
    \begin{equation*}
        \left\{{%
            {\bar{y}_{0}}_{\veps}, \bareps{u}, \bareps{y}, p_{\veps}
        }\right\}\in \Ldeux{\Omega}\times {\left({\Ldeux{Q}}\right)}^{3}
    \end{equation*}
    est solution du système d'optimalité approché défini par les systèmes
    d'équations aux dérivées partielles
    \begin{equation}
        \begin{cases}\displaystyle
            \begin{array}{rclll}
                \primetemps{\bareps{y}} - \Delta\bareps{y} & = & \bareps{u}
                & \text{dans} & Q,\\
                \\
                \bareps{y} & = & 0 & \text{sur} & \Sigma,\\
                \\
                \condinitiale{\bareps{y}} & = & {\bar{y}_{0}}_{\veps} &
                \text{dans} & \Omega,
            \end{array}
        \end{cases}
    \end{equation}
    et
    \begin{equation}\label{eq:control130}
        \begin{cases}\displaystyle
            \begin{array}{rclll}
                -\primetemps{p_{\veps}} - \Delta p_{\veps} & = & \bareps{y}
                - z_{d} & \text{dans} & Q,\\
                \\
                p_{\veps} & = & 0 & \text{sur} & \Sigma,\\
                \\
                \condfinale{p_{\veps}} & = & 0 & \text{dans} & \Omega,
            \end{array}
        \end{cases}
    \end{equation}
    l'estimation
    \begin{equation}
        \begin{array}{rcl}
            \normeohm{%
                \condfinale{\bareps{y}}
            } & < & \veps,
        \end{array}
    \end{equation}
    et l'inégalité variationnelle
    \begin{equation}
        \begin{array}{rcl}
            \scalaireq{%
                p_{\veps} + N\bareps{u}
            }{%
                v - \bareps{u}
            } & \geq & 0,\qquad \forall\,v\in\uad.
        \end{array}
    \end{equation}
\end{theoreme}

\subsection{Système d'optimalité singulier}\label{sec:singularso}

D'après les résultats de la Section~\ref{sec:convergence}, on a:
\begin{equation*}
    \begin{array}{rcllll}
        \bareps{u} & \longrightarrow & u & \text{dans} & \Ldeux{Q} &
        \text{fort}
    \end{array}
\end{equation*}
\begin{equation*}
    \begin{array}{rcllll}
        \bareps{y} & \longrightarrow & y & \text{dans} & \Ldeux{Q} &
        \text{fort},
    \end{array}
\end{equation*}
où $(u,y)$ est le couple contrôle-état optimal
pour~\eqref{eq:control80}\eqref{eq:control81}\eqref{eq:control82}.

Alors,~\eqref{eq:control130} étant bien posé au sens de Hadamard, il suit
qu'il existe $p\in\Ldeux{Q}$ tel que
\begin{equation*}
    \begin{array}{rcllll}
        p_{\veps} & \longrightarrow & p & \text{dans} & \Ldeux{Q} &
        \text{fort}.
    \end{array}
\end{equation*}
On passe alors aisément les résultats du~Théorème~\ref{thm:soapproche}
précédent à la limite lorsque $\veps\to 0$, pour obtenir que le système
d'optimalité singulier fort caractérisant le couple contrôle-état optimal
pour~\eqref{eq:control80}\eqref{eq:control81}\eqref{eq:control82} est tel
que ci-dessous précisé.

\begin{theoreme}\label{thm:sosingulier}%
    Le couple $(u,y)$ est solution unique du
    problème~\eqref{eq:control80}\eqref{eq:control81}\eqref{eq:control82}
    si et seulement si le triplet
    \begin{equation*}
        \left\{{%
            u,y,p
        }\right\}\in {\left({\Ldeux{Q}}\right)}^{3}
    \end{equation*}
    est solution du système d'optimalité singulier défini par les systèmes
    d'équations aux dérivées partielles
    \begin{equation}
        \begin{cases}\displaystyle
            \begin{array}{rclll}
                \primetemps{y} - \Delta y & = & u & \text{dans} & Q,\\
                \\
                y & = & 0 & \text{sur} & \Sigma,\\
                \\
                \condfinale{y} & = & 0 & \text{dans} & \Omega,
            \end{array}
        \end{cases}
    \end{equation}
    et
    \begin{equation}
        \begin{cases}\displaystyle
            \begin{array}{rclll}
                -\primetemps{p} - \Delta p & = & y - z_{d} & \text{dans} &
                Q,\\
                \\
                p & = & 0 & \text{sur} & \Sigma,\\
                \\
                \condfinale{p} & = & 0 & \text{dans} & \Omega,
            \end{array}
        \end{cases}
    \end{equation}
    et l'inégalité variationnelle
    \begin{equation}
        \begin{array}{rcll}
            \scalaireq{%
                p + Nu
            }{%
                v - u
            } & \geq & 0,\qquad\forall\,v\in\uad.
        \end{array}
    \end{equation}
\end{theoreme}


\section{Conclusion}

In this work, we succeed in characterizing the optimal control-state pair
of the control problem for the ill-posed backwards heat equation, using the
controllability concept. The method consists in interpreting the initial
problem as an inverse problem, we are also saying a controllability
problem. An approach that allows us to obtain a strong and decoupled
singular optimality system. As expected, the approach here proposed does
well without using an additional assumption of the class of the regularity
assumption~\eqref{eq:dorvilleassumption} used in~\cite{dorville} and the
Slater-type one~\eqref{eq:slater-type-assumption}. All that is required is
the following
\begin{equation}
    (u,y):\quad \condinitiale{y}\in\Ldeux{\Omega}.
\end{equation}
% assumption that the optimal state $y$ of the system verified
% $\condinitiale{y}\in\Ldeux{\Omega}$.

% Beyond, the method here adopted makes
% it possible to propose a characterization of the existence of a regular
% solution to the ill-posed backwards heat equation.
Finally, in view of the similar results obtained for the ill-posed Cauchy
system for an elliptic operator (cf.~\cite{ownElliptic} and~\cite{ownAAA}),
the controllability method here proposed seems relevant for control
problems (of singular distributed systems) which require recourse to
Slater-type assumptions such as~\eqref{eq:slater-type-assumption}.

\nocite{*}
\printbibliography{}

\end{document}
