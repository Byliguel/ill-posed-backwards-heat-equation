\section{Introduction}

Let us consider an open and bounded subset $\Omega$ of ${\Rbb}^{n}$, $n\in
{\mathbb{N}}^{*}$, of boundary ${\Gamma}$, twice continuously
differentiable, with $\Omega$ locally on one side only of $\Gamma$; that is
to say that $\bar{\Omega}$ is a variety with a boundary of class
${\Cscr}^{2}$.

For $T > 0$, we denote $\displaystyle Q = \Omega\times(0,T)$, $\displaystyle
\Sigma = \Gamma\times (0,T)$, and $\displaystyle {\Uscr}_{ad}$ a closed and
non-empty convex of $\Ldeux{Q}$.

Given $v\in\Ldeux{Q}$, we consider the following problem

\begin{equation}\label{eq:initial-problem}
    \begin{cases}\displaystyle
        \begin{array}{rclll}
            \primetemps{z} - \Delta z & = & v & \text{in} & Q,\\
            \\
            z & = & 0 & \text{on} & \Sigma,\\
            \\
            \condfinale{z} & = & 0 & \text{in} & \Omega,
        \end{array}
    \end{cases}
\end{equation}
being interested in the evolution, in $\Omega$ and at a time $t\in (0,T)$,
of the temperature $z$, the value of which is kwnown at the final time $T$.

It is well known that problem~\eqref{eq:initial-problem}, so called the
ill-posed backwards heat equation, does not admits solution for any initial
data $v\in\Ldeux{Q}$.

So we consider, a priori, pairs $(v,z)\in {\left({\Ldeux{Q}}\right)}^{2}$
satisfying~\eqref{eq:initial-problem}, saying that such pairs constitute
the set of control-state pairs.

A control-state pair $(v,z)$ for~\eqref{eq:initial-problem} will be said
admissible if
\begin{equation*}\label{eq:admissible-control-state-pair}
    v\in {\Uscr}_{ad};
\end{equation*}
and we will denote $(v,z)\in \Ascr$, designating by $\Ascr$ the set of
admissible control-state pairs.

Supposing that the set $\Ascr$ is non-empty, we introduce the cost function
\begin{equation}\label{eq:initial-cost-function}
    J(v,z) = \dfrac{1}{2}\normecq{z - z_{d}} + \dfrac{N}{2}\normecq{v},
\end{equation}
where $z_{d}\in\Ldeux{Q}$ and $N > 0$, and we are interested in the control
problem which consists in finding the control-state pair
\begin{equation}\label{eq:initial-control-problem}
    (u,y)\in\Ascr:\quad J(u,y) = \inf_{(v,z)\in \Ascr}\,J(v,z).
\end{equation}

Then, the following result is immediate
\begin{theoreme}%
    The optimal control problem~\eqref{eq:initial-control-problem} admits a
    unique solution $(u,y)$, called the optimal control-state pair.
\end{theoreme}

\begin{proof}%
    Due to its structure, the cost functional $J$ is clearly coercive,
    strictly convex and lower semi-continuous. From where, with the
    non-vacuity assumption of the closed convex set of admissible
    control-state pairs $\Ascr$, we can conclude to existence and unicity
    of the optimal control-state pair $(u,y)$.
\end{proof}

Hence, one can establish, (cf.~\cite{lions2}), using the first-order
Euler-Lagrange optimality condition, that the optimal control-state pair
$(u,y)$ is characterized by
\begin{equation*}\label{eq:initial-variational-inequation}
    \begin{array}{rcl}
        \scalaireq{%
            y - z_{d}
        }{%
            z - y
        } + N\scalaireq{u}{v - u} & \geq & 0,\qquad \forall\,(v,z)\in\Ascr.
    \end{array}
\end{equation*}
We now focus on the characterization of the optimal pair $(u,y)$ through a
strong and decoupled singular optimality system.

To do so, J.~L.~Lions proposed in~\cite{lions2}, the penalization method.
This approach allows well to achieve this, but requires resorting to the
Slater-type assumption that
\begin{equation}\label{eq:slater-type-assumption}
    \text{“the set of admissible controls }{\Uscr}_{ad}\ \text{is of non
    empty interior in}\ \Ldeux{Q}\text{”}.
\end{equation}

But in many situations, the Slater-type assumption does not hold; for
example in the case
\begin{equation*}
    \uad = {\left({\Ldeux{Q}}\right)}^{+} = \left\{{%
        v\in\Ldeux{Q}:\ v\geq 0
    }\right\}.
\end{equation*}
It therefore appears relevant to know how to do without this assumption, an
unrealistic assumption in the opinion of some authors.

For this purpose,  R.~Dorville, O.~Nakoulima and A.~Omrane propose
in~\cite{dorville}, the notion of least regrets control, applied to a
regularized elliptic state equation with missing data. An approach that
allows them to characterize the least regrets control through a strong and
decoupled singular optimality system. This, certainly without recourse to
the Slater-type assumption~\eqref{eq:slater-type-assumption}, but to the
detriment of the final condition. So that, the no-regret control thus
obtained does not satisfy the final condition
\begin{equation*}
    \begin{array}{rclll}
        \condfinale{y} & = & 0 & \text{in} & \Omega.
    \end{array}
\end{equation*}
So therefore, to the best of our knowledge, the problem remains, in all
generality, globally open.

Hence the proposal we make here to use the controllability method for the
analysis of the problem. This method has made it possible to propose
in~\cite{ownElliptic},~\cite{ownAAA},~\cite{ownParabolic}
and~\cite{ownhyperbolic}, an answer to the same question of the control,
without recourse to~\eqref{eq:slater-type-assumption}, of ill-posed Cauchy
system for elliptic, parabolic and hyperbolic operators. This point of view
consists in interpreting the initial problem as an inverse problem (a
controllability problem), which we approach, by density argument, by a
sequence of well-posed problems. This induces a regularization method
which, it seems to us, provides an answer to the question of controlling,
without recourse to the Slater-type
assumption~\eqref{eq:slater-type-assumption}, the ill-posed backwards heat
equation.

The rest of this paper is as follows. Section~\ref{sec:controllability}
is devoted to the announced interpretation of the initial problem as an
inverse problem. We define there the so-called exact controllability
problem which we approach, by density argument, by an equivalent problem
(this one called the approached controllability problem). In
Section~\ref{sec:controlproblem}, we return to the control
problem~\eqref{eq:initial-control-problem}, starting by regularized it
based on the controllability results previously obtained. After established
the convergence of the process in Section~\ref{sec:convergence}, then the
approached optimality system in Section~\ref{sec:approachedso}, we end in
Section~\ref{sec:singularso} with the optimality system for the initial
control problem.
