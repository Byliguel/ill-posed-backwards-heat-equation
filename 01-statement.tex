\section{Introduction}

Let us consider an open and bounded subset $\Omega$ of ${\Rbb}^{n}$, $n\in
{\mathbb{N}}^{*}$, of boundary ${\Gamma}$, twice continuously
differentiable, with $\Omega$ locally on one side only of $\Gamma$; that is
to say that $\bar{\Omega}$ is a variety with a boundary of class
${\Cscr}^{2}$.

For $T > 0$, we denote $\displaystyle Q = \Omega\times(0,T)$, $\displaystyle
\Sigma = \Gamma\times (0,T)$, and $\displaystyle {\Uscr}_{ad}$ a closed and
non-empty convex of $\Ldeux{Q}$.

Given $v\in\Ldeux{Q}$, we consider the following problem

\begin{equation}\label{eq:initial-problem}
    \begin{cases}\displaystyle
        \begin{array}{rclll}
            \primetemps{z} - \Delta z & = & v & \text{in} & Q,\\
            \\
            z & = & 0 & \text{on} & \Sigma,\\
            \\
            \condfinale{z} & = & 0 & \text{in} & \Omega,
        \end{array}
    \end{cases}
\end{equation}
being interested in the evolution, in $\Omega$ and at a time $t\in (0,T)$,
of the temperature $z$, the value of which is kwnown at the final time $T$.

It is well known that problem~\eqref{eq:initial-problem}, so called the
ill-posed backwards heat equation, does not admits solution for any initial
data $v\in\Ldeux{Q}$.

So we consider, a priori, pairs $(v,z)\in {\left({\Ldeux{Q}}\right)}^{2}$
satisfying~\eqref{eq:initial-problem}, saying that such pairs constitute
the set of control-state pairs.

A control-state pair $(v,z)$ for~\eqref{eq:initial-problem} will be said
admissible if
\begin{equation*}\label{eq:admissible-control-state-pair}
    v\in {\Uscr}_{ad};
\end{equation*}
and we will denote $(v,z)\in \Ascr$, designating by $\Ascr$ the set of
admissible control-state pairs.

Supposing that the set $\Ascr$ is non-empty, we introduce the cost function
\begin{equation}\label{eq:initial-cost-function}
    J(v,z) = \dfrac{1}{2}\normecq{z - z_{d}} + \dfrac{N}{2}\normecq{v},
\end{equation}
where $z_{d}\in\Ldeux{Q}$ and $N > 0$.

Then, we are interested in the control problem which consists in finding
the control-state pair
\begin{equation}\label{eq:initial-control-problem}
    (u,y)\in\Ascr:\quad J(u,y) = \inf_{(v,z)\in \Ascr}\,J(v,z).
\end{equation}

Then, the following result is immediate
\begin{theoreme}%
    The optimal control problem~\eqref{eq:initial-control-problem} admits a
    unique solution $(u,y)$, called the optimal control-state pair.
\end{theoreme}

\begin{proof}%
    Due to its structure, the cost functional $J$ is clearly coercive,
    strictly convex and lower semi-continuous. From where, with the
    non-vacuity assumption of the closed convex set of admissible
    control-state pairs $\Ascr$, we can conclude to existence and unicity
    of the optimal control-state pair $(u,y)$.
\end{proof}

Hence, one can establish, (cf.~\cite{lions2}), using the first-order
Euler-Lagrange optimality condition, that the optimal control-state pair
$(u,y)$ is characterized by
\begin{equation*}\label{eq:initial-variational-inequation}
    \begin{array}{rcl}
        \scalaireq{%
            y - z_{d}
        }{%
            z - y
        } + N\scalaireq{u}{v - u} & \geq & 0,\qquad \forall\,(v,z)\in\Ascr.
    \end{array}
\end{equation*}
We now focus on the characterization of the optimal pair $(u,y)$ through a
strong and decoupled singular optimality system.

A classic method to do so is the penalization method introduced by
J.~L.~Lions in~\cite{lions2}. This one consists of approaching the optimal
control-state pair $(u,y)$ by a penalized problem. More precisely, for
$\veps > 0$, we define the penalized cost function
\begin{equation*}\label{eq:penalized-cost-function}
    J_{\veps}\!\left({v,z}\right) = J(v,z) + \dfrac{1}{\veps}\normecq{%
        \primetemps{z} - \Delta z - v
    }.
\end{equation*}
We then establish that the optimal control-state pair
$\left({u_{\veps},z_{\veps}}\right)$ corresponding to this last cost
function converges towards the optimal control-state pair $(u,y)$. We
therefore have a theoretical approximation process for the optimal pair
$(u,y)$. Thanks to the first-order Euler-Lagrange optimality condition, we
characterize the approached optimal control-state pair
$\left({u_{\veps},y_{\veps}}\right)$ by a system of variational
inequalities which we interpret as an approached optimality system after
introducing the approached adjoint state
\begin{equation*}
    p_{\veps} = -\dfrac{1}{\veps}\left({%
        \primetemps{y_{\veps}} - \Delta y_{\veps} - u_{\veps}
    }\right).
\end{equation*}
The main step of this technique is the use of the \textit{a priori}
estimation method to obtain a strongly decoupled optimality system by
passing to the limit in the approached optimality system. But this final
step requires the Slater-type assumption that
\begin{equation}\label{eq:slater-type-assumption}
    \text{“the set of admissible controls }{\Uscr}_{ad}\ \text{is of non
    empty interior in}\ \Ldeux{Q}\text{”}.
\end{equation}
But in many situations, the Slater-type assumption does not hold; for
example in the case
\begin{equation*}
    \uad = {\left({\Ldeux{Q}}\right)}^{+} = \left\{{%
        v\in\Ldeux{Q}:\ v\geq 0
    }\right\}.
\end{equation*}
It therefore appears relevant to know how to do without this unrealistic
assumption.

For this purpose,  R.~Dorville, O.~Nakoulima and A.~Omrane propose
in~\cite{dorville}, the notion of least regrets control, applied to a
regularized elliptic state equation with missing data. An approach that
allows them to characterize the least regrets control $u^{\gamma}$ through
a strong and decoupled singular optimality system. This, certainly without
recourse to the Slater-type assumption~\eqref{eq:slater-type-assumption},
but to the detriment of the final condition
\begin{equation*}
    \begin{array}{rclll}
        \condfinale{y^{\gamma}} & = & 0 & \text{in} & \Omega.
    \end{array}
\end{equation*}
That is to say, more precisely, that the state $y^{\gamma}$ associated with
the least regrets control $u^{\gamma}$ is such that, in all generality,
\begin{equation*}
    \begin{array}{rclll}
        \condfinale{y^{\gamma}} & \neq & 0 & \text{in} & \Omega.
    \end{array}
\end{equation*}
In response to this problem, the authors introduce the notion of zero-order
corrector, which calls for the regularity assumption
\begin{equation}\label{eq:dorvilleassumption}
    \condfinale{y^{\gamma}}\in H_{0}^{1}(\Omega):\
    \primetemps{y^{\gamma}}\in \Ldeux{Q}.
\end{equation}
This last one allows them to obtain the characterization, specified below,
of the initial optimal control-state pair.

\begin{theoreme}[cf.~\cite{dorville}]%
    The no-regret control $u$ of the
    problem~\eqref{eq:initial-problem}\eqref{eq:initial-cost-function}\eqref{eq:initial-control-problem}
    is characterized by the unique $\left\{{u,y,\rho,p,\xi}\right\}$
    solution of the system:
    \begin{equation}
        \begin{cases}\displaystyle
            \begin{array}{rclrclrclrclll}
                Ly & = & u, & L\rho & = & 0, & L^{*}p & = & y - z_{d} +
                \rho, & L^{*}\xi & = & y & \text{in} & Q,\\
                y & = & 0, & \rho & = & 0, & p & = & 0, & \xi & = & 0, &
                \text{on} & \Sigma,\\
                y(0) & = & 0, & \rho(0) & = & \lambda(0) & {} & {} & {} &
                {} & {} & {} & \text{in} & \Omega,\\
                {} & {} & {} & {} & {} & {} & p(T) & = & 0 & \xi(T) & = & 0
                & \text{in} & \Omega,
            \end{array}
        \end{cases}
    \end{equation}
    and we have the variational inequality
    \begin{equation}
        \begin{array}{rcll}
            \scalaire{p + Nu}{v - u}{} & \geq & 0 & \forall\,v\in\uad,
        \end{array}
    \end{equation}
    with
    \begin{equation*}
        \left|{%
            \begin{array}{l}
                L = \primetemps{} - \Delta\quad\text{and}\quad L^{*} =
                -\primetemps{} - \Delta\quad\text{its adjoint operator};\\
                u, y, p, \rho, \xi\in \Ldeux{]0,T[;\Ldeux{\Omega}},\quad
                \lambda(0)\in\Ldeux{\Omega}.
            \end{array}
        }\right.
    \end{equation*}
\end{theoreme}

Ultimately, the no-regret control thus obtained does not satisfy the final
condition
\begin{equation*}
    \begin{array}{rclll}
        \condfinale{y} & = & 0 & \text{in} & \Omega,
    \end{array}
\end{equation*}
so that, in all generality, the problem remains, to the best of our
knowledge, globally open.

We propose in this paper the controllability method for the analysis,
without recourse to the Slater-type
assumption~\eqref{eq:slater-type-assumption}, of the control problem of the
ill-posed backwards heat equation. Note that this method has made it
possible to propose
in~\cite{ownElliptic},~\cite{ownAAA},~\cite{ownParabolic}
and~\cite{ownhyperbolic}, an answer to the same question of the control,
without recourse to~\eqref{eq:slater-type-assumption}, of ill-posed Cauchy
system for elliptic, parabolic and hyperbolic operators.

The rest of this paper is as follows. Section~\ref{sec:controllability}
is devoted to the announced interpretation of the initial problem as an
inverse problem. We define there the so-called exact controllability
problem which we approach, by density argument, by an equivalent problem
(this one called the approached controllability problem). In
Section~\ref{sec:controlproblem}, we return to the control
problem~\eqref{eq:initial-control-problem}, starting by regularized it
based on the controllability results previously obtained. After established
the convergence of the process in Section~\ref{sec:convergence}, then the
approached optimality system in Section~\ref{sec:approachedso}, we end in
Section~\ref{sec:singularso} with the optimality system for the initial
control problem.
