\section{Position du problème}

On considère $\Omega$ un ouvert borné de ${\Rbb}^{n}$ de frontière
${\Gamma}$ deux fois continûment différentiable, $\Omega$ étant localement
d'un seul côté de $\Gamma$, soit que $\bar{\Omega}$ est un ouvert est une
variété à bord de classe ${\Cscr}^{2}$.

Pour $T > 0$, on note $\displaystyle Q = \Omega\times]0,T[$, $\displaystyle
\Sigma = \Gamma\times ]0,T[$, et $\displaystyle {\Uscr}_{ad}$ un
sous-ensemble convexe fermé non vide de $\Ldeux{Q}$.

L'étude de l'évolution de la chaleur dans le domaine $\Omega$ à un instant
$t\in ]0,T[$, connaissant la valeur de la température à l'instant final est
connu pour être le prototype des problèmes mal posés. Elle consiste à
trouver l'état $z\in \Ldeux{Q}$ tel que
\begin{equation}\label{eq:initial-problem}
    \begin{cases}\displaystyle
        \begin{array}{rl}
            \primetemps{z} - \Delta z = v & \text{dans}\ Q,\\
            \\
            z = 0 & \text{sur}\ \Sigma,\\
            \\
            \condfinale{z} = 0 & \text{dans}\ \Omega,
        \end{array}
    \end{cases}
\end{equation}
où $v$, donné dans $\Ldeux{Q}$, est la variable de contrôle.

On dit qu'un couple contrôle-état $(v,z)\in\Ldeux{Q}\times \Ldeux{Q}$
est admissible s'il vérifie~\eqref{eq:initial-problem} avec $v\in
{\Uscr}_{ad}$.

Supposant que l'ensemble des couples contrôle-état admissible, noté
${\chi}_{ad}$, est non vide, on introduit la fonctionnelle coût
\begin{equation}\label{eq:initial-cost-function}
    J(v,z) = \dfrac{1}{2}\normecq{z - z_{d}} + \dfrac{N}{2}\normecq{v},
\end{equation}
avec $z_{d}\in\Ldeux{Q}$ et $N > 0$.

On s'intéresse alors au problème de contrôle consistant à trouver un couple
contrôle-état $\displaystyle (u,y)\in {\chi}_{ad}$ solution de
\begin{equation}\label{eq:initial-control-problem}
    J(u,y) = \inf_{(v,z)\in {\chi}_{ad}}\,J(v,z).
\end{equation}

On a immédiatement le résultat suivant.
\begin{theoreme}%
    Le problème de contrôle optimal~\eqref{eq:initial-control-problem}
    admet une solution unique $(u,y)$ appelée couple optimal.
\end{theoreme}

\begin{proof}%
    L'ensemble des couples admissibles étant non vide et $J$ étant une
    fonction semi-continue inférieurement, strictement convexe et coercive,
    il existe un unique couple admissible $(u,y)$ solution
    de~\eqref{eq:initial-control-problem}.
\end{proof}

\begin{remarque}%
    En utilisant la condition d'optimalité d'Euler-Lagrange:
    \begin{equation*}
        \forall\,(v,z)\in{\chi}_{ad},\quad {\left.{%
            \dfrac{\diff}{\diff{\lambda}}J\!\left({%
                u + \lambda(v - u), y + \lambda (z - y)
            }\right)
        }\right|}_{\lambda = 0} \geq 0,
    \end{equation*}
    le couple optimal $(u,y)$ est caractérisé par
    \begin{equation*}
        \forall\,(v,z)\in {\chi}_{ad},\quad \scalaireq{y - z_{d}}{z - y} +
        N\scalaireq{u}{v - u} \geq 0.
    \end{equation*}
\end{remarque}

L'objectif est alors ici de trouver un système d'optimalité découplé
caractérisant $(u,y)$. Une méthode classique en la matière est la méthode
pénalisation de J.-L.~Lions, consistant à approcher $(u,y)$ par un problème
pénalisé.
