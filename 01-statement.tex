\section{Introduction}

On considère $\Omega$ un ouvert borné de ${\Rbb}^{n}$, de frontière
${\Gamma}$ deux fois continûment différentiable avec $\Omega$ localement
d'un seul côté de $\Gamma$; soit que $\bar{\Omega}$ est une variété à bord
de classe ${\Cscr}^{2}$.

Pour $T > 0$, on note $\displaystyle Q = \Omega\times(0,T)$, $\displaystyle
\Sigma = \Gamma\times (0,T)$, et $\displaystyle {\Uscr}_{ad}$ un
sous-ensemble convexe fermé non vide de $\Ldeux{Q}$.

Étant donné $v\in\Ldeux{Q}$, on considère le problème

\begin{equation}\label{eq:initial-problem}
    \begin{cases}\displaystyle
        \begin{array}{rclll}
            \primetemps{z} - \Delta z & = & v & \text{dans} & Q,\\
            \\
            z & = & 0 & \text{sur} & \Sigma,\\
            \\
            \condfinale{z} & = & 0 & \text{dans} & \Omega,
        \end{array}
    \end{cases}
\end{equation}
étudiant l'évolution, dans le domaine $\Omega$ et à l'instant $t\in (0,T)$,
de la chaleur (la température) $z$ connaissant la valeur de celle-ci à
l'instant final $T$.

Il est bien connu que le problème~\eqref{eq:initial-problem}, dit équation
de la chaleur rétrograde, n'admet pas de solution pour des données
initiales quelconques.

On considère donc a priori les couples $(v,z)\in
{\left({\Ldeux{Q}}\right)}^{2}$ satisfaisant~\eqref{eq:initial-problem},
disant de ceux-ci qu'ils constituent l'ensemble des couples contrôle-état.

Un couple contrôle-état $(v,z)$ de~\eqref{eq:initial-problem} sera dit
admissible si on a en plus
\begin{equation*}\label{eq:admissible-control-state-pair}
    v\in {\Uscr}_{ad};
\end{equation*}
on notera alors $(v,z)\in \Ascr$, désignant par $\Ascr$ l'ensemble des
couples contrôle-état admissibles.

Supposant que l'ensemble $\Ascr$ des couples contrôle-état admissibles est
non vide, on introduit la fonctionnelle
\begin{equation*}\label{eq:initial-cost-function}
    J(v,z) = \dfrac{1}{2}\normecq{z - z_{d}} + \dfrac{N}{2}\normecq{v},
\end{equation*}
avec $z_{d}\in\Ldeux{Q}$ et $N > 0$.

On s'intéresse alors au problème de contrôle consistant à trouver un couple
contrôle-état
\begin{equation}\label{eq:initial-control-problem}
    (u,y)\in {\Ascr}_{ad}:\quad J(u,y) = \inf_{(v,z)\in
    {\Ascr}_{ad}}\,J(v,z).
\end{equation}

On a immédiatement le résultat suivant.
\begin{theoreme}%
    Le problème de contrôle optimal~\eqref{eq:initial-control-problem}
    admet une solution unique $(u,y)$ appelée couple contrôle-état optimal.
\end{theoreme}

\begin{proof}%
    De par sa structure, que la fonctionnelle $J$ est coercive, strictement
    convexe et semi-continue inférieurement. Ce qui, avec l'hypothèse de
    non vacuité du convexe fermé $\Ascr$ des couples contrôle-état
    admissibles, permet bien de conclure à l'existence et l'unicité du
    couple contrôle-état optimal $(u,y)$.
\end{proof}

Dès lors on établit (cf.~\cite{lions2}), avec la condition d'optimalité du
premier ordre d'Euler-Lagrange, que le couple optimal $(u,y)$ est
caractérisé par
\begin{equation*}\label{eq:initial-variational-inequation}
    \begin{array}{rcl}
        \scalaireq{%
            y - z_{d}
        }{%
            z - y
        } + N\scalaireq{u}{v - u} & \geq & 0,\qquad \forall\,(v,z)\in\Ascr.
    \end{array}
\end{equation*}
Le problème consiste alors à trouver un système d'optimalité découplé
caractérisant $(u,y)$.

Une méthode classique en la matière est la méthode de pénalisation
introduite par J.-L.~Lions dans~\cite{lions2}. Celle-ci consiste à
approcher $(u,y)$ par un problème pénalisé. Plus précisément, pour $\veps >
0$, on définit la fonction coût pénalisée
\begin{equation*}\label{eq:penalized-cost-function}
    J_{\veps}\!\left({v,z}\right) = J(v,z) + \dfrac{1}{\veps}\normecq{%
        \primetemps{z} - \Delta z - v
    }.
\end{equation*}
On établit alors que le couple optimal $\left({u_{\veps},z_{\veps}}\right)$
correspondant à cette nouvelle fonction coût converge vers le couple
optimal $(u,y)$. On a donc un procédé d'approximation théorique du couple
optimal $(u,y)$. À l'aide des conditions du premier ordre d'Euler-Lagrange,
on caractérise le couple optimal approché
$\left({u_{\veps},y_{\veps}}\right)$ par un système d'inégalités
variationnelles qu'on interprète en un système d'optimalité approché après
introduction de l'état adjoint approché
\begin{equation*}
    p_{\veps} = -\dfrac{1}{\veps}\left({%
        \primetemps{y_{\veps}} - \Delta y_{\veps} - u_{\veps}
    }\right).
\end{equation*}
Le point essentiel de cette technique consiste en l'utilisation de la
méthode des estimations \textit{a priori} pour l'obtention d'un système
d'optimalité découplé fort par passage à la limite dans le système
d'optimalité approché. Mais cette ultime étape requiert l'hypothèse de type
Slater que
\begin{equation}\label{eq:slater-type-assumption}
    \text{“}{\Uscr}_{ad}\ \text{soit d'intérieur non vide dans}\
    \Ldeux{Q}\text{”}.
\end{equation}
Mais dans bien de situations, la condition de Slater n'est pas vérifiée; on
peut citer en exemple le cas
\begin{equation*}
    \uad = {\left({\Ldeux{Q}}\right)}^{+} = \left\{{%
        v\in\Ldeux{Q}:\ v\geq 0
    }\right\}.
\end{equation*}
Il apparaît donc pertinent de savoir faire sans cette hypothèse très peu
réaliste.

Pour ce faire, R.~Dorville, O.~Nakoulima et A.~Omrane proposent
dans~\cite{dorville}, la notion de contrôle à moindres regrets, appliquée à
une régularisée elliptique à données manquantes du
problème~\eqref{eq:initial-problem}. Une approche qui leur permet de
caractériser le contrôle à moindres regrets $u^{\gamma}$ via un système
d'optimalité singulier découplé fort. Cela, certes sans recours à
l'hypothèse~\eqref{eq:slater-type-assumption}, mais au détriment de la
condition finale
\begin{equation*}
    \begin{array}{rclll}
        \condfinale{y^{\gamma}} & = & 0 & \text{dans} & \Omega,
    \end{array}
\end{equation*}
soit plus précisément que l'état associé au contrôle à moindres regrets
$u^{\gamma}$ est alors tel qu'en toute généralité
\begin{equation*}
    \begin{array}{rclll}
        \condfinale{y^{\gamma}} & \neq & 0 & \text{dans} & \Omega.
    \end{array}
\end{equation*}
Pour contourner le problème,
les auteurs introduisent la notion de correcteur d'ordre $0$, laquelle
appelle à l'hypothèse de régularité
\begin{equation}\label{eq:dorvilleassumption}
    \condfinale{y^{\gamma}}\in H_{0}^{1}(\Omega):\
    \primetemps{y^{\gamma}}\in \Ldeux{Q}.
\end{equation}
Laquelle hypothèse permet d'obtenir la caractérisation ci-dessous précisé
du couple contrôle-état optimal initial.

\begin{theoreme}[cf.~\cite{dorville}]%
    Le contrôle sans regret $u$ du
    problème~\eqref{eq:initial-problem}\eqref{eq:initial-cost-function}\eqref{eq:initial-control-problem}
    est caractérisé par l'unique $\left\{{u,y,\rho,p,\xi}\right\}$ solution
    du système:
    \begin{equation}
        \begin{cases}\displaystyle
            \begin{array}{rclrclrclrclll}
                Ly & = & u, & L\rho & = & 0, & L^{*}p & = & y - z_{d} +
                \rho, & L^{*}\xi & = & y & \text{dans} & Q,\\
                y & = & 0, & \rho & = & 0, & p & = & 0, & \xi & = & 0, &
                \text{sur} & \Sigma,\\
                y(0) & = & 0, & \rho(0) & = & \lambda(0) & {} & {} & {} &
                {} & {} & {} & \text{dans} & \Omega,\\
                {} & {} & {} & {} & {} & {} & p(T) & = & 0 & \xi(T) & = & 0
                & \text{dans} & \Omega,
            \end{array}
        \end{cases}
    \end{equation}
    et on a l'inégalité variationnelle
    \begin{equation}
        \begin{array}{rcll}
            \scalaire{p + Nu}{v - u}{} & \geq & 0 & \forall\,v\in\uad,
        \end{array}
    \end{equation}
    avec
    \begin{equation*}
        \left|{%
            \begin{array}{l}
                L = \primetemps{} - \Delta\quad\text{et}\quad L^{*} =
                -\primetemps{} - \Delta\quad\text{son opérateur adjoint};\\
                u, y, p, \rho, \xi\in \Ldeux{]0,T[;\Ldeux{\Omega}},\quad
                \lambda(0)\in\Ldeux{\Omega}.
            \end{array}
        }\right.
    \end{equation*}
\end{theoreme}

En définitive, le contrôle sans regret ainsi obtenu ne satisfait pas la
condition finale
\begin{equation*}
    \begin{array}{rclll}
        \condfinale{y} & = & 0 & \text{dans} & \Omega,
    \end{array}
\end{equation*}
de sorte qu'en toute généralité le problème reste, du mieux de nos
connaissances, globalement ouvert.

Nous proposons à travers le présent article la méthode de contrôlabilité
pour l'analyse, sans recours à
l'hypothèse~\eqref{eq:slater-type-assumption}, du problème de contrôle de
l'équation de la chaleur rétrograde. Notons que cette méthode a permis de
proposer dans~\cite{ownElliptic}, puis~\cite{ownAAA}, une réponse à la même
question du contrôle, sans recours à~\eqref{eq:slater-type-assumption}, de
problèmes de contrôle du système de Cauchy mal posé pour opérateur
elliptique.

La suite de cet article se présente comme suit. La
Section~\ref{sec:controllability} est consacrée à l'interprétation annoncée
du problème initial comme un problème inverse. On y définit le problème dit
de contrôlabilité exacte que nous approchons, par argument de densité, par
un problème équivalent (celui-ci dit de contrôlabilité approchée). Dans la
Section~\ref{sec:controlproblem}, nous revenons au problème de
contrôle~\eqref{eq:initial-control-problem}, commençant par le régulariser
d'après les résultats de contrôlabilité précédemment obtenus. Après avoir
établit la convergence du procédé dans la Section~\ref{sec:convergence},
puis le système d'optimalité approché dans la
Section~\ref{sec:approachedso}, nous finissons dans la
Section~\ref{sec:singularso} par le système d'optimalité singulier.
