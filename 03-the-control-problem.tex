\section{La méthode de contrôlabilité}\label{sec:controlproblem}

Commençons par rappeler que nous sommes ici intéressés par le contrôle de
l'équation de la chaleur rétrograde. Soit, plus précisément, que pour
$v,z\in \Ldeux{Q}$ vérifiant
\begin{equation}\label{eq:control80}
    \begin{cases}\displaystyle
        \begin{array}{rclll}
            \primetemps{z} - \Delta z & = & v & \text{dans} & Q,\\
            \\
            z & = & 0 & \text{sur} & \Sigma,\\
            \\
            \condfinale{z} & = & 0 & \text{sur} & \Omega,
        \end{array}
    \end{cases}
\end{equation}
on pose
\begin{equation}\label{eq:control81}
    J(v,z) = \dfrac{1}{2}\normecq{z - z_{d}} + \dfrac{N}{2}\normecq{v},
\end{equation}
s'intéressant au problème de contrôle
\begin{equation}\label{eq:control82}
    \inf\!\left\{{%
        J(v,z);\ (v,z)\in\Ascr
    }\right\}.
\end{equation}

Comme souligné à l'introduction, le
problème~\eqref{eq:control80}\eqref{eq:control81}\eqref{eq:control82} admet
une solution unique $(u,y)$ dont la caractérisation, via un système
d'optimalité singulier découplé fort, et cela sans recours à l'hypothèse de
Slater~\eqref{eq:slater-type-assumption}, est le principal objet ici.

% \subsection{La méthode de contrôlabilité}

Pour ce faire, partant de l'hypothèse de non-vacuité de l'ensemble des
couples contrôle-état admissibles
pour~\eqref{eq:control80}\eqref{eq:control81}\eqref{eq:control82} et des
résultats de la section précédente, on a pour tous $v\in\Ldeux{Q}$ et
$\veps > 0$ qu'il existe
\begin{equation*}
    {y_{0}}_{\veps}\in\Ldeux{\Omega}\qquad\text{et}\qquad y_{\veps}^{v}\in
    \Vbb
\end{equation*}
% comme souligné à la Remarque~\eqref{rq:naturalassumption}, de
% l'hypothèse
% \begin{equation*}
%     \left\{{%
%         (v,z)\in \Ascr: \condinitiale{z} \in\Ldeux{\Omega}
%     }\right\}\neq \emptyset,
% \end{equation*}
% on a pour tous $v\in\Ldeux{Q}$ et $\veps > 0$ qu'il existe
% \begin{equation*}
%     {y_{0}}_{\veps}\in\Ldeux{\Omega}\qquad\text{et}\qquad y_{\veps}^{v}\in
%     \Vbb
% \end{equation*}
tels que
\begin{equation*}
    \begin{cases}\displaystyle
        \begin{array}{rclll}
            \primetemps{y_{\veps}^{v}} - \Delta y_{\veps}^{v} & = & v &
            \text{dans} & Q,\\
            \\
            y_{\veps}^{v} & = & 0 & \text{sur} & \Sigma,\\
            \\
            \condinitiale{y_{\veps}^{v}} & = & {y_{0}}_{\veps} &
            \text{dans} & \Omega,
        \end{array}
    \end{cases}
\end{equation*}
avec l'estimation
\begin{equation*}
    \normeohm{%
        \condfinale{y_{\veps}^{v}}
    } < \veps.
\end{equation*}
Supposant alors que la solution optimale $(u,y)$
de~\eqref{eq:control80}\eqref{eq:control81}\eqref{eq:control82} est telle
que $\displaystyle \condinitiale{y}\in\Ldeux{\Omega}$, on introduit la
fonctionnelle
\begin{equation*}
    J_{\veps}(v) = \dfrac{1}{2}\normecq{y_{\veps}^{v} - z_{d}} +
    \dfrac{N}{2}\normecq{v} + \dfrac{1}{2\veps}\normecohm{{y_{0}}_{\veps} -
    \condinitiale{y}},
\end{equation*}
et on s'intéresse au problème
\begin{equation}\label{eq:control87}
    \inf\!\left\{{%
        J_{\veps}(v);\ v\in\uad
    }\right\}.
\end{equation}
Le résultat suivant est immédiat.

\begin{proposition}%
    Pour tout $\veps > 0$, le problème de contrôle~\eqref{eq:control87}
    admet une solution unique: le contrôle optimal approché $\bareps{u}$.
\end{proposition}

\subsection{Convergence de la méthode}\label{sec:convergence}

Soit $\veps > 0$. On a existence et unicité du contrôle approché
$\bareps{u}$. Soit, avec les résultats de la section précédente, qu'il
existe
\begin{equation*}
    {\bar{y}_{0}}_{\veps}\in\Ldeux{\Omega}\qquad\text{et}\qquad
    \bareps{y}\in\Vbb
\end{equation*}
vérifiant
\begin{equation}\label{eq:control89}
    \begin{cases}\displaystyle
        \begin{array}{rclll}
            \primetemps{\bareps{y}} - \Delta\bareps{y} & = & \bareps{u} &
            \text{dans} & Q,\\
            \\
            \bareps{y} & = & 0 & \text{sur} & \Sigma,\\
            \\
            \condinitiale{\bareps{y}} & = & {\bar{y}_{0}}_{\veps} &
            \text{dans} & \Omega,
        \end{array}
    \end{cases}
\end{equation}
et l'estimation
\begin{equation*}
    \normeohm{\condfinale{\bareps{y}} } < \veps,
\end{equation*}
avec
\begin{equation*}
    J_{\veps}\!\left({\bareps{u}}\right)\leq J_{\veps}(v),\quad
    \forall\,v\in\uad.
\end{equation*}
En particulier donc
\begin{equation}\label{eq:control92}
    J_{\veps}\!\left({\bareps{u}}\right) \leq J_{\veps}(u),
\end{equation}
où $u$ est le contrôle optimal
pour~\eqref{eq:control80}\eqref{eq:control81}\eqref{eq:control82}.

On a en fait que $J_{\veps}(u)$ est indépendant de $\veps$. En effet, comme
l'état optimal $y$ associé à $u$ vérifie
$\condinitiale{y}\in\Ldeux{\Omega}$, on obtient en posant
${y_{0}^{*}}_{\veps} = \condinitiale{y}$ que $y$ satisfait
\begin{equation*}
    \begin{cases}\displaystyle
        \begin{array}{rclll}
            \primetemps{y} - \Delta y & = & u & \text{dans} & Q,\\
            \\
            y & = & 0 & \text{sur} & \Sigma,\\
            \\
            \condinitiale{y} & = & {y_{0}^{*}}_{\veps} & \text{dans} &
            \Omega
        \end{array}
    \end{cases}
\end{equation*}
avec
\begin{equation*}
    \begin{array}{rclll}
        \condfinale{y} & = & 0 & \text{dans} & \Omega,
    \end{array}
    \qquad\text{\textit{a fortiori}}\qquad \normeohm{\condfinale{y}} <
    \veps.
\end{equation*}
Il suit que $J_{\veps}(u)$ est défini, avec
\begin{equation*}
    \begin{split}
        J_{\veps}(u) &= \dfrac{1}{2}\normecq{y - z_{d}} +
        \dfrac{N}{2}\normecq{u} + \dfrac{1}{2\veps}\normecohm{%
            {y_{0}^{*}}_{\veps} - \condinitiale{y}
        }\\
        & = \dfrac{1}{2}\normecq{y - z_{d}} + \dfrac{N}{2}\normecq{u} =
        J(u,y).
    \end{split}
\end{equation*}
Ainsi~\eqref{eq:control92} devient
\begin{equation}\label{eq:control95}
    J_{\veps}\!\left({\bareps{u}}\right) \leq J_{\veps}(u) = J(u,y),
\end{equation}
et il vient qu'il existe des constantes $C_{i}\in{\Rbb}^{*}$ indépendantes
de $\veps$ telles qu'on a
\begin{equation*}
    \normeq{\bareps{y}} \leq C_{1},\qquad
    \normeq{\bareps{u}}\leq C_{2}\qquad\text{et}\qquad
    \normeohm{{\bar{y}_{0}}_{\veps}}\leq C_{3}.
\end{equation*}
On en déduit immédiatement qu'il existe $\hat{u}\in\Ldeux{Q}$,
$\hat{y}\in\Ldeux{Q}$ et ${\hat{y}}_{0}\in\Ldeux{\Omega}$ tels que
\begin{equation*}
    \begin{cases}\displaystyle
        \begin{array}{rcllll}
            \bareps{u} & \longrightarrow & \hat{u} & \text{dans} &
            \Ldeux{Q} & \text{faible},\\
            \bareps{y} & \longrightarrow & \hat{y} & \text{dans} &
            \Ldeux{Q} & \text{faible},\\
            {\bar{y}_{0}}_{\veps} & \longrightarrow & {\hat{y}}_{0} &
            \text{dans} & \Ldeux{\Omega} & \text{faible}.
        \end{array}
    \end{cases}
\end{equation*}
Mais encore,~\eqref{eq:control95} entraîne aussi
\begin{equation}\label{eq:control106}
    \normeohm{{\bar{y}_{0}}_{\veps} - \condinitiale{y}} \leq 2\veps\,C_{4},
\end{equation}
ce qui conduit, comme
\begin{equation}\label{eq:control106p}
    \begin{array}{rcllll}
        {\bar{y}_{0}}_{\veps} & \longrightarrow & {\hat{y}}_{0} &
        \text{dans} & \Ldeux{\Omega} & \text{faible},
    \end{array}
\end{equation}
à
\begin{equation}\label{eq:control107}
    \begin{array}{rclll}
        {\hat{y}}_{0} & = & \condinitiale{y} & \text{dans} & \Omega.
    \end{array}
\end{equation}
Alors, il vient avec~\eqref{eq:control89}, que pour tout
$\vphi\in\Dscr(Q)$,
\begin{equation*}
    \begin{split}
        \scalaireq{\primetemps{\bareps{y}}}{\vphi} &- \scalaireq{%
            \Delta\bareps{y}
        }{%
            \vphi%
        } = \scalaireq{\bareps{u}}{\vphi}\\
        &\iff \scalaireohm{%
            \condfinale{\bareps{y}}
        }{%
            \condfinale{\vphi}
        } - \scalaireohm{%
            \condinitiale{\bareps{y}}
        }{%
            \condinitiale{\vphi}
        } - \scalaireq{%
            \bareps{y}
        }{%
            \primetemps{\vphi}
        }\\
        &\qquad - \scalaireq{%
            \bareps{y}
        }{%
            \Delta\vphi%
        } - \scalaires{%
            \deriveenormale{\bareps{y}}
        }{%
            \vphi%
        } + \scalaires{%
            \bareps{y}
        }{%
            \deriveenormale{\vphi}
        } = \scalaireq{\bareps{u}}{\vphi}\\
        &\iff \scalaireohm{%
            \condfinale{\bareps{y}}
        }{%
            \condfinale{\vphi}
        } - \scalaireohm{%
            {\bar{y}_{0}}_{\veps}
        }{%
            \condinitiale{\vphi}
        } - \scalaireq{%
            \bareps{y}
        }{%
            \primetemps{\vphi}
        }\\
        &\qquad - \scalaireq{%
            \bareps{y}
        }{%
            \Delta\vphi%
        } - \scalaires{%
            \deriveenormale{\bareps{y}}
        }{%
            \vphi%
        } = \scalaireq{\bareps{u}}{\vphi},
    \end{split}
\end{equation*}
soit à la limite
\begin{equation*}
    \begin{split}
        \scalaireohm{%
            \condfinale{\hat{y}}
        }{%
            \condfinale{\vphi}
        } & - \scalaireohm{%
            {\hat{y}}_{0}
        }{%
            \condinitiale{\vphi}
        } - \scalaireq{%
            \hat{y}
        }{%
            \primetemps{\vphi}
        } - \scalaireq{%
            \hat{y}
        }{%
            \Delta\vphi%
        }\\
        &\qquad - \scalaires{%
            \deriveenormale{\hat{y}}
        }{%
            \vphi%
        } = \scalaireq{\hat{u}}{\vphi}\\
        &\iff \scalaireohm{%
            \condinitiale{\hat{y}} - {\hat{y}}_{0}
        }{%
            \condinitiale{\vphi}
        } + \scalaireq{%
            \primetemps{\hat{y}} - \Delta\hat{y}
        }{%
            \vphi%
        }\\
        &\qquad - \scalaires{%
            \hat{y}
        }{%
            \deriveenormale{\vphi}
        } = \scalaireq{\hat{u}}{\vphi},
    \end{split}
\end{equation*}
\ie{}
\begin{equation}\label{eq:control102}
    \begin{cases}\displaystyle
        \begin{array}{rclll}
            \primetemps{\hat{y}} - \Delta\hat{y} & = & \hat{u} &
            \text{dans} & Q,\\
            \\
            \hat{y} & = & 0 & \text{sur} & \Sigma,\\
            \\
            \condinitiale{\hat{y}} & = & {\hat{y}}_{0} & \text{dans} &
            \Omega.
        \end{array}
    \end{cases}
\end{equation}
Par ailleurs, la norme $\normeohm{\cdot}$ étant continue, \textit{a
fortiori} faiblement continue,
\begin{equation*}
    \begin{array}{rcllll}
        \bareps{y} & \longrightarrow & \hat{y} & \text{dans} &
        \Ldeux{Q} &  \text{faible}
    \end{array}\qquad\text{et}\qquad
    \normeohm{\condfinale{\bareps{y}}} < \veps
\end{equation*}
amènent
\begin{equation}\label{eq:control103}
    \begin{array}{rclll}
        \condfinale{\hat{y}} & = & 0 & \text{dans} & \Omega.
    \end{array}
\end{equation}
Ainsi~\eqref{eq:control102}~et~\eqref{eq:control103} permettent de conclure
que $\hat{y}\in\Ldeux{Q}$ satisfait
\begin{equation*}
    \begin{cases}\displaystyle
        \begin{array}{rclll}
            \primetemps{\hat{y}} - \Delta\hat{y} & = & \hat{u} &
            \text{dans} & Q,\\
            \\
            \hat{y} & = & 0 & \text{sur} & \Sigma,\\
            \\
            \condfinale{\hat{y}} & = & 0 & \text{dans} & \Omega,
        \end{array}
    \end{cases}
\end{equation*}
notant que $\hat{u}\in\uad$, puisque $\bareps{u}\in\uad$ et $\uad$ est
fermé donc fermé faible: le couple $\left({\hat{u},\hat{y}}\right)$ est
admissible
pour~\eqref{eq:control80}\eqref{eq:control81}\eqref{eq:control82}, et donc
\begin{equation}\label{eq:control105}
    J(u,y)\leq J\!\left({\hat{u},\hat{y}}\right).
\end{equation}
D'autre part, passant~\eqref{eq:control95} à la limite lorsque $\veps\to
0$, il vient $J\!\left({\hat{u},\hat{y}}\right)\leq J(u,y)$; soit,
avec~\eqref{eq:control105}, que $J(u,y) =
J\!\left({\hat{u},\hat{y}}\right)$.

On conclut bien alors, par unicité du couple contrôle-état optimal $(u,y)$,
que $\left({\hat{u},\hat{y}}\right) = (u,y)$, ce qui finit de prouver le
résultat suivant.

\begin{proposition}%
    Pour tout $\veps > 0$, le contrôle optimal approché $\bareps{u}$ et
    l'état $\bareps{y}$ associé vérifient
    \begin{equation*}
        \begin{cases}\displaystyle
            \begin{array}{rcllll}
                \bareps{u} & \longrightarrow & u & \text{dans} & \Ldeux{Q}
                & \text{faible},\\
                % \\
                \bareps{y} & \longrightarrow & y & \text{dans} & \Ldeux{Q}
                & \text{faible},
            \end{array}
        \end{cases}
    \end{equation*}
    où $(u,y)$ est le couple contrôle-état optimal
    pour~\eqref{eq:control80}\eqref{eq:control81}\eqref{eq:control82}.
\end{proposition}

On établit ci-après, qu'on a en fait plus: la convergence forte.

\begin{theoreme}%
    Pour tout $\veps > 0$, le contrôle optimal approché $\bareps{u}$ et
    l'état $\bareps{y}$ associé sont tels que, lorsque $\veps \to 0$,
    \begin{equation*}
        \begin{cases}\displaystyle
            \begin{array}{rcllll}
                \bareps{u} & \longrightarrow & u & \text{dans} &
                \Ldeux{Q} & \text{fort},\\
                % \\
                \bareps{y} & \longrightarrow & y & \text{dans} & \Ldeux{Q}
                & \text{fort},
            \end{array}
        \end{cases}
    \end{equation*}
    où $(u,y)$ est le couple contrôle-état optimal
    pour~\eqref{eq:control80}\eqref{eq:control81}\eqref{eq:control82}.
\end{theoreme}

\begin{proof}%
    Des résultats précédents, on a
    \begin{equation}\label{eq:control112}
        \begin{array}{rcllll}
            \bareps{u} & \longrightarrow & u & \text{dans} & \Ldeux{Q} &
            \text{faible},
        \end{array}
    \end{equation}

    \begin{equation}\label{eq:control113}
        \begin{array}{rcllll}
            \bareps{y} & \longrightarrow & y & \text{dans} & \Ldeux{Q} &
            \text{faible},
        \end{array}
    \end{equation}
    et
    \begin{equation*}
        J(u,y) = \lim_{\veps\to 0}J_{\veps}\!\left({\bareps{u}}\right).
    \end{equation*}
    Où,
    d'après~\eqref{eq:control106}\eqref{eq:control106p}
    et~\eqref{eq:control107}, cette dernière égalité s'écrit encore
    \begin{equation}\label{eq:control115}
        \normecq{y - z_{d}} + N\normecq{u} = \lim_{\veps\to 0}\left({%
            \normecq{\bareps{y} - z_{d}} + N\normecq{\bareps{u}}
        }\right).
    \end{equation}
    Mais alors, la norme $\normeq{\cdot}$ étant continue, \textit{a
    fortiori} semi-continue inférieurement faible, il vient
    avec~\eqref{eq:control112} et~\eqref{eq:control113} que
    \begin{equation*}
        \begin{cases}\displaystyle
            \begin{array}{rcll}
                \normecq{y - z_{d}} & \leq & \underset{\veps\to
                0}{\lim\inf}\, \normecq{\bareps{y} - z_{d}},\\
                \\
                \normecq{u} & \leq & \underset{\veps\to 0}{\lim\inf}\,
                \normecq{\bareps{u}}.
            \end{array}
        \end{cases}
    \end{equation*}
    D'où il suit, avec~\eqref{eq:control115}, que
    \begin{equation}\label{eq:control118}
        \normecq{y -z_{d}} = \lim_{\veps\to 0}\normecq{\bareps{y} - z_{d}},
    \end{equation}
    et
    \begin{equation}\label{eq:control119}
        \normecq{u} = \lim_{\veps\to 0}\normecq{\bareps{u}}.
    \end{equation}
    Ainsi, comme
    \begin{equation*}
        \normecq{\bareps{y} - y} = \normecq{\bareps{y} - z_{d}} + \normeq{y
        - z_{d}} -2\scalaireq{%
            \bareps{y} - z_{d}
        }{%
            y - z_{d}
        },
    \end{equation*}
    on conclut avec~\eqref{eq:control113} et~\eqref{eq:control118} que
    \begin{equation*}\label{eq:control120}
        \lim_{\veps\to 0}\normecq{\bareps{y} - y} = 0\qquad\ie\qquad
        \begin{array}{rcllll}
            \bareps{y} & \longrightarrow & y & \text{dans} & \Ldeux{Q} &
            \text{fort}.
        \end{array}
    \end{equation*}
    De manière analogue,~\eqref{eq:control112} et~\eqref{eq:control119}
    amènent que
    \begin{equation*}
        \begin{array}{rcllll}
            \bareps{u} & \longrightarrow & u & \text{dans} & \Ldeux{Q} &
            \text{fort},
        \end{array}
    \end{equation*}
    ce qui finit de prouver le résultat annoncé.
\end{proof}

\subsection{Système d'optimalité approché}\label{sec:approachedso}

Soit $\veps > 0$. Rappelons qu'on a pour le contrôle $\bareps{u}\in\uad$,
solution optimale de~\eqref{eq:control87}, qu'il existe
\begin{equation*}
    {\bar{y}_{0}}_{\veps}\in\Ldeux{\Omega}\qquad\text{et}\qquad
    \bareps{y}\in\Ldeux{Q},
\end{equation*}
vérifiant
\begin{equation*}
    \begin{cases}\displaystyle
        \begin{array}{rclll}
            \primetemps{\bareps{y}} - \Delta\bareps{y} & = & \bareps{u} &
            \text{dans} & Q,\\
            \\
            \bareps{y} & = & 0 & \text{sur} & \Sigma,\\
            \\
            \condinitiale{\bareps{y}} & = & {\bar{y}_{0}}_{\veps} &
            \text{dans} & \Omega,
        \end{array}
    \end{cases}
\end{equation*}
et l'estimation
\begin{equation*}
    \normeohm{\condfinale{\bareps{y}}} < \veps.
\end{equation*}
Soit alors $v\in\uad$ et $\lambda\in{\Rbb}^{*}$; on a:
\begin{equation*}
    \begin{split}
        J_{\veps}\!\left({%
            \bareps{u} + \lambda\left({v - \bareps{u}}\right)
        }\right) &= \dfrac{1}{2}\normecq{%
            y\!\left({%
                \bareps{u} + \lambda\left({v - \bareps{u}}\right),
                {\bar{y}_{0}}_{\veps}
            }\right) - z_{d}
        } + \dfrac{N}{2}\normecq{%
            \bareps{u} + \lambda\left({v - \bareps{u}}\right)
        }\\
        &\qquad + \dfrac{1}{2\veps}\normecohm{%
            {\bar{y}_{0}}_{\veps} - \condinitiale{y}
        }\\
        & = \dfrac{1}{2}\normecq{%
            \bareps{y} - z_{d} + \lambda{\fieps}
        } + \dfrac{N}{2}\normecq{%
            \bareps{u} + \lambda\left({v - \bareps{u}}\right)
        }\\
        &\qquad + \dfrac{1}{2\veps}\normecohm{%
            {\bar{y}_{0}}_{\veps} - \condinitiale{y}
        }\\
        & = J_{\veps}\!\left({\bareps{u}}\right) +
        \dfrac{{\lambda}^{2}}{2}\left({%
            \normecq{\fieps} + N\normecq{v - \bareps{u}}
        }\right)\\
        &\qquad + \lambda\scalaireq{%
            \bareps{y} - z_{d}
        }{%
            \fieps{}
        } + \lambda N\scalaireq{%
            \bareps{u}
        }{%
            v - \bareps{u}
        },
    \end{split}
\end{equation*}
ce qui amène que
\begin{equation*}
    {\left.{%
        \dfrac{\diff}{\diff{\lambda}}J_{\veps}\!\left({%
            \bareps{u} + \lambda\left({v - \bareps{u}}\right)
        }\right)
    }\right|}_{\lambda = 0} = \scalaireq{%
        \bareps{y} - z_{d}
    }{%
        \fieps{}
    } + N\scalaireq{\bareps{u}}{v - \bareps{u}},
\end{equation*}
où $\fieps = y\!\left({v - \bareps{u}, {\bar{y}_{0}}_{\veps}}\right) -
y\!\left({0, {\bar{y}_{0}}_{\veps}}\right)$ est donné par
\begin{equation}\label{eq:control125}
    \begin{cases}\displaystyle
        \begin{array}{rclll}
            \primetemps{\fieps} - \Delta\fieps & = & v - \bareps{u} &
            \text{dans} & Q,\\
            \\
            \fieps & = & 0 & \text{sur} & \Sigma,\\
            \\
            \condinitiale{\fieps} & = & 0 & \text{dans} & \Omega.
        \end{array}
    \end{cases}
\end{equation}
Ainsi donc on obtient par la condition d'optimalité du premier ordre
d'Euler-Lagrange que le contrôle optimal approché $\bareps{u}$ est l'unique
élément de $\uad$ satisfaisant
\begin{equation}\label{eq:control126}
    \scalaireq{%
        \bareps{y} - z_{d}
    }{%
        \fieps{}
    } + N\scalaireq{%
        \bareps{u}
    }{%
        v - \bareps{u}
    } \geq 0,\qquad\forall\,v\in\uad.
\end{equation}
Introduisons alors l'état adjoint $p_{\veps}\in\Ldeux{Q}$ par
\begin{equation*}
    \begin{cases}\displaystyle
        \begin{array}{rclll}
            -\primetemps{p_{\veps}} - \Delta p_{\veps} & = & \bareps{y} -
            z_{d} & \text{dans} & Q,\\
            \\
            p_{\veps} & = & 0 & \text{sur} & \Sigma,\\
            \\
            \condfinale{p_{\veps}} & = & 0 & \text{dans} & \Omega.
        \end{array}
    \end{cases}
\end{equation*}
Il vient avec~\eqref{eq:control125} que
\begin{equation*}
    \begin{array}{rclll}
        -\primetemps{p_{\veps}} - \Delta p_{\veps} & = & \bareps{y} - z_{d}
        & \text{dans} & Q
    \end{array}
\end{equation*}
amène
\begin{equation*}
    \begin{split}
         \scalaireq{%
            \bareps{y} - z_{d}
        }{%
            \fieps{}
        } & = -\scalaireq{%
            \primetemps{p_{\veps}}
        }{%
            \fieps{}
        } - \scalaireq{%
            \Delta p_{\veps}
        }{%
            \fieps{}
        }\\
        & = \scalaireohm{%
            \condinitiale{p_{\veps}}
        }{%
            \condinitiale{\fieps}
        } - \scalaireohm{%
            \condfinale{p_{\veps}}
        }{%
            \condfinale{\fieps}
        } + \scalaireq{%
            p_{\veps}
        }{%
            \primetemps{\fieps}
        }\\
        &\qquad - \scalaireq{%
            p_{\veps}
        }{%
            \Delta\fieps{}
        } - \scalaires{%
            \deriveenormale{p_{\veps}}
        }{%
            \fieps{}
        } + \scalaires{%
            p_{\veps}
        }{%
            \deriveenormale{\fieps}
        }\\
        & = \scalaireq{%
            p_{\veps}
        }{%
            \primetemps{\fieps} - \Delta\fieps{}
        } = \scalaireq{%
            p_{\veps}
        }{%
            v - \bareps{u}
        },
    \end{split}
\end{equation*}
de sorte que la condition d'optimalité~\eqref{eq:control126} se réduit à
\begin{equation*}
    \begin{array}{rcl}
        \scalaireq{%
            p_{\veps} + N\bareps{u}
        }{%
            v - \bareps{u}
        } & \geq & 0,\qquad \forall\,v\in\uad.
    \end{array}
\end{equation*}

On obtient ainsi le résultat suivant.

\begin{theoreme}\label{thm:soapproche}%
    Soit $\veps > 0$. Le contrôle $\bareps{u}$ est solution unique
    de~\eqref{eq:control87} si et seulement si le quadruplet
    \begin{equation*}
        \left\{{%
            {\bar{y}_{0}}_{\veps}, \bareps{u}, \bareps{y}, p_{\veps}
        }\right\}\in \Ldeux{\Omega}\times {\left({\Ldeux{Q}}\right)}^{3}
    \end{equation*}
    est solution du système d'optimalité approché défini par les systèmes
    d'équations aux dérivées partielles
    \begin{equation}
        \begin{cases}\displaystyle
            \begin{array}{rclll}
                \primetemps{\bareps{y}} - \Delta\bareps{y} & = & \bareps{u}
                & \text{dans} & Q,\\
                \\
                \bareps{y} & = & 0 & \text{sur} & \Sigma,\\
                \\
                \condinitiale{\bareps{y}} & = & {\bar{y}_{0}}_{\veps} &
                \text{dans} & \Omega,
            \end{array}
        \end{cases}
    \end{equation}
    et
    \begin{equation}\label{eq:control130}
        \begin{cases}\displaystyle
            \begin{array}{rclll}
                -\primetemps{p_{\veps}} - \Delta p_{\veps} & = & \bareps{y}
                - z_{d} & \text{dans} & Q,\\
                \\
                p_{\veps} & = & 0 & \text{sur} & \Sigma,\\
                \\
                \condfinale{p_{\veps}} & = & 0 & \text{dans} & \Omega,
            \end{array}
        \end{cases}
    \end{equation}
    l'estimation
    \begin{equation}
        \begin{array}{rcl}
            \normeohm{%
                \condfinale{\bareps{y}}
            } & < & \veps,
        \end{array}
    \end{equation}
    et l'inégalité variationnelle
    \begin{equation}
        \begin{array}{rcl}
            \scalaireq{%
                p_{\veps} + N\bareps{u}
            }{%
                v - \bareps{u}
            } & \geq & 0,\qquad \forall\,v\in\uad.
        \end{array}
    \end{equation}
\end{theoreme}

\subsection{Système d'optimalité singulier}\label{sec:singularso}

D'après les résultats de la Section~\ref{sec:convergence}, on a:
\begin{equation*}
    \begin{array}{rcllll}
        \bareps{u} & \longrightarrow & u & \text{dans} & \Ldeux{Q} &
        \text{fort}
    \end{array}
\end{equation*}
\begin{equation*}
    \begin{array}{rcllll}
        \bareps{y} & \longrightarrow & y & \text{dans} & \Ldeux{Q} &
        \text{fort},
    \end{array}
\end{equation*}
où $(u,y)$ est le couple contrôle-état optimal
pour~\eqref{eq:control80}\eqref{eq:control81}\eqref{eq:control82}.

Alors,~\eqref{eq:control130} étant bien posé au sens de Hadamard, il suit
qu'il existe $p\in\Ldeux{Q}$ tel que
\begin{equation*}
    \begin{array}{rcllll}
        p_{\veps} & \longrightarrow & p & \text{dans} & \Ldeux{Q} &
        \text{fort}.
    \end{array}
\end{equation*}
On passe alors aisément les résultats du~Théorème~\ref{thm:soapproche}
précédent à la limite lorsque $\veps\to 0$, pour obtenir que le système
d'optimalité singulier fort caractérisant le couple contrôle-état optimal
pour~\eqref{eq:control80}\eqref{eq:control81}\eqref{eq:control82} est tel
que ci-dessous précisé.

\begin{theoreme}\label{thm:sosingulier}%
    Le couple $(u,y)$ est solution unique du
    problème~\eqref{eq:control80}\eqref{eq:control81}\eqref{eq:control82}
    si et seulement si le triplet
    \begin{equation*}
        \left\{{%
            u,y,p
        }\right\}\in {\left({\Ldeux{Q}}\right)}^{3}
    \end{equation*}
    est solution du système d'optimalité singulier défini par les systèmes
    d'équations aux dérivées partielles
    \begin{equation}
        \begin{cases}\displaystyle
            \begin{array}{rclll}
                \primetemps{y} - \Delta y & = & u & \text{dans} & Q,\\
                \\
                y & = & 0 & \text{sur} & \Sigma,\\
                \\
                \condfinale{y} & = & 0 & \text{dans} & \Omega,
            \end{array}
        \end{cases}
    \end{equation}
    et
    \begin{equation}
        \begin{cases}\displaystyle
            \begin{array}{rclll}
                -\primetemps{p} - \Delta p & = & y - z_{d} & \text{dans} &
                Q,\\
                \\
                p & = & 0 & \text{sur} & \Sigma,\\
                \\
                \condfinale{p} & = & 0 & \text{dans} & \Omega,
            \end{array}
        \end{cases}
    \end{equation}
    et l'inégalité variationnelle
    \begin{equation}
        \begin{array}{rcll}
            \scalaireq{%
                p + Nu
            }{%
                v - u
            } & \geq & 0,\qquad\forall\,v\in\uad.
        \end{array}
    \end{equation}
\end{theoreme}
