\section{The optimal control problem}\label{sec:controlproblem}

Let us start by recalling that we are here interested in the control
problem of the ill-posed backwards heat equation. More precisely, that is
to say that, for $v,z\in \Ldeux{Q}$ satisfying
\begin{equation}\label{eq:control80}
    \begin{cases}\displaystyle
        \begin{array}{rclll}
            \primetemps{z} - \Delta z & = & v & \text{in} & Q,\\
            \\
            z & = & 0 & \text{on} & \Sigma,\\
            \\
            \condfinale{z} & = & 0 & \text{on} & \Omega,
        \end{array}
    \end{cases}
\end{equation}
we introduce the cost function
\begin{equation}\label{eq:control81}
    J(v,z) = \dfrac{1}{2}\normecq{z - z_{d}} + \dfrac{N}{2}\normecq{v},
\end{equation}
being interested in the control problem
\begin{equation}\label{eq:control82}
    \inf\!\left\{{%
        J(v,z);\ (v,z)\in\Ascr
    }\right\}.
\end{equation}

As underlined in the introduction, the optimal control
problem~\eqref{eq:control80}\eqref{eq:control81}\eqref{eq:control82} admits
a unique solution $(u,y)$ whose characterization, via a strong and
decoupled singular optimality system, and this without using the
Slater-type assumption~\eqref{eq:slater-type-assumption}, is the main
objective.

To do so, starting from the non-vacuity assumption of the set of admissible
control-state pairs, and using the results previously obtained, we have
that, for any $v\in\Ldeux{Q}$ and $\veps > 0$, there exist
\begin{equation*}
    {y_{0}}_{\veps}\in\Ldeux{\Omega}\qquad\text{and}\qquad y_{\veps}^{v}\in
    \Vbb
\end{equation*}
such that
\begin{equation*}
    \begin{cases}\displaystyle
        \begin{array}{rclll}
            \primetemps{y_{\veps}^{v}} - \Delta y_{\veps}^{v} & = & v &
            \text{in} & Q,\\
            \\
            y_{\veps}^{v} & = & 0 & \text{on} & \Sigma,\\
            \\
            \condinitiale{y_{\veps}^{v}} & = & {y_{0}}_{\veps} &
            \text{in} & \Omega,
        \end{array}
    \end{cases}
\end{equation*}
with the estimate
\begin{equation*}
    \normeohm{%
        \condfinale{y_{\veps}^{v}}
    } < \veps.
\end{equation*}
Assuming that the optimal solution $(u,y)$
for~\eqref{eq:control80}\eqref{eq:control81}\eqref{eq:control82} satisfies
$\displaystyle \condinitiale{y}\in\Ldeux{\Omega}$, we introduce the
functional
\begin{equation*}
    J_{\veps}(v) = \dfrac{1}{2}\normecq{y_{\veps}^{v} - z_{d}} +
    \dfrac{N}{2}\normecq{v} + \dfrac{1}{2\veps}\normecohm{{y_{0}}_{\veps} -
    \condinitiale{y}},
\end{equation*}
being interested in the control problem
\begin{equation}\label{eq:control87}
    \inf\!\left\{{%
        J_{\veps}(v);\ v\in\uad
    }\right\}.
\end{equation}
The following result is then immediate.

\begin{proposition}%
    For any $\veps > 0$, the control problem~\eqref{eq:control87} admits a
    unique solution: the approached optimal control $\bareps{u}$.
\end{proposition}

\begin{remarque}%
    The assumption that the optimal control-state pair $(u,y)$
    for~\eqref{eq:control80}\eqref{eq:control81}\eqref{eq:control82}
    satisfies $\displaystyle \condinitiale{y}\in \Ldeux{\Omega}$ is
    supported by the results obtained in~\cite{dorville}.
\end{remarque}

\subsection{Convergence of the method}\label{sec:convergence}

Let $\veps > 0$. For the approached optimal control, the results previously
obtained lead to the existence of
\begin{equation*}
    {\bar{y}_{0}}_{\veps}\in\Ldeux{\Omega}\qquad\text{and}\qquad
    \bareps{y}\in\Vbb
\end{equation*}
satisfying
\begin{equation}\label{eq:control89}
    \begin{cases}\displaystyle
        \begin{array}{rclll}
            \primetemps{\bareps{y}} - \Delta\bareps{y} & = & \bareps{u} &
            \text{in} & Q,\\
            \\
            \bareps{y} & = & 0 & \text{on} & \Sigma,\\
            \\
            \condinitiale{\bareps{y}} & = & {\bar{y}_{0}}_{\veps} &
            \text{in} & \Omega,
        \end{array}
    \end{cases}
\end{equation}
and the estimate
\begin{equation}
    \normeohm{\condfinale{\bareps{y}} } < \veps,
\end{equation}
with
\begin{equation}
    J_{\veps}\!\left({\bareps{u}}\right)\leq J_{\veps}(v),\quad
    \forall\,v\in\uad.
\end{equation}
In particular
\begin{equation}\label{eq:control92}
    J_{\veps}\!\left({\bareps{u}}\right) \leq J_{\veps}(u),
\end{equation}
where $u$ is the optimal control
for~\eqref{eq:control80}\eqref{eq:control81}\eqref{eq:control82}.

We have in fact that $J_{\veps}(u)$ is independant of $\veps$. Indeed,
since the optimal state $y$ associated with the optimal control $u$
satisfies $\condinitiale{y}\in\Ldeux{\Omega}$, we can take
${y_{0}^{*}}_{\veps} = \condinitiale{y}$ to obtain that $y$ satisfies
\begin{equation*}
    \begin{cases}\displaystyle
        \begin{array}{rclll}
            \primetemps{y} - \Delta y & = & u & \text{in} & Q,\\
            \\
            y & = & 0 & \text{on} & \Sigma,\\
            \\
            \condinitiale{y} & = & {y_{0}^{*}}_{\veps} & \text{in} &
            \Omega
        \end{array}
    \end{cases}
\end{equation*}
with
\begin{equation*}
    \begin{array}{rclll}
        \condfinale{y} & = & 0 & \text{in} & \Omega,
    \end{array}
    \qquad\text{\textit{a fortiori}}\qquad \normeohm{\condfinale{y}} <
    \veps.
\end{equation*}
It follows that $J_{\veps}(u)$ is defined, with
\begin{equation*}
    \begin{split}
        J_{\veps}(u) &= \dfrac{1}{2}\normecq{y - z_{d}} +
        \dfrac{N}{2}\normecq{u} + \dfrac{1}{2\veps}\normecohm{%
            {y_{0}^{*}}_{\veps} - \condinitiale{y}
        }\\
        & = \dfrac{1}{2}\normecq{y - z_{d}} + \dfrac{N}{2}\normecq{u} =
        J(u,y).
    \end{split}
\end{equation*}
Hence~\eqref{eq:control92} becomes
\begin{equation}\label{eq:control95}
    J_{\veps}\!\left({\bareps{u}}\right) \leq J_{\veps}(u) = J(u,y),
\end{equation}
and it follows that there exist constants $C_{i}\in{\Rbb}_{+}^{*}$
independant of $\veps$ such as
\begin{equation*}
    \normeq{\bareps{y}} \leq C_{1},\qquad
    \normeq{\bareps{u}}\leq C_{2}\qquad\text{and}\qquad
    \normeohm{{\bar{y}_{0}}_{\veps}}\leq C_{3}.
\end{equation*}
Then, we immediately deduce that there exists $\hat{u}\in\Ldeux{Q}$,
$\hat{y}\in\Ldeux{Q}$ and ${\hat{y}}_{0}\in\Ldeux{\Omega}$ such as
\begin{equation*}
    \begin{cases}\displaystyle
        \begin{array}{rclll}
            \bareps{u} & \longrightarrow & \hat{u} & \text{weakly in} &
            \Ldeux{Q},\\
            \bareps{y} & \longrightarrow & \hat{y} & \text{weakly in} &
            \Ldeux{Q},\\
            {\bar{y}_{0}}_{\veps} & \longrightarrow & {\hat{y}}_{0} &
            \text{weakly in} & \Ldeux{\Omega}.
        \end{array}
    \end{cases}
\end{equation*}
But much more,~\eqref{eq:control95} also implies
\begin{equation}\label{eq:control106}
    \normeohm{{\bar{y}_{0}}_{\veps} - \condinitiale{y}} \leq 2\veps\,C_{4},
\end{equation}
which leads, since
\begin{equation}\label{eq:control106p}
    \begin{array}{rclll}
        {\bar{y}_{0}}_{\veps} & \longrightarrow & {\hat{y}}_{0} &
        \text{weakly in} & \Ldeux{\Omega},
    \end{array}
\end{equation}
to
\begin{equation}\label{eq:control107}
    \begin{array}{rclll}
        {\hat{y}}_{0} & = & \condinitiale{y} & \text{in} & \Omega.
    \end{array}
\end{equation}
Moreover and once again, since $\displaystyle y_{\veps}^{v}\in \Vbb\subset
\Ldeux{0,T\,; H_{0}^{1}(\Omega)}$ and by continuity of the zero-order trace
operator, we have that
\begin{equation*}
    \begin{array}{rclll}
        \bareps{y} & = & 0 & \text{on} & \Sigma
    \end{array}\qquad\text{and}\qquad
    \begin{array}{rclll}
        \bareps{y} & \longrightarrow & \hat{y} & \text{weakly in} &
        \Ldeux{Q}
    \end{array}
\end{equation*}
lead to
\begin{equation}\label{eq:new-convergence-34}
    \begin{array}{rclll}
        \hat{y} & = & 0 & \text{on} & \Sigma.
    \end{array}
\end{equation}
Now, let us multiply~\eqref{eq:control89} by $\vphi\in\Dscr(Q)$, and
integrate by parts over $Q$; we obtain
\begin{equation*}
    \begin{split}
        \scalaireq{\primetemps{\bareps{y}}}{\vphi} &- \scalaireq{%
            \Delta\bareps{y}
        }{%
            \vphi%
        } = \scalaireq{\bareps{u}}{\vphi}\\
        &\iff \scalaireohm{%
            \condfinale{\bareps{y}}
        }{%
            \condfinale{\vphi}
        } - \scalaireohm{%
            {\bar{y}_{0}}_{\veps}
        }{%
            \condinitiale{\vphi}
        } - \scalaireq{%
            \bareps{y}
        }{%
            \primetemps{\vphi}
        }\\
        &\qquad - \scalaireq{%
            \bareps{y}
        }{%
            \Delta\vphi%
        } = \scalaireq{\bareps{u}}{\vphi},
    \end{split}
\end{equation*}
which gives, passing to the limit,
\begin{equation*}
    \begin{split}
        \scalaireohm{%
            \condfinale{\hat{y}}
        }{%
            \condfinale{\vphi}
        } & - \scalaireohm{%
            {\hat{y}}_{0}
        }{%
            \condinitiale{\vphi}
        } - \scalaireq{%
            \hat{y}
        }{%
            \primetemps{\vphi}
        } - \scalaireq{%
            \hat{y}
        }{%
            \Delta\vphi%
        } = \scalaireq{\hat{u}}{\vphi}\\
        &\iff \scalaireohm{%
            \condinitiale{\hat{y}} - {\hat{y}}_{0}
        }{%
            \condinitiale{\vphi}
        } + \scalaireq{%
            \primetemps{\hat{y}} - \Delta\hat{y}
        }{%
            \vphi%
        } = \scalaireq{\hat{u}}{\vphi},
    \end{split}
\end{equation*}
which gives, with~\eqref{eq:new-convergence-34}, that
\begin{equation}\label{eq:control102}
    \begin{cases}\displaystyle
        \begin{array}{rclll}
            \primetemps{\hat{y}} - \Delta\hat{y} & = & \hat{u} &
            \text{in} & Q,\\
            \\
            \hat{y} & = & 0 & \text{on} & \Sigma,\\
            \\
            \condinitiale{\hat{y}} & = & {\hat{y}}_{0} & \text{in} &
            \Omega.
        \end{array}
    \end{cases}
\end{equation}
Furthermore, the norm $\normeohm{\cdot}$ being continuous, \textit{a fortiori}
weakly continuous,
\begin{equation*}
    \begin{array}{rclll}
        \bareps{y} & \longrightarrow & \hat{y} & \text{weakly in} &
        \Ldeux{Q}
    \end{array}\qquad\text{and}\qquad
    \normeohm{\condfinale{\bareps{y}}} < \veps
\end{equation*}
lead to
\begin{equation}\label{eq:control103}
    \begin{array}{rclll}
        \condfinale{\hat{y}} & = & 0 & \text{in} & \Omega.
    \end{array}
\end{equation}
Thus~\eqref{eq:control102}~and~\eqref{eq:control103} allow to conclude that
$\hat{y}\in\Ldeux{Q}$ satisfies
\begin{equation*}
    \begin{cases}\displaystyle
        \begin{array}{rclll}
            \primetemps{\hat{y}} - \Delta\hat{y} & = & \hat{u} &
            \text{in} & Q,\\
            \\
            \hat{y} & = & 0 & \text{on} & \Sigma,\\
            \\
            \condfinale{\hat{y}} & = & 0 & \text{in} & \Omega.
        \end{array}
    \end{cases}
\end{equation*}
Then, noting that $\hat{u}\in\uad$, since $\bareps{u}\in\uad$ and $\uad$ is
closed and therefore weakly closed, we get that the control-state pair
$\left({\hat{u},\hat{y}}\right)$ is admissible
for~\eqref{eq:control80}\eqref{eq:control81}\eqref{eq:control82}, so that
\begin{equation}\label{eq:control105}
    J(u,y)\leq J\!\left({\hat{u},\hat{y}}\right).
\end{equation}
On the other hand, passing~\eqref{eq:control95} to the limit when $\veps\to
0$, it comes $J\!\left({\hat{u},\hat{y}}\right)\leq J(u,y)$; that is to
say, with~\eqref{eq:control105}, that $J(u,y) =
J\!\left({\hat{u},\hat{y}}\right)$.

Then we conclude, by uniqueness of the optimal control-state pair $(u,y)$,
that $\left({\hat{u},\hat{y}}\right) = (u,y)$, which ends up proving the
following result.

\begin{proposition}%
    For any $\veps > 0$, the approached optimal control $\bareps{u}$ and
    the associated state $\bareps{y}$ satisfy
    \begin{equation*}
        \begin{cases}\displaystyle
            \begin{array}{rclll}
                \bareps{u} & \longrightarrow & u & \text{weakly in} &
                \Ldeux{Q},\\
                % \\
                \bareps{y} & \longrightarrow & y & \text{weakly in} &
                \Ldeux{Q},
            \end{array}
        \end{cases}
    \end{equation*}
    where $(u,y)$ is the optimal control-state pair
    for~\eqref{eq:control80}\eqref{eq:control81}\eqref{eq:control82}.
\end{proposition}

We establish below that we actually have more: the strong convergence.

\begin{theoreme}%
    For any $\veps > 0$, the approached optimal control $\bareps{u}$ and
    the associated state $\bareps{y}$ are such that, when $\veps \to 0$,
    \begin{equation*}
        \begin{cases}\displaystyle
            \begin{array}{rclll}
                \bareps{u} & \longrightarrow & u & \text{strongly in} &
                \Ldeux{Q},\\
                % \\
                \bareps{y} & \longrightarrow & y & \text{strongly in} &
                \Ldeux{Q},
            \end{array}
        \end{cases}
    \end{equation*}
    where $(u,y)$ is the optimal control-state pair
    for~\eqref{eq:control80}\eqref{eq:control81}\eqref{eq:control82}.
\end{theoreme}

\begin{proof}%
    From the results previously obtained, we have that
    \begin{equation}\label{eq:control112}
        \begin{array}{rclll}
            \bareps{u} & \longrightarrow & u & \text{weakly in} & \Ldeux{Q},
        \end{array}
    \end{equation}

    \begin{equation}\label{eq:control113}
        \begin{array}{rcllll}
            \bareps{y} & \longrightarrow & y & \text{weakly in} & \Ldeux{Q},
        \end{array}
    \end{equation}
    and
    \begin{equation*}
        J(u,y) = \lim_{\veps\to 0}J_{\veps}\!\left({\bareps{u}}\right).
    \end{equation*}
    Where, from~\eqref{eq:control106}\eqref{eq:control106p}
    and~\eqref{eq:control107}, the last equality above can still be written
    \begin{equation}\label{eq:control115}
        \normecq{y - z_{d}} + N\normecq{u} = \lim_{\veps\to 0}\left({%
            \normecq{\bareps{y} - z_{d}} + N\normecq{\bareps{u}} +
            \dfrac{1}{\veps}\normecq{\bareps{y} - \condinitiale{y}}
        }\right).
    \end{equation}
    But then, the norm $\normeq{\cdot}$ being continous, \textit{a
    fortiori} weakly lower semi-continous, it comes,
    with~\eqref{eq:control112} and~\eqref{eq:control113}, that
    \begin{equation*}
        \begin{cases}\displaystyle
            \begin{array}{rcll}
                \normecq{y - z_{d}} & \leq & \underset{\veps\to
                0}{\lim\inf}\, \normecq{\bareps{y} - z_{d}},\\
                \\
                \normecq{u} & \leq & \underset{\veps\to 0}{\lim\inf}\,
                \normecq{\bareps{u}}.
            \end{array}
        \end{cases}
    \end{equation*}
    From where it follows, with~\eqref{eq:control115}, that
    \begin{equation}\label{eq:control118}
        \normecq{y -z_{d}} = \lim_{\veps\to 0}\normecq{\bareps{y} - z_{d}},
    \end{equation}
    and
    \begin{equation}\label{eq:control119}
        \normecq{u} = \lim_{\veps\to 0}\normecq{\bareps{u}}.
    \end{equation}
    Hence, since
    \begin{equation*}
        \normecq{\bareps{y} - y} = \normecq{\bareps{y} - z_{d}} + \normeq{y
        - z_{d}} -2\scalaireq{%
            \bareps{y} - z_{d}
        }{%
            y - z_{d}
        },
    \end{equation*}
    we conclude with~\eqref{eq:control113} and~\eqref{eq:control118} that
    \begin{equation*}\label{eq:control120}
        \lim_{\veps\to 0}\normecq{\bareps{y} - y} = 0\qquad\ie\qquad
        \begin{array}{rclll}
            \bareps{y} & \longrightarrow & y & \text{strongly in} &
            \Ldeux{Q}.
        \end{array}
    \end{equation*}
    In a similar way,~\eqref{eq:control112} and~\eqref{eq:control119} lead
    to
    \begin{equation*}
        \begin{array}{rclll}
            \bareps{u} & \longrightarrow & u & \text{strongly in} &
            \Ldeux{Q},
        \end{array}
    \end{equation*}
    which ends up proving the announced result.
\end{proof}

\subsection{Approached optimality system}\label{sec:approachedso}

Let us start by recalling that, for any $\veps > 0$ and for the optimal
control $\bareps{u}\in\uad$, we have the existence of
\begin{equation*}
    {\bar{y}_{0}}_{\veps}\in\Ldeux{\Omega}\qquad\text{and}\qquad
    \bareps{y}\in\Ldeux{Q},
\end{equation*}
verifying
\begin{equation*}
    \begin{cases}\displaystyle
        \begin{array}{rclll}
            \primetemps{\bareps{y}} - \Delta\bareps{y} & = & \bareps{u} &
            \text{in} & Q,\\
            \\
            \bareps{y} & = & 0 & \text{on} & \Sigma,\\
            \\
            \condinitiale{\bareps{y}} & = & {\bar{y}_{0}}_{\veps} &
            \text{in} & \Omega,
        \end{array}
    \end{cases}
\end{equation*}
and the estimate
\begin{equation*}
    \normeohm{\condfinale{\bareps{y}}} < \veps.
\end{equation*}
So, given $v\in\uad$ and $\lambda\in{\Rbb}^{*}$; we have:
\begin{equation*}
    \begin{split}
        J_{\veps}\!\left({%
            \bareps{u} + \lambda\left({v - \bareps{u}}\right)
        }\right) &= \dfrac{1}{2}\normecq{%
            y\!\left({%
                \bareps{u} + \lambda\left({v - \bareps{u}}\right),
                {\bar{y}_{0}}_{\veps}
            }\right) - z_{d}
        } + \dfrac{N}{2}\normecq{%
            \bareps{u} + \lambda\left({v - \bareps{u}}\right)
        }\\
        &\qquad + \dfrac{1}{2\veps}\normecohm{%
            {\bar{y}_{0}}_{\veps} - \condinitiale{y}
        }\\
        & = \dfrac{1}{2}\normecq{%
            \bareps{y} - z_{d} + \lambda{\fieps}
        } + \dfrac{N}{2}\normecq{%
            \bareps{u} + \lambda\left({v - \bareps{u}}\right)
        }\\
        &\qquad + \dfrac{1}{2\veps}\normecohm{%
            {\bar{y}_{0}}_{\veps} - \condinitiale{y}
        }\\
        & = J_{\veps}\!\left({\bareps{u}}\right) +
        \dfrac{{\lambda}^{2}}{2}\left({%
            \normecq{\fieps} + N\normecq{v - \bareps{u}}
        }\right)\\
        &\qquad + \lambda\scalaireq{%
            \bareps{y} - z_{d}
        }{%
            \fieps{}
        } + \lambda N\scalaireq{%
            \bareps{u}
        }{%
            v - \bareps{u}
        },
    \end{split}
\end{equation*}
which gives
\begin{equation*}
    {\left.{%
        \dfrac{\diff}{\diff{\lambda}}J_{\veps}\!\left({%
            \bareps{u} + \lambda\left({v - \bareps{u}}\right)
        }\right)
    }\right|}_{\lambda = 0} = \scalaireq{%
        \bareps{y} - z_{d}
    }{%
        \fieps{}
    } + N\scalaireq{\bareps{u}}{v - \bareps{u}},
\end{equation*}
where $\fieps = y\!\left({v - \bareps{u}, {\bar{y}_{0}}_{\veps}}\right) -
y\!\left({0, {\bar{y}_{0}}_{\veps}}\right)$ is defined by
\begin{equation}\label{eq:control125}
    \begin{cases}\displaystyle
        \begin{array}{rclll}
            \primetemps{\fieps} - \Delta\fieps & = & v - \bareps{u} &
            \text{in} & Q,\\
            \\
            \fieps & = & 0 & \text{on} & \Sigma,\\
            \\
            \condinitiale{\fieps} & = & 0 & \text{in} & \Omega.
        \end{array}
    \end{cases}
\end{equation}
Hence, with the first-order optimality condition of Euler-Lagrange, we
obtain that the approached optimal control $\bareps{u}$ is the unique
element of $\uad$ satisfying
\begin{equation}\label{eq:control126}
    \scalaireq{%
        \bareps{y} - z_{d}
    }{%
        \fieps{}
    } + N\scalaireq{%
        \bareps{u}
    }{%
        v - \bareps{u}
    } \geq 0,\qquad\forall\,v\in\uad.
\end{equation}
Let introduce here the adjoint state $p_{\veps}\in\Ldeux{Q}$ by
\begin{equation*}
    \begin{cases}\displaystyle
        \begin{array}{rclll}
            -\primetemps{p_{\veps}} - \Delta p_{\veps} & = & \bareps{y} -
            z_{d} & \text{in} & Q,\\
            \\
            p_{\veps} & = & 0 & \text{on} & \Sigma,\\
            \\
            \condfinale{p_{\veps}} & = & 0 & \text{in} & \Omega.
        \end{array}
    \end{cases}
\end{equation*}
It comes with~\eqref{eq:control125} that
\begin{equation*}
    \begin{array}{rclll}
        -\primetemps{p_{\veps}} - \Delta p_{\veps} & = & \bareps{y} - z_{d}
        & \text{in} & Q
    \end{array}
\end{equation*}
leads to
\begin{equation*}
    \begin{split}
         \scalaireq{%
            \bareps{y} - z_{d}
        }{%
            \fieps{}
        } & = -\scalaireq{%
            \primetemps{p_{\veps}}
        }{%
            \fieps{}
        } - \scalaireq{%
            \Delta p_{\veps}
        }{%
            \fieps{}
        }\\
        & = \scalaireohm{%
            \condinitiale{p_{\veps}}
        }{%
            \condinitiale{\fieps}
        } - \scalaireohm{%
            \condfinale{p_{\veps}}
        }{%
            \condfinale{\fieps}
        } + \scalaireq{%
            p_{\veps}
        }{%
            \primetemps{\fieps}
        }\\
        &\qquad - \scalaireq{%
            p_{\veps}
        }{%
            \Delta\fieps{}
        } - \scalaires{%
            \deriveenormale{p_{\veps}}
        }{%
            \fieps{}
        } + \scalaires{%
            p_{\veps}
        }{%
            \deriveenormale{\fieps}
        }\\
        & = \scalaireq{%
            p_{\veps}
        }{%
            \primetemps{\fieps} - \Delta\fieps{}
        } = \scalaireq{%
            p_{\veps}
        }{%
            v - \bareps{u}
        },
    \end{split}
\end{equation*}
so that the optimality condition~\eqref{eq:control126} reduces to
\begin{equation*}
    \begin{array}{rcl}
        \scalaireq{%
            p_{\veps} + N\bareps{u}
        }{%
            v - \bareps{u}
        } & \geq & 0,\qquad \forall\,v\in\uad.
    \end{array}
\end{equation*}

We thus obtain the following result.

\begin{theoreme}\label{thm:soapproche}%
    Let $\veps > 0$. The control $\bareps{u}$ is unique solution
    of~\eqref{eq:control87} if and only if the quadruplet
    \begin{equation*}
        \left\{{%
            {\bar{y}_{0}}_{\veps}, \bareps{u}, \bareps{y}, p_{\veps}
        }\right\}\in \Ldeux{\Omega}\times {\left({\Ldeux{Q}}\right)}^{3}
    \end{equation*}
    is solution of the approached optimality system defined by the partial
    differential equations systems
    \begin{equation}
        \begin{cases}\displaystyle
            \begin{array}{rclll}
                \primetemps{\bareps{y}} - \Delta\bareps{y} & = & \bareps{u}
                & \text{in} & Q,\\
                \\
                \bareps{y} & = & 0 & \text{on} & \Sigma,\\
                \\
                \condinitiale{\bareps{y}} & = & {\bar{y}_{0}}_{\veps} &
                \text{in} & \Omega,
            \end{array}
        \end{cases}
    \end{equation}
    and
    \begin{equation}\label{eq:control130}
        \begin{cases}\displaystyle
            \begin{array}{rclll}
                -\primetemps{p_{\veps}} - \Delta p_{\veps} & = & \bareps{y}
                - z_{d} & \text{in} & Q,\\
                \\
                p_{\veps} & = & 0 & \text{on} & \Sigma,\\
                \\
                \condfinale{p_{\veps}} & = & 0 & \text{in} & \Omega,
            \end{array}
        \end{cases}
    \end{equation}
    the estimate
    \begin{equation}
        \begin{array}{rcl}
            \normeohm{%
                \condfinale{\bareps{y}}
            } & < & \veps,
        \end{array}
    \end{equation}
    and the variational inequality
    \begin{equation}
        \begin{array}{rcl}
            \scalaireq{%
                p_{\veps} + N\bareps{u}
            }{%
                v - \bareps{u}
            } & \geq & 0,\qquad \forall\,v\in\uad.
        \end{array}
    \end{equation}
\end{theoreme}

\subsection{Singular optimality system}\label{sec:singularso}

From the results obtained in Section~\ref{sec:convergence}, we have:
\begin{equation*}
    \begin{array}{rclll}
        \bareps{u} & \longrightarrow & u & \text{strongly in} & \Ldeux{Q}
    \end{array}
\end{equation*}
\begin{equation*}
    \begin{array}{rclll}
        \bareps{y} & \longrightarrow & y & \text{strongly in} & \Ldeux{Q},
    \end{array}
\end{equation*}
where $(u,y)$ is the optimal control-state pair
of~\eqref{eq:control80}\eqref{eq:control81}\eqref{eq:control82}.

So that,~\eqref{eq:control130} being well-posed in the sense of Hadamard,
it follows that there exists $p\in\Ldeux{Q}$ such that
\begin{equation*}
    \begin{array}{rclll}
        p_{\veps} & \longrightarrow & p & \text{strongly in} & \Ldeux{Q}.
    \end{array}
\end{equation*}
Then, we easily pass the results of Theorem~\ref{thm:soapproche} to the
limit, when $\veps\to 0$, to obtain that the strong singular optimaly
system characterizing the optimal control-state pair
of~\eqref{eq:control80}\eqref{eq:control81}\eqref{eq:control82} is as
specified below.

\begin{theoreme}\label{thm:sosingulier}%
    The control-state pair $(u,y)$ is unique solution of the control
    problem~\eqref{eq:control80}\eqref{eq:control81}\eqref{eq:control82} if
    and only if the triple
    \begin{equation*}
        \left\{{%
            u,y,p
        }\right\}\in {\left({\Ldeux{Q}}\right)}^{3}
    \end{equation*}
    is solution of the singular optimality system defined by the partial
    differential equation systems
    \begin{equation}
        \begin{cases}\displaystyle
            \begin{array}{rclll}
                \primetemps{y} - \Delta y & = & u & \text{in} & Q,\\
                \\
                y & = & 0 & \text{on} & \Sigma,\\
                \\
                \condfinale{y} & = & 0 & \text{in} & \Omega,
            \end{array}
        \end{cases}
    \end{equation}
    and
    \begin{equation}
        \begin{cases}\displaystyle
            \begin{array}{rclll}
                -\primetemps{p} - \Delta p & = & y - z_{d} & \text{in} &
                Q,\\
                \\
                p & = & 0 & \text{on} & \Sigma,\\
                \\
                \condfinale{p} & = & 0 & \text{in} & \Omega,
            \end{array}
        \end{cases}
    \end{equation}
    and the variational inequality
    \begin{equation}
        \begin{array}{rcll}
            \scalaireq{%
                p + Nu
            }{%
                v - u
            } & \geq & 0,\qquad\forall\,v\in\uad.
        \end{array}
    \end{equation}
\end{theoreme}
