\section{Contrôlabilité de l'équation de la chaleur rétrograde}

On interprète la problème~\eqref{eq:initial-problem} comme un problème
inverse; on considère le système
\begin{equation}\label{eq:inverse-problem-1}
    \begin{cases}\displaystyle
        \begin{array}{rl}
            \primetemps{y} - \Delta y = v & \text{dans}\ Q,\\
            \\
            y = 0 & \text{sur}\ \Sigma,
        \end{array}
    \end{cases}
\end{equation}
se posant le problème de trouver $\displaystyle y_{0}\in\Ldeux{\Omega}$ tel
que si
\begin{equation}\label{eq:inverse-problem-2}
    \begin{cases}\displaystyle
        \begin{array}{rl}
            \primetemps{y} - \Delta y = v & \text{dans}\ Q,\\
            \\
            y = 0 & \text{sur}\ \Sigma,\\
            \\
            \condinitiale{y} = y_{0} & \text{dans}\ \Omega,
        \end{array}
    \end{cases}
\end{equation}
alors
\begin{equation}\label{eq:inverse-problem-3}
    \condfinale{y} = 0\quad\text{dans}\ \Omega.
\end{equation}
On dit que~\eqref{eq:inverse-problem-2}\eqref{eq:inverse-problem-3}
constitue un problème de contrôlabilité exacte associée au
problème~\eqref{eq:initial-problem}.

\begin{remarque}%
    Le problème de contrôlabilité est bien défini. En effet, pour tous
    $v\in\Ldeux{Q}$ et $y_{0}\in\Ldeux{\Omega}$, on sait que l'équation de
    la chaleur~\eqref{eq:inverse-problem-2} est bien posé au sens de
    Hadamard. Soit qu'elle admet une solution unique
    \begin{equation}\label{eq:well-posedness-heat-equation}
        y\!\left({v,y_{0}}\right)\in \Ldeux{0,T; H_{0}^{1}(\Omega)}\cap
        \Cscr\!\left({[0,T];\Ldeux{\Omega}}\right)
    \end{equation}
    dépendant continûment des données. On en déduit, qu'après modification
    éventuelle sur un ensemble de mesure nulle, la solution
    $y\!\left({v,y_{0}}\right)$ de~\eqref{eq:inverse-problem-2} se confond
    à une fonction continue de $[0,T]$ dans $\Ldeux{\Omega}$. De sorte
    qu'on peut bien parler de la valeur terminale
    $y\!\left({\cdot,T;v,y_{0}}\right)$ de $y\!\left({v,y_{0}}\right)$ dans
    $\Omega$.
\end{remarque}

\begin{remarque}%
    De plus, l'application
    \begin{equation}
        \left({v,y_{0}}\right)\longmapsto y\!\left({v,y_{0}}\right) =
        y(v,0) + y\!\left({0,y_{0}}\right)
    \end{equation}
    étant linéaire continue de $\Ldeux{Q}\times\Ldeux{\Omega}$ dans
    $\Ldeux{0,T;H_{0}^{1}(\Omega)}\cap \Cscr\!\left({%
        [0,T];\Ldeux{\Omega}
    }\right)\subset\Ldeux{Q}$, le problème de contrôlabilité
    exacte~\eqref{eq:inverse-problem-2}\eqref{eq:inverse-problem-3} est
    équivalent à celui consistant à trouver $y_{0}\in\Ldeux{\Omega}$ tel
    que, si
    \begin{equation}\label{eq:default-inverse-problem}
        \begin{cases}\displaystyle
            \begin{array}{rl}
                \primetemps{y} - \Delta y = 0 & \text{dans}\ Q,\\
                \\
                y = 0 & \text{sur}\ \Sigma,\\
                \\
                y\!\left({\cdot,y_{0}}\right) = 0 & \text{dans}\ \Omega,
            \end{array}
        \end{cases}
    \end{equation}
    alors $y$ satisfait~\eqref{eq:inverse-problem-3}.
\end{remarque}

On approche, par argument de densité, le problème de contrôlabilité
exacte~\eqref{eq:default-inverse-problem}\eqref{eq:inverse-problem-3}, par
un problème, dit de contrôlabilité approchée et en réponse de quoi on a le
résultat suivant.

\begin{proposition}\label{propo:controllability-result-1}%
    Notons
    \begin{equation}\label{eq:controllability-result-1}
        E = \left\{{%
            y(\cdot,T)\,;\ y_{0}\in\Ldeux{\Omega}
        }\right\}
    \end{equation}
    l'ensemble des valeurs terminales de $y$, solution unique
    de~\eqref{eq:default-inverse-problem}, lorsque la donnée initiale
    $y_{0}$ parcourt $\Ldeux{\Omega}$. Alors l'ensemble $E$ est dense dans
    $\Ldeux{\Omega}$.
\end{proposition}

\begin{proof}%
    On a clairement que $E$ est un sous-espace vectoriel de
    $\Ldeux{\Omega}$. D'où, par le Théorème de Hahn-Banach, $E$ est dense
    dans $\Ldeux{\Omega}$ si et seulement si $E^{\perp} = \{0\}$.

    Soit $k\in E^{\perp}$; on a
    \begin{equation}\label{eq:proof-controllability-26}
        \forall\,y_{0}\in\Ldeux{\Omega},\qquad \scalaireohm{k}{y(\cdot,T)}
        = 0.
    \end{equation}
    Mais, de~\eqref{eq:default-inverse-problem}, on a pour toute fonction
    test $\vphi\in\Dscr(Q)$, que:
    \begin{equation*}
        \begin{split}
            \scalaireq{\primetemps{y} - \Delta y}{\vphi} = 0 &\iff
            \scalaireq{\primetemps{y}}{\vphi} - \scalaireq{\Delta y}{\vphi}
            = 0\\
            &\iff \scalaireq{\primetemps{y}}{\vphi} - \scalaireq{\Delta
            y}{\vphi} = 0.
        \end{split}
    \end{equation*}
\end{proof}

On en déduit le corollaire suivant.

\begin{corollaire}\label{coro:controllability-result-2}%
    Pour tout $\veps > 0$, il existe ${y_{0}}_{\veps}\in\Ldeux{\Omega}$ tel
    que la solution
    \begin{equation}\label{eq:controllability-result-2}
        y_{\veps}\in \Ldeux{0,T;H_{0}^{1}(\Omega)}\cap
        \Cscr\!\left({O,T;\Ldeux{\Omega}}\right)
    \end{equation}
    de
    \begin{equation}\label{eq:controllability-result-3}
        \begin{cases}\displaystyle
            \begin{array}{rl}
                \primetemps{y_{\veps}} - \Delta y_{\veps} = v &
                \text{dans}\ Q,\\
                \\
                y_{\veps} = 0 & \text{sur}\ \Sigma,\\
                \\
                y_{\veps}\!\left({\cdot,0}\right) = {y_{0}}_{\veps} &
                \text{dans}\ \Omega,
            \end{array}
        \end{cases}
    \end{equation}
    satisfait
    \begin{equation}\label{eq:controllability-result-4}
        \normeohm{y_{\veps}\!\left({\cdot,T}\right)} < \veps.
    \end{equation}
\end{corollaire}

De plus, on a le théorème caractérisant l'existence d'une solution
régulière à l'équation de la chaleur rétrograde.

\begin{theoreme}\label{thm:controllability-result-5}%
    Soit $v\in\Ldeux{Q}$. L'équation de la chaleur rétrograde
    \begin{equation}\label{eq:initial-problem-2}
        \begin{cases}\displaystyle
            \begin{array}{rl}
                \primetemps{z} - \Delta z = v & \text{dans}\ Q,\\
                \\
                z = 0 & \text{sur}\ \Sigma,\\
                \\
                z(\cdot,T) = 0 & \text{dans}\ \Omega
            \end{array}
        \end{cases}
    \end{equation}
    admet une unique solution régulière
    \begin{equation}
        z\in \Ldeux{0,T;H_{0}^{1}(\Omega)}\cap
        \Cscr\!\left({0,T;\Ldeux{Q}}\right)
    \end{equation}
    si et seulement si la suite $\displaystyle
    {\left({{y_{0}}_{\veps}}\right)}_{\veps}$ est bornée dans
    $\Ldeux{\Omega}$.
\end{theoreme}

\begin{proof}%
\end{proof}
