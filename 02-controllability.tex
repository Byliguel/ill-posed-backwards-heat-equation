\section{Controllability of the ill-posed backwards heat
equation}\label{sec:controllability}

In the present section, we introduce the controllability viewpoint here
proposed. Which consists, starting from the notion of controllability
introduced by J.-L.~Lions in~\cite[222]{lions1}, in interpreting the
initial state equation as an inverse problem (we also say a controllability
problem).

We establish that, when it exists, the solution of the ill-posed backwards
heat equation~\eqref{eq:initial-problem} can be approximated by that of the
approached controllability problem. Which implicitly allows us to propose a
necessary and sufficient condition for the existence of a regular solution
to the problem~\eqref{eq:initial-problem}.

% We therefore interpret~\eqref{eq:initial-problem} as an inverse problem
% whose observation objective consists in the final condition
% \begin{equation*}
%     \begin{array}{rclll}
%         \condfinale{z} & = & 0 & \text{in} & \Omega.
%     \end{array}
% \end{equation*}

More precisely, we consider the problem of finding $\displaystyle
y_{0}\in\Ldeux{\Omega}$ such that if
\begin{equation}\label{eq:inverse-problem-2}
    \begin{cases}\displaystyle
        \begin{array}{rclll}
            \primetemps{y} - \Delta y & = & v & \text{in} & Q,\\
            \\
            y & = & 0 & \text{on} & \Sigma,\\
            \\
            \condinitiale{y} & = & y_{0} & \text{in} & \Omega,
        \end{array}
    \end{cases}
\end{equation}
then
\begin{equation}\label{eq:inverse-problem-3}
    \begin{array}{rclll}
        \condfinale{y} & = & 0 & \text{in} & \Omega.
    \end{array}
\end{equation}
We say that~\eqref{eq:inverse-problem-2}\eqref{eq:inverse-problem-3}
constitute an exact controllability problem associated with the ill-posed
backwards heat equation~\eqref{eq:initial-problem}.

\begin{remarque}%
    The controllability problem is well defined. Indeed, it is well known
    that for any $v\in\Ldeux{Q}$ and
    $y_{0}\in\Ldeux{\Omega}$,~\eqref{eq:inverse-problem-2} is well posed in
    the sense of Hadamard. That is to say that its admits a unique solution
    \begin{equation*}\label{eq:well-posedness-heat-equation}
        y\!\left({v,y_{0}}\right)\in \Vbb := \Ldeux{%
            \left]{0,T}\right[; H_{0}^{1}(\Omega)
        }\cap \Cscr\!\left({[0,T];\Ldeux{\Omega}}\right)
    \end{equation*}
    which depends continuously on the data. We deduce from this, after
    possible modification on a set of zero measure, that the solution
    $y\!\left({v,y_{0}}\right)$ of~\eqref{eq:inverse-problem-2} merges with
    a continuous function from $(0,T)$ to $\Ldeux{\Omega}$.

    Thus we can indeed speak of the final value
    $y\!\left({\cdot,T;v,y_{0}}\right)$ of $y\!\left({v,y_{0}}\right)$ in
    $\Omega$.
\end{remarque}

\begin{remarque}%
    Let us denote
    \begin{itemize}
        \item $y^{v}$ the unique solution of~\eqref{eq:inverse-problem-2};
        \item $y_{0}^{v}\in \Vbb \subset \Ldeux{Q}$ that of
            \begin{equation*}\label{eq:null-initial-condition}
                \begin{cases}\displaystyle
                    \begin{array}{rclll}
                        \primetemps{y_{0}^{v}} - \Delta y_{0}^{v} & = & v &
                        \text{in} & Q,\\
                        \\
                        y_{0}^{v} & = & 0 & \text{on} & \Sigma,\\
                        \\
                        y_{0}^{v}\!\left({\cdot,0}\right) & = & 0 &
                        \text{in} & \Omega,
                    \end{array}
                \end{cases}
            \end{equation*}
        \item and $y^{0}\in \Vbb \subset \Ldeux{Q}$ that of
            \begin{equation}\label{eq:null-source-term}
                \begin{cases}\displaystyle
                    \begin{array}{rclll}
                        \primetemps{y^{0}} - \Delta y^{0} & = & 0 &
                        \text{in} & Q,\\
                        \\
                        y^{0} & = & 0 & \text{on} & \Sigma,\\
                        \\
                        y^{0}\!\left({\cdot,0}\right) & = & y_{0} &
                        \text{in} & \Omega.
                    \end{array}
                \end{cases}
            \end{equation}
    \end{itemize}
    Then, the mapping
    \begin{equation*}
        \left({v,y_{0}}\right)\longmapsto y^{v} = y_{0}^{v} + y^{0},
    \end{equation*}
    being linear and continuous from $\Ldeux{Q}\times \Ldeux{Q}$ to
    $\Vbb\subset \Ldeux{Q}$, we deduce that the exact controllability
    problem~\eqref{eq:inverse-problem-2}\eqref{eq:inverse-problem-3} is
    equivalent to the following
    \begin{equation}\label{eq:inverse-problem-4}
        \begin{cases}\displaystyle
            \text{find}\ y_{0}\in \Ldeux{\Omega}\ \text{such that:}\\
            \text{if}\ y^{0}\ \text{is solution
            of}~\eqref{eq:null-source-term},\ \text{then}\\
            y^{0}\!\left({\cdot,T}\right) = -y_{0}^{v}\!\left({\cdot,
            T}\right)\quad \text{in}\ \Omega.
        \end{cases}
    \end{equation}
\end{remarque}

% \begin{remarque}\label{rq:naturalassumption}%
%     Avec ce qui précède, il apparaît naturel de considérer, dans le cadre
%     du problème de contrôle, que c'est l'ensemble
%     \begin{equation*}
%         \left\{{%
%             (v,z)\in\Ascr:\ \condinitiale{z}\in\Ldeux{\Omega}
%         }\right\}
%     \end{equation*}
%     qui est non vide et non seulement $\Ascr$.
% \end{remarque}

With this last remark, we start by approaching the exact controllability
problem~\eqref{eq:inverse-problem-4} by density argument as specified
below.

\begin{proposition}\label{propo:controllability-result-1}%
    When the initial data $y_{0}$ traverses $\Ldeux{\Omega}$, the set
    \begin{equation*}\label{eq:controllability-result-1}
        E = \left\{{%
            y^{0}\!\left({\cdot,T}\right)\,;\ y_{0}\in\Ldeux{\Omega}
        }\right\},
    \end{equation*}
    described by the final values of the solution $y^{0}$
    of~\eqref{eq:null-source-term}, is dense in $\Ldeux{\Omega}$.
\end{proposition}

\begin{proof}%
    It is clear that the set $E$ constitute a vector subspace of
    $\Ldeux{\Omega}$. From where, by the Hahn-Banach Theorem, $E$ is dense
    in $\Ldeux{\Omega}$ if and only if $E^{\perp} = \{0\}$.

    Let us consider $k\in E^{\perp}$; so we have
    \begin{equation*}\label{eq:proof-controllability-26}
        \forall\,y_{0}\in\Ldeux{\Omega},\qquad
        \scalaireohm{k}{y^{0}\!\left({\cdot,T}\right)} = 0.
    \end{equation*}
    But it comes from~\eqref{eq:null-source-term} that, for any test
    function $\vphi\in\Dscr(Q)$, we have:
    \begin{equation*}
        \begin{split}
            \scalaireq{\primetemps{y^{0}} - \Delta y^{0}}{\vphi} = 0 &\iff
            \scalaireq{\primetemps{y^{0}}}{\vphi} - \scalaireq{\Delta
            y^{0}}{\vphi} = 0\\
            &\iff \scalaireohm{%
                y^{0}\!\left({\cdot,T}\right)
            }{%
                \vphi\!\left({\cdot,T}\right)
            } - \scalaireohm{%
                y_{0}
            }{%
                \vphi\!\left({\cdot,0}\right)
            }\\
            &\qquad - \scalaireq{y^{0}}{\primetemps{\vphi}} - \scalaireq{y^{0}
            }{%
                \Delta \vphi
            } + \scalaires{y^{0}
            }{\deriveenormale{\vphi}
            }\\
            &\qquad\qquad - \scalaires{\deriveenormale{y^{0}}}{\vphi} = 0
        \end{split}
    \end{equation*}
    \ie{}
    \begin{equation}\label{eq:controllability-proof-30}
        % \begin{split}
            \scalaireohm{%
                y^{0}\!\left({\cdot,T}\right)
            }{%
                \vphi\!\left({\cdot,T}\right)
            }% &
            - \scalaireohm{%
                y_{0}
            }{%
                \vphi\!\left({\cdot,0}\right)
            } - \scalaireq{%
                y^{0}
            }{%
                \primetemps{\vphi}
            }%\\
            %&\qquad
            - \scalaireq{%
                y^{0}
            }{%
                \Delta\vphi
            } = 0.
        % \end{split}
    \end{equation}
    Choosing in~\eqref{eq:controllability-proof-30}, $\vphi$ such that
    \begin{equation}\label{eq:controllability-proof-31}
        \begin{cases}\displaystyle
            \begin{array}{rclll}
                -\primetemps{\vphi} - \Delta\vphi & = & 0 & \text{in} & Q,\\
                % \\
                % \vphi & = & 0 & \text{on} & \Sigma,\\
                \\
                \vphi\!\left({\cdot,T}\right) & = & k & \text{in} &
                \Omega,
            \end{array}
        \end{cases}
    \end{equation}
    it comes that~\eqref{eq:controllability-proof-30} is equivalent to
    \begin{equation}\label{eq:controllability-proof-32}
        \scalaireohm{k}{y^{0}\!\left({\cdot,T}\right)} - \scalaireohm{%
            y_{0}
        }{%
            \vphi\!\left({\cdot,0}\right)
        } = 0,
    \end{equation}
    where
    \begin{equation*}
        k\in E^{\perp}\iff \scalaireohm{k}{y^{0}\!\left({\cdot,T}\right)} =
        0.
    \end{equation*}
    Thus~\eqref{eq:controllability-proof-32} becomes
    \begin{equation}\label{eq:controllability-proof-33}
        \forall\,y_{0}\in\Ldeux{\Omega},\qquad \scalaireohm{y_{0}}{%
            \vphi\!\left({\cdot,0}\right)
        } = 0.
    \end{equation}
    But we can still choose, in~\eqref{eq:controllability-proof-33},
    $\displaystyle y_{0} = \vphi\!\left({\cdot,0}\right)$ in $\Omega$, and
    then it follows
    \begin{equation*}
        \normecohm{\vphi\!\left({\cdot,0}\right)} = 0\qquad\ie\qquad
        \begin{array}{rclll}
            \vphi\!\left({\cdot,0}\right) & = & 0 & \text{in} & \Omega.
        \end{array}
    \end{equation*}
    Which brings, with~\eqref{eq:controllability-proof-31}, that $\vphi$ is
    solution of
    \begin{equation*}
        \begin{cases}\displaystyle
            \begin{array}{rclll}
                -\primetemps{\vphi} - \Delta\vphi & = & 0 & \text{in} &
                Q,\\
                \\
                \vphi & = & 0 & \text{on} & \Sigma,\\
                \\
                \vphi\!\left({\cdot,0}\right) & = & 0 & \text{in} &
                \Omega,
            \end{array}
        \end{cases}
    \end{equation*}
    that is to say that $\vphi\equiv 0$ and consequently that
    \begin{equation*}
        \begin{array}{rclcll}
            \vphi\!\left({\cdot,T}\right) & = & k & = 0 & \text{in} &
            \Omega.
        \end{array}
    \end{equation*}
    From where we deduce that $E^{\perp} = \{0\}$ and therefore that $E$ is
    dense in $\Ldeux{\Omega}$.
\end{proof}

The following result is then immediate.

\begin{corollaire}\label{coro:controllability-result-2}%
    For any $\veps > 0$ and $\kappa\in\Ldeux{\Omega}$, there exists
    ${y_{0}}_{\veps}\in\Ldeux{\Omega}$ such that the unique solution
    $\displaystyle y_{\veps}^{0}\in \Vbb$ of
    \begin{equation}\label{eq:controllability-result-3}
        \begin{cases}\displaystyle
            \begin{array}{rclll}
                \primetemps{y_{\veps}^{0}} - \Delta y_{\veps}^{0} & = & 0 &
                \text{in} & Q,\\
                \\
                y_{\veps}^{0} & = & 0 & \text{on} & \Sigma,\\
                \\
                y_{\veps}^{0}\!\left({\cdot,0}\right) & = & {y_{0}}_{\veps}
                & \text{in} & \Omega,
            \end{array}
        \end{cases}
    \end{equation}
    also satisfies
    \begin{equation}\label{eq:controllability-result-4}
        \normeohm{y_{\veps}^{0}\!\left({\cdot,T}\right) - \kappa} < \veps.
    \end{equation}
\end{corollaire}

\begin{proof}%
    Indeed, due to the density of the set $\displaystyle E = \left\{{%
        y^{0}\!\left({\cdot,T}\right)\,;\ y_{0}\in \Ldeux{\Omega}
    }\right\}$ in $\Ldeux{\Omega}$, it comes that for any target
    $\kappa\in\Ldeux{\Omega}$, we can find ${y_{0}}_{\veps}\in
    \Ldeux{\Omega}$ such that the final value of the solution
    $\displaystyle {y_{0}}_{\veps}$ of~\eqref{eq:controllability-result-3}
    approaches $\kappa$ to $\veps$ close. From where the result.
\end{proof}

It also follows, from the above, the result below.

\begin{corollaire}\label{coro:controllability-result-3}%
    Given $v\in\Ldeux{Q}$ and $\veps > 0$, there exists ${y_{0}}_{\veps}\in
    \Ldeux{\Omega}$ such that the unique solution $\displaystyle
    y_{\veps}^{v}\in \Vbb$ of
    \begin{equation}\label{eq:controllability-result-5}
        \begin{cases}\displaystyle
            \begin{array}{rclll}
                \primetemps{y_{\veps}^{v}} - \Delta y_{\veps}^{v} & = & v &
                \text{in} & Q,\\
                \\
                y_{\veps}^{v} & = & 0 & \text{on} & \Sigma,\\
                \\
                y_{\veps}^{v}\!\left({\cdot,0}\right) & = & {y_{0}}_{\veps}
                & \text{in} & \Omega,
            \end{array}
        \end{cases}
    \end{equation}
    satisfies
    \begin{equation}\label{eq:controllability-result-6}
        \normeohm{y_{\veps}^{v}\!\left({\cdot,T}\right)} < \veps.
    \end{equation}
\end{corollaire}

\begin{proof}%
    Let $v\in\Ldeux{Q}$ and $\veps > 0$. We know that there exists a unique
    $y_{0}^{v}\in \Vbb$, almost everywhere equal to a continuous mapping
    from $(0,T)$ to $\Ldeux{\Omega}$, solution of
    \begin{equation*}\label{eq:controllability-proof-39}
        \begin{cases}\displaystyle
            \begin{array}{rclll}
                \primetemps{y_{0}^{v}} - \Delta y_{0}^{v} & = & v &
                \text{in} & Q,\\
                \\
                y_{0}^{v} & = & 0 & \text{on} & \Sigma,\\
                \\
                y_{0}^{v}\!\left({\cdot,0}\right) & = & 0 & \text{in} &
                \Omega.
            \end{array}
        \end{cases}
    \end{equation*}
    So it follows from Corollary~\ref{coro:controllability-result-2} that
    there exists ${y_{0}}_{\veps}\in\Ldeux{\Omega}$ such that the solution
    $y_{\veps}^{0}\in \Vbb$ of~\eqref{eq:controllability-result-3}
    satisfies
    \begin{equation*}\label{eq:controllability-proof-40}
        \normeohm{%
            y_{\veps}^{0}\!\left({\cdot,T}\right) - \left({%
                -y_{0}^{v}\!\left({\cdot,T}\right)
            }\right)
        } < \veps.
    \end{equation*}
    Which allows us to conclude that $\displaystyle y_{\veps}^{v} =
    \left({%
        y_{0}^{v} + y_{\veps}^{0}
    }\right) \in \Vbb$ is unique solution
    of~\eqref{eq:controllability-result-5}, with the estimate
    \begin{equation*}
        \normeohm{y_{\veps}^{v}\!\left({\cdot,T}\right)} = \normeohm{%
            y_{0}^{\veps}\!\left({\cdot,T}\right) +
            y_{0}^{v}\!\left({\cdot,T}\right)
        } < \veps.
    \end{equation*}
\end{proof}

Furthermore, we manage to characterize, as stated below, the existence of a
regular solution to the ill-posed backwards heat equation.

\begin{theoreme}\label{thm:controllability-result-5}%
    Let $v\in\Ldeux{Q}$. The ill-posed backwards heat equation
    \begin{equation}\label{eq:initial-problem-2}
        \begin{cases}\displaystyle
            \begin{array}{rclll}
                \primetemps{z} - \Delta z & = & v & \text{in} & Q,\\
                \\
                z & = & 0 & \text{on} & \Sigma,\\
                \\
                z(\cdot,T) & = & 0 & \text{in} & \Omega,
            \end{array}
        \end{cases}
    \end{equation}
    admits a regular solution $z\in \Vbb$ if and only if the sequence
    $\displaystyle {\left({{y_{0}}_{\veps}}\right)}_{\veps}$, given
    by~Corollary~\ref{coro:controllability-result-3}, is bounded in
    $\Ldeux{\Omega}$.
\end{theoreme}

\begin{proof}%
    \begin{enumerate}
        \item Let $\veps > 0$. From
            Corollary~\ref{coro:controllability-result-3}, there exists
            ${y_{0}}_{\veps}\in \Ldeux{\Omega}$ such that $\displaystyle
            y_{\veps}^{v}\in \Vbb$ is solution of
            \begin{equation}\label{eq:proof-thm-controllability-42}
                \begin{cases}\displaystyle
                    \begin{array}{rclll}
                        \primetemps{y_{\veps}^{v}} - \Delta y_{\veps}^{v} &
                        = & v & \text{in} & Q,\\
                        \\
                        y_{\veps}^{v} & = & 0 & \text{on} & \Sigma,\\
                        \\
                        y_{\veps}^{v}\!\left({\cdot,0}\right) & = &
                        {y_{0}}_{\veps} & \text{in} & \Omega,
                    \end{array}
                \end{cases}
            \end{equation}
            with the estimate
            \begin{equation}\label{eq:proof-thm-controllability-43}
                \normeohm{y_{\veps}^{v}\!\left({\cdot, T}\right)} < \veps.
            \end{equation}
            Then, we generate sequences
            \begin{equation*}
                {\left({{y_{0}}_{\veps}}\right)}_{\veps}\subset
                \Ldeux{\Omega}\qquad\text{and}\qquad
                {\left({y_{\veps}^{v}}\right)}_{\veps}\subset \Vbb.
            \end{equation*}
            Let suppose that the sequence
            ${\left({{y_{0}}_{\veps}}\right)}_{\veps}$ is bounded in
            $\Ldeux{\Omega}$. It therefore follows, the homogeneous
            Dirichlet problem~\eqref{eq:proof-thm-controllability-42} for
            the heat equation being well-posed, that the sequence
            ${\left({y_{\veps}^{v}}\right)}_{\veps}$ is bounded in $\Vbb$,
            and also in $\Ldeux{Q}$.

            So we can extract from
            ${\left({{y_{0}}_{\veps}}\right)}_{\veps}$ and
            ${\left({y_{\veps}^{v}}\right)}_{\veps}$ subsequences, still
            denoted in the same way, which converge weakly in
            $\Ldeux{\Omega}$ and $\Vbb$, respectively.

            So that, there exist $\displaystyle y_{0}\in \Ldeux{\Omega}$
            and $\displaystyle y^{v}\in \Vbb$, such that
            \begin{equation*}\label{eq:proof-thm-controllability-44}
                \begin{cases}\displaystyle
                    \begin{array}{rclll}
                        {y_{0}}_{\veps} & \longrightarrow & y_{0} &
                        \text{weakly in} & \Ldeux{\Omega},\\
                        \\
                        y_{\veps}^{v} & \longrightarrow & y^{v} &
                        \text{weakly in} & \Vbb.
                    \end{array}
                \end{cases}
            \end{equation*}
            Thus we immediately have, by continuity of the norm, that
            \begin{equation*}\label{eq:proof-thm-controllability-45}
                \normeohm{y_{\veps}^{v}\!\left({\cdot,T}\right)} < \veps
                \quad\text{and}\quad
                \begin{array}{rclll}
                    y_{\veps}^{v} & \longrightarrow & y^{v} & \text{weakly
                    in} & \Vbb
                \end{array}
            \end{equation*}
            imply%, $y_{\veps}^{v}$ being almost everywhere equal to a
            % continuous mapping from $(0,T)$ to $\Ldeux{\Omega}$, that
            \begin{equation}\label{eq:proof-thm-controllability-46}
                \begin{array}{rclll}
                    y^{v}\!\left({\cdot, T}\right) & = & 0 & \text{in} &
                    \Omega.
                \end{array}
            \end{equation}
            Moreover, since $y_{\veps}^{v}\in\Vbb\subset
            \Ldeux{0,T\,;H_{0}^{1}(\Omega)}$ and by continuity of the
            zero-order trace operator, we have that
            \begin{equation*}
                \begin{array}{rclll}
                    y_{\veps}^{v} & = & 0 & \text{on} & \Sigma
                \end{array}
                \qquad\text{and}\qquad
                \begin{array}{rclll}
                    y_{\veps}^{v} & \longrightarrow & y^{v} & \text{weakly
                    in} & \Vbb
                \end{array}
            \end{equation*}
            imply
            \begin{equation}\label{eq:proof-thm-controllability-47}
                \begin{array}{rclll}
                    y^{v} & = & 0 & \text{on} & \Sigma.
                \end{array}
            \end{equation}
            Now, let us multiply~\eqref{eq:proof-thm-controllability-42} by
            $\vphi\in \Dscr(Q)$ and integrate by parts over $Q$; we get
            \begin{equation*}
                \begin{split}
                    {} & \primetemps{y_{\veps}^{v}} - \Delta y_{\veps}^{v}
                    = v\ \text{in}\ Q \iff \scalaireq{%
                        \primetemps{y_{\veps}^{v}} - \Delta y_{\veps}^{v}
                    }{\vphi} = \scalaireq{v}{\vphi}\\
                    &\qquad\iff \scalaireohm{%
                        \condfinale{y_{\veps}^{v}}
                    }{%
                        \condfinale{\vphi}
                    } - \scalaireohm{%
                        {y_{0}}_{\veps}
                    }{%
                        \condinitiale{\vphi}
                    }\\
                    &\qquad\qquad - \scalaireq{%
                        y_{\veps}^{v}
                    }{%
                        \primetemps{\vphi}
                    } - \scalaireq{%
                        y_{\veps}^{v}
                    }{%
                        \Delta\vphi{}
                    } = \scalaireq{v}{\vphi},
                \end{split}
            \end{equation*}
            which gives, by passing to the limit,
            \begin{equation*}
                \begin{split}
                    {} & \scalaireohm{%
                        \condfinale{y^{v}}
                    }{%
                        \condfinale{\vphi}
                    } - \scalaireohm{%
                        y_{0}
                    }{%
                        \condinitiale{\vphi}
                    } - \scalaireq{%
                        y^{v}
                    }{%
                        \primetemps{\vphi}
                    }\\
                    &\qquad\qquad - \scalaireq{%
                        y^{v}
                    }{%
                        \Delta\vphi{}
                    } = \scalaireq{v}{\vphi}\\
                    &\qquad \iff \scalaireohm{%
                        \condinitiale{y^{v}} - y_{0}
                    }{%
                        \condinitiale{\vphi}
                    } + \scalaireq{%
                        \primetemps{y^{v}} - \Delta y^{v}
                    }{%
                        \vphi{}
                    } = \scalaireq{v}{\vphi}.
                \end{split}
            \end{equation*}
            This last equality being valid for any $\vphi\in\Dscr(Q)$, it
            follows with~\eqref{eq:proof-thm-controllability-47} that
            \begin{equation*}
                \begin{cases}\displaystyle
                    \begin{array}{rclll}
                        \primetemps{y^{v}} - \Delta y^{v} & = & v &
                        \text{in} & Q,\\
                        \\
                        y^{v} & = & 0 & \text{on} & \Sigma,\\
                        \\
                        \condinitiale{y^{v}} & = & y_{0} & \text{in} &
                        \Omega,
                    \end{array}
                \end{cases}
            \end{equation*}
            which implies, with~\eqref{eq:proof-thm-controllability-46},
            that
            \begin{equation*}
                \begin{cases}\displaystyle
                    \begin{array}{rclll}
                        \primetemps{y^{v}} - \Delta y^{v} & = & v &
                        \text{in} & Q,\\
                        \\
                        y^{v} & = & 0 & \text{on} & \Sigma,\\
                        \\
                        \condfinale{y^{v}} & = & 0 & \text{in} & \Omega,
                    \end{array}
                \end{cases}
            \end{equation*}
            that is to say that $y^{v}\in \Vbb$ is solution of the
            ill-posed backwards heat equation~\eqref{eq:initial-problem-2}.
        \item Now, let assume that the ill-posed backwards heat
            equation~\eqref{eq:initial-problem-2} admits a regular solution
            $z\in \Vbb$.

            Then, since $z\in \Vbb$ implies that $z$ is almost everywhere
            equal to a continuous function from $(0,T)$ to
            $\Ldeux{\Omega}$, we can, after possible modification on a set
            of zero measure, speak of the initial value $\displaystyle
            \condinitiale{z}\in\Ldeux{\Omega}$.

            Then, by choosing, for any $\veps > 0$,
            \begin{equation*}
                \begin{array}{rclll}
                    {y_{0}}_{\veps} & = & \condinitiale{z} & \text{in} &
                    \Omega,
                \end{array}
            \end{equation*}
            we obtain that the sequence
            ${\left({{y_{0}}_{\veps}}\right)}_{\veps}$, since constant, is
            bounded in $\Ldeux{\Omega}$.
    \end{enumerate}
\end{proof}
