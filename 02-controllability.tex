\section{Contrôlabilité de l'équation de la chaleur
rétrograde}\label{sec:controllability}

Dans la présente section, nous introduisons le point de vue par
contrôlabilité ici proposé. Lequel consiste, partant de la notion de
contrôlabilité introduite par J.-L.~Lions dans~\cite[222]{lions1}, à
interpréter le problème initial comme un problème inverse (nous disons
aussi un problème de contrôlabilité).

On établit que, lorsqu'elle existe, la solution de l'équation de la chaleur
rétrograde~\eqref{eq:initial-problem} est aussi solution du problème de
contrôlabilité. Ce qui permet alors d'établir une condition nécessaire et
suffisante caractérisant l'existence d'une solution régulière au
problème~\eqref{eq:initial-problem}.

On interprète donc~\eqref{eq:initial-problem} comme un problème inverse
dont l'objectif d'observation consiste en la condition finale
\begin{equation*}
    \begin{array}{rclll}
        \condfinale{z} & = & 0 & \text{dans} & \Omega.
    \end{array}
\end{equation*}

Plus précisément, on considère le système
\begin{equation*}\label{eq:inverse-problem-1}
    \begin{cases}\displaystyle
        \begin{array}{rclll}
            \primetemps{y} - \Delta y & = & v & \text{dans} & Q,\\
            \\
            y & = & 0 & \text{sur} & \Sigma,
        \end{array}
    \end{cases}
\end{equation*}
se posant le problème de trouver $\displaystyle y_{0}\in\Ldeux{\Omega}$ tel
que si
\begin{equation}\label{eq:inverse-problem-2}
    \begin{cases}\displaystyle
        \begin{array}{rclll}
            \primetemps{y} - \Delta y & = & v & \text{dans} & Q,\\
            \\
            y & = & 0 & \text{sur} & \Sigma,\\
            \\
            \condinitiale{y} & = & y_{0} & \text{dans} & \Omega,
        \end{array}
    \end{cases}
\end{equation}
alors
\begin{equation}\label{eq:inverse-problem-3}
    \begin{array}{rclll}
        \condfinale{y} & = & 0 & \text{dans} & \Omega.
    \end{array}
\end{equation}
On dit que~\eqref{eq:inverse-problem-2}\eqref{eq:inverse-problem-3}
constitue un problème de contrôlabilité exacte associé à l'équation de la
chaleur rétrograde~\eqref{eq:initial-problem}.

\begin{remarque}%
    Le problème de contrôlabilité est bien défini. En effet, il est bien
    connu que pour tous $v\in\Ldeux{Q}$ et
    $y_{0}\in\Ldeux{\Omega}$,~\eqref{eq:inverse-problem-2} est bien posé au
    sens de Hadamard, soit qu'il admet une solution unique
    \begin{equation*}\label{eq:well-posedness-heat-equation}
        y\!\left({v,y_{0}}\right)\in \Vbb := \Ldeux{0,T;
        H_{0}^{1}(\Omega)}\cap \Cscr\!\left({[0,T];\Ldeux{\Omega}}\right)
    \end{equation*}
    qui dépend continûment des données. On en déduit, qu'après modification
    éventuelle sur un ensemble de mesure nulle, la solution
    $y\!\left({v,y_{0}}\right)$ de~\eqref{eq:inverse-problem-2}
    se confond à une fonction continue de $[0,T]$ dans $\Ldeux{\Omega}$.

    Ainsi on peut bien parler de la valeur finale
    $y\!\left({\cdot,T;v,y_{0}}\right)$ de
    $y\!\left({v,y_{0}}\right)$ dans $\Omega$.
\end{remarque}

\begin{remarque}%
    Notons
    \begin{itemize}
        \item $y^{v}$ la solution unique de~\eqref{eq:inverse-problem-2};
        \item $y_{0}^{v}\in \Vbb \subset \Ldeux{Q}$ celle de
            \begin{equation*}\label{eq:null-initial-condition}
                \begin{cases}\displaystyle
                    \begin{array}{rclll}
                        \primetemps{y_{0}^{v}} - \Delta y_{0}^{v} & = & v &
                        \text{dans} & Q,\\
                        \\
                        y_{0}^{v} & = & 0 & \text{sur} & \Sigma,\\
                        \\
                        y_{0}^{v}\!\left({\cdot,0}\right) & = & 0 &
                        \text{dans} & \Omega,
                    \end{array}
                \end{cases}
            \end{equation*}
        \item et $y^{0}\in \Vbb \subset \Ldeux{Q}$ celle de
            \begin{equation}\label{eq:null-source-term}
                \begin{cases}\displaystyle
                    \begin{array}{rclll}
                        \primetemps{y^{0}} - \Delta y^{0} & = & 0 &
                        \text{dans} & Q,\\
                        \\
                        y^{0} & = & 0 & \text{sur} & \Sigma,\\
                        \\
                        y^{0}\!\left({\cdot,0}\right) & = & y_{0} &
                        \text{dans} & \Omega.
                    \end{array}
                \end{cases}
            \end{equation}
    \end{itemize}
    On a alors que l'application
    \begin{equation*}
        \left({v,y_{0}}\right)\longmapsto y^{v} = y_{0}^{v} + y^{0},
    \end{equation*}
    est linéaire continue de $\Ldeux{Q}\times \Ldeux{Q}$ dans $\Vbb\subset
    \Ldeux{Q}$.

    On en déduit que le problème de contrôlabilité
    exacte~\eqref{eq:inverse-problem-2}\eqref{eq:inverse-problem-3} est
    équivalent au suivant
    \begin{equation}\label{eq:inverse-problem-4}
        \begin{cases}\displaystyle
            \text{trouver}\ y_{0}\in \Ldeux{\Omega}\ \text{tel que:}\\
            \text{si}\ y^{0}\ \text{est solution
            de}~\eqref{eq:null-source-term},\ \text{alors}\\
            y^{0}\!\left({\cdot,T}\right) = -y_{0}^{v}\!\left({\cdot,
            T}\right)\quad \text{dans}\ \Omega.
        \end{cases}
    \end{equation}
\end{remarque}

% \begin{remarque}\label{rq:naturalassumption}%
%     Avec ce qui précède, il apparaît naturel de considérer, dans le cadre
%     du problème de contrôle, que c'est l'ensemble
%     \begin{equation*}
%         \left\{{%
%             (v,z)\in\Ascr:\ \condinitiale{z}\in\Ldeux{\Omega}
%         }\right\}
%     \end{equation*}
%     qui est non vide et non seulement $\Ascr$.
% \end{remarque}

On approche, par argument de densité, le problème de contrôlabilité
exacte~\eqref{eq:inverse-problem-4} par un problème, dit de contrôlabilité
approchée et en réponse duquel on a le résultat suivant.

\begin{proposition}\label{propo:controllability-result-1}%
    Lorsque la donnée initiale $y_{0}$ parcourt $\Ldeux{\Omega}$,
    l'ensemble
    \begin{equation*}\label{eq:controllability-result-1}
        E = \left\{{%
            y^{0}\!\left({\cdot,T}\right)\,;\ y_{0}\in\Ldeux{\Omega}
        }\right\},
    \end{equation*}
    décrit par les valeurs finales de la solution $y^{0}$
    de~\eqref{eq:null-source-term}, est dense dans $\Ldeux{\Omega}$.
\end{proposition}

\begin{proof}%
    On a clairement que $E$ est un sous-espace vectoriel de
    $\Ldeux{\Omega}$. D'où, par le Théorème de Hahn-Banach, $E$ est dense
    dans $\Ldeux{\Omega}$ si et seulement si $E^{\perp} = \{0\}$.

    Soit $k\in E^{\perp}$; on a
    \begin{equation*}\label{eq:proof-controllability-26}
        \forall\,y_{0}\in\Ldeux{\Omega},\qquad
        \scalaireohm{k}{y^{0}\!\left({\cdot,T}\right)} = 0.
    \end{equation*}
    Mais il vient de~\eqref{eq:null-source-term} que, pour toute fonction
    test $\vphi\in\Dscr(Q)$, on a:
    \begin{equation*}
        \begin{split}
            \scalaireq{\primetemps{y^{0}} - \Delta y^{0}}{\vphi} = 0 &\iff
            \scalaireq{\primetemps{y^{0}}}{\vphi} - \scalaireq{\Delta
            y^{0}}{\vphi} = 0\\
            &\iff \scalaireohm{%
                y^{0}\!\left({\cdot,T}\right)
            }{%
                \vphi\!\left({\cdot,T}\right)
            } - \scalaireohm{%
                y_{0}
            }{%
                \vphi\!\left({\cdot,0}\right)
            }\\
            &\qquad - \scalaireq{y^{0}}{\primetemps{\vphi}} - \scalaireq{y^{0}
            }{%
                \Delta \vphi
            } + \scalaires{y^{0}
            }{\deriveenormale{\vphi}
            }\\
            &\qquad\qquad - \scalaires{\deriveenormale{y^{0}}}{\vphi} = 0
        \end{split}
    \end{equation*}
    \ie{}
    \begin{equation}\label{eq:controllability-proof-30}
        \begin{split}
            \scalaireohm{%
                y^{0}\!\left({\cdot,T}\right)
            }{%
                \vphi\!\left({\cdot,T}\right)
            } &- \scalaireohm{%
                y_{0}
            }{%
                \vphi\!\left({\cdot,0}\right)
            } - \scalaireq{%
                y^{0}
            }{%
                \primetemps{\vphi}
            }\\
            &\qquad - \scalaireq{%
                y^{0}
            }{%
                \Delta\vphi
            } - \scalaires{%
                \deriveenormale{y^{0}}
            }{\vphi} = 0.
        \end{split}
    \end{equation}
    Choisissons dans ce qui précède $\vphi$ telle que
    \begin{equation}\label{eq:controllability-proof-31}
        \begin{cases}\displaystyle
            \begin{array}{rclll}
                -\primetemps{\vphi} - \Delta\vphi & = & 0 & \text{dans} &
                Q,\\
                \\
                \vphi & = & 0 & \text{sur} & \Sigma,\\
                \\
                \vphi\!\left({\cdot,T}\right) & = & k & \text{dans} &
                \Omega.
            \end{array}
        \end{cases}
    \end{equation}
    Alors il vient que~\eqref{eq:controllability-proof-30} équivaut à
    \begin{equation}\label{eq:controllability-proof-32}
        \scalaireohm{k}{y^{0}\!\left({\cdot,T}\right)} - \scalaireohm{%
            y_{0}
        }{%
            \vphi\!\left({\cdot,0}\right)
        } = 0,
    \end{equation}
    où
    \begin{equation*}
        k\in E^{\perp}\iff \scalaireohm{k}{y^{0}\!\left({\cdot,T}\right)} =
        0.
    \end{equation*}
    Ainsi~\eqref{eq:controllability-proof-32} devient
    \begin{equation}\label{eq:controllability-proof-33}
        \forall\,y_{0}\in\Ldeux{\Omega},\qquad \scalaireohm{y_{0}}{%
            \vphi\!\left({\cdot,0}\right)
        } = 0.
    \end{equation}
    Mais on peut encore choisir dans~\eqref{eq:controllability-proof-33},
    $\displaystyle y_{0} = \vphi\!\left({\cdot,0}\right)$ dans $\Omega$, et
    alors il suit
    \begin{equation*}
        \normecohm{\vphi\!\left({\cdot,0}\right)} = 0\qquad\ie\qquad
        \begin{array}{rclll}
            \vphi\!\left({\cdot,0}\right) & = & 0 & \text{dans} & \Omega.
        \end{array}
    \end{equation*}
    Ce qui amène, avec~\eqref{eq:controllability-proof-31}, que $\vphi$
    est solution de
    \begin{equation*}
        \begin{cases}\displaystyle
            \begin{array}{rclll}
                -\primetemps{\vphi} - \Delta\vphi & = & 0 & \text{dans} &
                Q,\\
                \\
                \vphi & = & 0 & \text{sur} & \Sigma,\\
                \\
                \vphi\!\left({\cdot,0}\right) & = & 0 & \text{dans} &
                \Omega,
            \end{array}
        \end{cases}
    \end{equation*}
    c'est-à-dire que $\vphi\equiv 0$ et par suite que
    \begin{equation*}
        \begin{array}{rclll}
            \vphi\!\left({\cdot,T}\right) & = & k = 0 & \text{dans} &
            \Omega.
        \end{array}
    \end{equation*}
    % On en déduit que
    % \begin{equation}
    %     \forall\,k\in E^{\perp},\qquad k = 0,
    % \end{equation}
    D'où on déduit bien que $E^{\perp} = \{0\}$ et donc que $E$ est dense
    dans $\Ldeux{\Omega}$.
\end{proof}

Le résultat suivant est alors immédiat.

\begin{corollaire}\label{coro:controllability-result-2}%
    Pour tous $\veps > 0$ et $\kappa\in\Ldeux{\Omega}$, il existe
    ${y_{0}}_{\veps}\in\Ldeux{\Omega}$ tel que la solution $\displaystyle
    y_{\veps}^{0}\in \Vbb$ de
    % \begin{equation}\label{eq:controllability-result-2}
    %     y_{\veps}^{0}\in \Vbb
    % \end{equation}
    % de
    \begin{equation}\label{eq:controllability-result-3}
        \begin{cases}\displaystyle
            \begin{array}{rclll}
                \primetemps{y_{\veps}^{0}} - \Delta y_{\veps}^{0} & = & v &
                \text{dans} & Q,\\
                \\
                y_{\veps}^{0} & = & 0 & \text{sur} & \Sigma,\\
                \\
                y_{\veps}^{0}\!\left({\cdot,0}\right) & = & {y_{0}}_{\veps}
                & \text{dans} & \Omega,
            \end{array}
        \end{cases}
    \end{equation}
    satisfait
    \begin{equation*}\label{eq:controllability-result-4}
        \normeohm{y_{\veps}^{0}\!\left({\cdot,T}\right) - \kappa} < \veps.
    \end{equation*}
\end{corollaire}

D'où il découle aussi que

\begin{corollaire}\label{coro:controllability-result-3}%
    Pour tous $v\in\Ldeux{Q}$ et $\veps > 0$, il existe ${y_{0}}_{\veps}\in
    \Ldeux{\Omega}$ tel que la solution $\displaystyle
    y_{\veps}^{v}\in \Vbb$ de
    % \begin{equation}\label{eq:controllability-result-7}
    %     y_{\veps}^{v}\in \Vbb
    % \end{equation}
    % de
    \begin{equation}\label{eq:controllability-result-5}
        \begin{cases}\displaystyle
            \begin{array}{rclll}
                \primetemps{y_{\veps}^{v}} - \Delta y_{\veps}^{v} & = & v &
                \text{dans} & Q,\\
                \\
                y_{\veps}^{v} & = & 0 & \text{sur} & \Sigma,\\
                \\
                y_{\veps}^{v}\!\left({\cdot,0}\right) & = & {y_{0}}_{\veps}
                & \text{dans} & \Omega,
            \end{array}
        \end{cases}
    \end{equation}
    satisfait
    \begin{equation*}\label{eq:controllability-result-6}
        \normeohm{y_{\veps}^{v}\!\left({\cdot,T}\right)} < \veps.
    \end{equation*}
\end{corollaire}

\begin{proof}%
    Soient $v\in\Ldeux{Q}$ et $\veps > 0$. On sait qu'il existe un unique
    $y_{0}^{v}\in \Vbb$, égal presque partout à une fonction continue de
    $[0,T]$ dans $\Ldeux{\Omega}$, solution de
    \begin{equation*}\label{eq:controllability-proof-39}
        \begin{cases}\displaystyle
            \begin{array}{rclll}
                \primetemps{y_{0}^{v}} - \Delta y_{0}^{v} & = & v &
                \text{dans} & Q,\\
                \\
                y_{0}^{v} & = & 0 & \text{sur} & \Sigma,\\
                \\
                y_{0}^{v}\!\left({\cdot,0}\right) & = & 0 & \text{dans} &
                \Omega.
            \end{array}
        \end{cases}
    \end{equation*}
    Il suit, avec le
    Corollaire~\ref{coro:controllability-result-2}, qu'il existe bien
    ${y_{0}}_{\veps}\in\Ldeux{\Omega}$ tel que la solution
    $y_{\veps}^{0}\in \Vbb$ de~\eqref{eq:controllability-result-3} vérifie
    \begin{equation*}\label{eq:controllability-proof-40}
        \normeohm{%
            y_{\veps}^{0}\!\left({\cdot,T}\right) - \left({%
                -y_{0}^{v}\!\left({\cdot,T}\right)
            }\right)
        } < \veps.
    \end{equation*}
    Ce qui permet bien de conclure que $\displaystyle y_{\veps}^{v} =
    \left({%
        y_{0}^{v} + y_{\veps}^{0}
    }\right) \in \Vbb$ est solution unique
    de~\eqref{eq:controllability-result-5} avec
    \begin{equation*}
        \normeohm{y_{\veps}^{v}\!\left({\cdot,T}\right)} = \normeohm{%
            y_{0}^{\veps}\!\left({\cdot,T}\right) +
            y_{0}^{v}\!\left({\cdot,T}\right)
        } < \veps.
    \end{equation*}
\end{proof}

De plus, on a le théorème suivant caractérisant l'existence d'une solution
régulière à l'équation de la chaleur rétrograde.

\begin{theoreme}\label{thm:controllability-result-5}%
    Soit $v\in\Ldeux{Q}$. L'équation de la chaleur rétrograde
    \begin{equation}\label{eq:initial-problem-2}
        \begin{cases}\displaystyle
            \begin{array}{rclll}
                \primetemps{z} - \Delta z & = & v & \text{dans} & Q,\\
                \\
                z & = & 0 & \text{sur} & \Sigma,\\
                \\
                z(\cdot,T) & = & 0 & \text{dans} & \Omega,
            \end{array}
        \end{cases}
    \end{equation}
    admet une solution régulière $z\in \Vbb$, si et seulement si la
    suite $\displaystyle {\left({{y_{0}}_{\veps}}\right)}_{\veps}$ est
    bornée dans $\Ldeux{\Omega}$.
\end{theoreme}

\begin{proof}%
    \begin{enumerate}
        \item Soit $\veps > 0$. D'après
            le~Corollaire~\ref{coro:controllability-result-2}, il existe
            ${y_{0}}_{\veps}\in \Ldeux{\Omega}$ tel que $\displaystyle
            y_{\veps}^{v}\in \Vbb$ soit solution de
            \begin{equation}\label{eq:proof-thm-controllability-42}
                \begin{cases}\displaystyle
                    \begin{array}{rclll}
                        \primetemps{y_{\veps}^{v}} - \Delta y_{\veps}^{v} &
                        = & v & \text{dans} & Q,\\
                        \\
                        y_{\veps}^{v} & = & 0 & \text{sur} & \Sigma,\\
                        \\
                        y_{\veps}^{v}\!\left({\cdot,0}\right) & = &
                        {y_{0}}_{\veps} & \text{dans} & \Omega,
                    \end{array}
                \end{cases}
            \end{equation}
            avec l'estimation
            \begin{equation*}\label{eq:proof-thm-controllability-43}
                \normeohm{y_{\veps}^{v}\!\left({\cdot, T}\right)} < \veps.
            \end{equation*}
            On génère ainsi des suites
            \begin{equation*}
                {\left({{y_{0}}_{\veps}}\right)}_{\veps}\subset
                \Ldeux{\Omega}\qquad\text{et}\qquad
                {\left({y_{\veps}^{v}}\right)}_{\veps}\subset \Vbb.
            \end{equation*}
            Supposons alors que la suite
            ${\left({{y_{0}}_{\veps}}\right)}_{\veps}$ bornée dans
            $\Ldeux{\Omega}$. Il suit que, le
            problème~\eqref{eq:proof-thm-controllability-42} de Dirichlet
            homogène pour l'équation de la chaleur, étant bien posé, que la
            suite ${\left({y_{\veps}^{v}}\right)}_{\veps}$ est bornée dans
            $\Vbb$, et donc encore dans $\Ldeux{Q}$.

            On en déduit qu'on peut extraire, de
            ${\left({{y_{0}}_{\veps}}\right)}_{\veps}$ et
            ${\left({y_{\veps}^{v}}\right)}_{\veps}$, des sous-suites
            encore notées de même qui convergent faiblement dans
            $\Ldeux{\Omega}$ et $\Vbb$, respectivement.

            Soit qu'il existe $\displaystyle y_{0}\in \Ldeux{\Omega}$ et
            $\displaystyle y^{v}\in \Vbb$, tels que
            \begin{equation*}\label{eq:proof-thm-controllability-44}
                \begin{cases}\displaystyle
                    \begin{array}{rcllll}
                        {y_{0}}_{\veps} & \longrightarrow & y_{0} &
                        \text{dans} & \Ldeux{\Omega} & \text{faible},\\
                        \\
                        y_{\veps}^{v} & \longrightarrow & y^{v} &
                        \text{dans} & \Vbb & \text{faible}.
                    \end{array}
                \end{cases}
            \end{equation*}
            Mais alors, on a d'une part que
            \begin{equation*}\label{eq:proof-thm-controllability-45}
                \normeohm{y_{\veps}^{v}\!\left({\cdot,T}\right)} < \veps
                \quad\text{et}\quad
                \begin{array}{rcllll}
                    y_{\veps}^{v} & \longrightarrow & y^{v} & \text{dans} &
                    \Vbb & \text{faible}
                \end{array}
            \end{equation*}
            entraînent, $y_{\veps}^{v}$ étant presque partout égale à une
            fonction continue de $[0,T]$ dans $\Ldeux{\Omega}$, que
            \begin{equation}\label{eq:proof-thm-controllability-46}
                \begin{array}{rclll}
                    y^{v}\!\left({\cdot, T}\right) & = & 0 & \text{dans} &
                    \Omega.
                \end{array}
            \end{equation}
            D'autre part, pour tout $\vphi\in\Cscr(Q)$, on a:
            \begin{equation*}
                \begin{split}
                    {} & \primetemps{y_{\veps}^{v}} - \Delta y_{\veps}^{v}
                    = v\ \text{dans}\ Q \iff \scalaireq{%
                        \primetemps{y_{\veps}^{v}} - \Delta y_{\veps}^{v}
                    }{\vphi} = \scalaireq{v}{\vphi}\\
                    % &\qquad \iff \scalaireohm{%
                    %     \condfinale{y_{\veps}^{v}}
                    % }{%
                    %     \condfinale{\vphi}
                    % } - \scalaireohm{%
                    %     \condinitiale{y_{\veps}^{v}}
                    % }{%
                    %     \condinitiale{\vphi}
                    % }\\
                    % &\qquad\qquad - \scalaireq{%
                    %     y_{\veps}^{v}
                    % }{\primetemps{\vphi}} - \scalaireq{%
                    %     y_{\veps}^{v}
                    % }{%
                    %     \Delta\vphi{}
                    % } - \scalaires{%
                    %     \deriveenormale{y_{\veps}^{v}}
                    % }{%
                    %     \vphi{}
                    % }\\
                    % &\qquad\qquad + \scalaires{%
                    %     y_{\veps}^{v}
                    % }{%
                    %     \deriveenormale{\vphi}
                    % } = \scalaireq{v}{\vphi}\\
                    &\qquad\iff \scalaireohm{%
                        \condfinale{y_{\veps}^{v}}
                    }{%
                        \condfinale{\vphi}
                    } - \scalaireohm{%
                        {y_{0}}_{\veps}
                    }{%
                        \condinitiale{\vphi}
                    }\\
                    &\qquad\qquad - \scalaireq{%
                        y_{\veps}^{v}
                    }{%
                        \primetemps{\vphi}
                    } - \scalaireq{%
                        y_{\veps}^{v}
                    }{%
                        \Delta\vphi{}
                    } - \scalaires{%
                        \deriveenormale{y_{\veps}^{v}}
                    }{%
                        \vphi{}
                    }\\
                    &\qquad\qquad\qquad = \scalaireq{v}{\vphi},
                \end{split}
            \end{equation*}
            soit, par passage à la limite,
            \begin{equation*}
                \begin{split}
                    {} & \scalaireohm{%
                        \condfinale{y^{v}}
                    }{%
                        \condfinale{\vphi}
                    } - \scalaireohm{%
                        y_{0}
                    }{%
                        \condinitiale{\vphi}
                    } - \scalaireq{%
                        y^{v}
                    }{%
                        \primetemps{\vphi}
                    }\\
                    &\qquad\qquad - \scalaireq{%
                        y^{v}
                    }{%
                        \Delta\vphi{}
                    } - \scalaires{%
                        \deriveenormale{y^{v}}
                    }{%
                        \vphi{}
                    } = \scalaireq{v}{\vphi}\\
                    &\qquad \iff \scalaireohm{%
                        \condinitiale{y^{v}} - y_{0}
                    }{%
                        \condinitiale{\vphi}
                    } + \scalaireq{%
                        \primetemps{y^{v}} - \Delta y^{v}
                    }{%
                        \vphi{}
                    }\\
                    &\qquad\qquad - \scalaires{%
                        y^{v}
                    }{%
                        \deriveenormale{\vphi}
                    } = \scalaireq{v}{\vphi}.
                \end{split}
            \end{equation*}
            Cette dernière égalité valant pour tout $\vphi\in\Cscr(Q)$, il
            en découle que
            \begin{equation*}
                \begin{cases}\displaystyle
                    \begin{array}{rclll}
                        \primetemps{y^{v}} - \Delta y^{v} & = & v &
                        \text{dans} & Q,\\
                        \\
                        y^{v} & = & 0 & \text{sur} & \Sigma,\\
                        \\
                        \condinitiale{y^{v}} & = & y_{0} & \text{dans} &
                        \Omega,
                    \end{array}
                \end{cases}
            \end{equation*}
            soit, avec~\eqref{eq:proof-thm-controllability-46}, que
            \begin{equation*}
                \begin{cases}\displaystyle
                    \begin{array}{rclll}
                        \primetemps{y^{v}} - \Delta y^{v} & = & v &
                        \text{dans} & Q,\\
                        \\
                        y^{v} & = & 0 & \text{sur} & \Sigma,\\
                        \\
                        \condfinale{y^{v}} & = & 0 & \text{dans} & \Omega,
                    \end{array}
                \end{cases}
            \end{equation*}
            c'est-à-dire que $y^{v}\in \Vbb$ est solution de l'équation de
            la chaleur rétrograde~\eqref{eq:initial-problem-2}.
        \item On suppose à présent que l'équation de la chaleur
            rétrograde~\eqref{eq:initial-problem-2} admet une solution
            régulière $z\in \Vbb$.

            Alors, comme $z\in \Vbb$ implique que $z$ est presque partout
            égale à une fonction continue de $[0,T]$ dans $\Ldeux{\Omega}$,
            on peut, après modification éventuelle sur un ensemble de
            mesure nulle, parler de sa valeur initiale $\displaystyle
            \condinitiale{z}\in\Ldeux{\Omega}$.

            Et donc, posant pour tout $\veps > 0$,
            \begin{equation*}
                \begin{array}{rclll}
                    {y_{0}}_{\veps} & = & \condinitiale{z} & \text{dans} &
                    \Omega,
                \end{array}
            \end{equation*}
            on obtient que la suite
            ${\left({{y_{0}}_{\veps}}\right)}_{\veps}$, puisque constante,
            est bornée dans $\Ldeux{\Omega}$.
    \end{enumerate}
\end{proof}
